% Usage :
% \repereglobal{x}{y}{rotation}{echelle}
% \poutre{x1}{y1}{x2}{y2}
% \force{x1}{y1}{x2}{y2}{couleur}
% \moment{x}{y}{rayon}{angleDeb}{angleFin}{couleur}
% \point{x}{y}
% \cote{x1}{y1}{x2}{y2}{label}
% \cotep{x1}{y1}{x2}{y2}{position}{label}
% \appuibati{x}{y}{rotation}{echelle}
% \articulbati{x}{y}{rotation}{echelle}
% \encastrbati{x}{y}{rotation}{echelle}
% \upe{x}{y}{rotation}{echelle}
% \ipe{x}{y}{rotation}{echelle}
% \forceRep{xleft}{yleft}{xright}{yright}{text}
% \navierAxes{xmin}{ymin}{xmax}{ymax}{yaxe}{sigaxe}
% \navierStress{x0}{y0}{x1}{y1}{sigaxe}
% \navierStress{x0}{y0}{x1}{y1}{sigaxe}
% \velo{x}{y}{rotation}{echelle}
% ======================================================

\usetikzlibrary{arrows}
\usetikzlibrary{bending}
\usetikzlibrary{shadings}
\usetikzlibrary{decorations.pathmorphing}

\pgfdeclarepatternformonly{spaced north east lines}{\pgfqpoint{-1pt}{-1pt}}{\pgfqpoint{10pt}{10pt}}{\pgfqpoint{9pt}{9pt}}%
{
    \pgfsetlinewidth{1pt}
    \pgfpathmoveto{\pgfqpoint{0pt}{0pt}}
    \pgfpathlineto{\pgfqpoint{9.1pt}{9.1pt}}
    \pgfusepath{stroke}
}

\pgfdeclarepatternformonly{clean north east lines}{\pgfqpoint{-1pt}{-1pt}}{\pgfqpoint{5pt}{5pt}}{\pgfqpoint{5pt}{5pt}}%
{
    \pgfsetlinewidth{0.6pt}
    \pgfpathmoveto{\pgfqpoint{0pt}{0pt}}
    \pgfpathlineto{\pgfqpoint{5pt}{5pt}}
    \pgfusepath{stroke}
}

%\cercletrigo{rayon}
\newcommand{\cercletrigo}[2]%
{
\draw [black,line width=1pt,] (0,0) circle [radius=#1];
\draw[>=angle 60,->,line width=1pt, black, line cap=round] ({-#1-#2},0) -- ({#1+#2},0);
\draw[>=angle 60,->,line width=1pt, black, line cap=round] (0,{-#1-#2}) -- (0,{#1+#2});
}


% \SymbolVq{x0}{y0}{L}{H}{r}{color}{$L$}{$q\times L$}
\newcommand{\SymbolVq}[8]%
{
\draw ({#1-#3/2},#2) node[fill=white,above]{#7};
\draw ({#1-#3},{#2-#4/2}) node[fill=white,right]{#8};
\draw[>=angle 60,->,dashed,line width=2.5pt,line cap=round, #6] (#1, #2) -- ++({-(#3-#5)},0) arc (90:180:#5) -- ++(0,{-(#4-#5)});
}

\newcommand{\SymbolVqSmaller}[8]%
{
\draw ({#1-#3/2},#2) node[fill=white,above]{\tiny$\leftarrow$ #7, $\downarrow$ #8};
\draw[>=angle 60,->,dashed,line width=2.5pt,line cap=round, #6] (#1, #2) -- ++({-(#3-#5)},0) arc (90:180:#5) -- ++(0,{-(#4-#5)});
}

% \forceEquiv{x0}{y0}{x1}{y1}{textforce}{textpos}
\newcommand{\forceEquiv}[6]%
{
% boîte avec des bords ondulés
\draw[line width=1pt,line cap=round, decorate, decoration={random steps, amplitude=0.05 cm, segment length=0.4 cm}, gray, dotted]
({#1-0.6}, {#2-0.2}) -- ++({#3-#1+1.2}, 0) -- ++(0, {#4-#2+0.4}) -- ++({-#3+#1-1.2}, 0) -- cycle;
\draw[>=angle 60,->,line width=1.5pt, line cap=round, blue, dotted] ({(#1+#3)*0.5},{#2+1}) -- ({(#1+#3)*0.5},#2);
\draw ({(#1+#3)/2},{#2+1}) node[above]{#5};
%
\draw[line width=0.5pt,gray] (#1,{#2+0.2}) -- (#1,{#2+0.6});
\draw[>=angle 45,<->,line width=1pt,gray] (#1,{#2+0.4}) -- ({(#1+#3)/2},{#2+0.4});
\draw ({#1+(#3-#1)/4},{#2+0.4}) node[above]{#6};
%
\draw[line width=0.5pt,gray] (#3,{#2+0.2}) -- (#3,{#2+0.6});
\draw[>=angle 45,<->,line width=1pt,gray] ({(#1+#3)/2},{#2+0.4}) -- (#3,{#2+0.4});
\draw ({#1+3*(#3-#1)/4},{#2+0.4}) node[above]{#6};
%
\draw ({#3+0.8}, {#2-0.2}) node[fill=white, left]{\small Equivalence};
}

% \forceEquivSmaller{x0}{y0}{x1}{y1}{textforce}{textpos}
\newcommand{\forceEquivSmaller}[6]%
{
% boîte avec des bords ondulés
\draw[line width=1pt,line cap=round, decorate, decoration={random steps, amplitude=0.05 cm, segment length=0.4 cm}, gray, dotted]
({#1-0.6}, {#2-0.2}) -- ++({#3-#1+1.2}, 0) -- ++(0, {#4-#2+0.2}) -- ++({-#3+#1-1.2}, 0) -- cycle;
\draw[>=angle 60,->,line width=1.5pt, line cap=round, blue, dotted] ({(#1+#3)*0.5},{#2+1}) -- ({(#1+#3)*0.5},#2);
\draw ({(#1+#3)/2},{#2+1}) node[above]{#5};
%
\draw[line width=0.5pt,gray] (#1,{#2+0.2}) -- (#1,{#2+0.6});
\draw[>=angle 45,<->,line width=1pt,gray] (#1,{#2+0.4}) -- ({(#1+#3)/2},{#2+0.4});
\draw ({#1+(#3-#1)/4},{#2+0.4}) node[above]{#6};
%
\draw[line width=0.5pt,gray] (#3,{#2+0.2}) -- (#3,{#2+0.6});
\draw[>=angle 45,<->,line width=1pt,gray] ({(#1+#3)/2},{#2+0.4}) -- (#3,{#2+0.4});
\draw ({#1+3*(#3-#1)/4},{#2+0.4}) node[above]{#6};
%
\draw ({#3+0.8}, {#2-0.2}) node[fill=white, left]{\small Equiv.};
}

\newcommand{\debfin}[4]%
{
\draw[decorate, decoration={snake, pre length=5mm, post length=5mm}, line width=1pt] (#1,#2) -- (#3,#4);
\draw [black, fill=darkgray] (#1,#2) circle [radius=0.1];
\draw [black, fill=gray] (#3,#4) circle [radius=0.1];
}

% \upe{x}{y}{rotation}{echelle}
\newcommand{\upe}[4]%
{
\begin{scope}[xshift=#1cm, yshift=#2cm, rotate=#3, scale=#4]
\draw[draw=black, line width=1] (0,0) -- (2,0);
\draw[draw=black, line width=1] (0,0) -- (0,1);
\draw[draw=black, line width=1] (0,1) -- (0.2,1);
\draw[draw=black, line width=1] (0.2,1) -- (0.2,0.4);
\draw[draw=black, line width=1] (0.2,0.4) arc (180:270:0.2);
\draw[draw=black, line width=1] (0.4,0.2) -- (1.6,0.2);
\draw[draw=black, line width=1] (1.6,0.2) arc (270:360:0.2);
\draw[draw=black, line width=1] (1.8,0.4) -- (1.8,1);
\draw[draw=black, line width=1] (1.8,1) -- (2,1);
\draw[draw=black, line width=1] (2,1) -- (2,0);
\end{scope}
}

% \ipe{x}{y}{rotation}{echelle}
\newcommand{\ipe}[4]%
{
\begin{scope}[xshift=#1cm, yshift=#2cm, rotate=#3, scale=#4]
\draw[draw=black, line width=1] (0,0) -- (1,0);
\draw[draw=black, line width=1] (0,0) -- (0,0.2);
\draw[draw=black, line width=1] (0,0.2) -- (0.2,0.2);
\draw[draw=black, line width=1] (0.2,0.2) arc (270:360:0.2);
\draw[draw=black, line width=1] (0.4,0.4) -- (0.4,1.6);
\draw[draw=black, line width=1] (0.4,1.6) arc (0:90:0.2);
\draw[draw=black, line width=1] (0.2,1.8) -- (0,1.8);
\draw[draw=black, line width=1] (0,1.8) -- (0,2);
\draw[draw=black, line width=1] (0,2) -- (1,2);
\draw[draw=black, line width=1] (1,2) -- (1,1.8);
\draw[draw=black, line width=1] (1,1.8) -- (0.8,1.8);
\draw[draw=black, line width=1] (0.8,1.8) arc (90:180:0.2);
\draw[draw=black, line width=1] (0.6,1.6) -- (0.6,0.4);
\draw[draw=black, line width=1] (0.6,0.4) arc (180:270:0.2);
\draw[draw=black, line width=1] (0.8,0.2) -- (1,0.2);
\draw[draw=black, line width=1] (1,0.2) -- (1,0);
\end{scope}
}

% \forceRep{xleft}{yleft}{xright}{yright}{text}
\newcommand{\forceRep}[5]%
{
\draw[line width=1pt, blue] (#1,#4) -- (#3,#4);
\draw[>=angle 60,->,line width=1pt, blue, line cap=round] (#1,#4) -- (#1,#2);
\draw[>=angle 60,->,line width=1pt, blue, line cap=round] ({#1+(#3-#1)/4},#4) -- ({#1+(#3-#1)/4},#2);
\draw[>=angle 60,->,line width=1pt, blue, line cap=round] ({(#1+#3)/2},#4) -- ({(#1+#3)/2},#2);
\draw[>=angle 60,->,line width=1pt, blue, line cap=round] ({#3-(#3-#1)/4},#4) -- ({#3-(#3-#1)/4},#2);
\draw[>=angle 60,->,line width=1pt, blue, line cap=round] (#3,#4) -- (#3,#2);
\draw ({(#1+#3)/2},#4) node[above]{#5};
}

\newcommand{\forceRepMini}[5]%
{
\draw[line width=1pt, blue] (#1,#4) -- (#3,#4);
\draw[>=angle 60,->,line width=1pt, blue, line cap=round] (#1,#4) -- (#1,#2);
\draw[>=angle 60,->,line width=1pt, blue, line cap=round] ({#1+(#3-#1)/3},#4) -- ({#1+(#3-#1)/3},#2);
\draw[>=angle 60,->,line width=1pt, blue, line cap=round] ({#3-(#3-#1)/3},#4) -- ({#3-(#3-#1)/3},#2);
\draw[>=angle 60,->,line width=1pt, blue, line cap=round] (#3,#4) -- (#3,#2);
\draw ({(#1+#3)/2},#4) node[above]{#5};
}

% \flashForce{x}{y1}{y2}{w}
\newcommand{\flashForce}[4]%
{
\draw[line width=1pt, blue] (#1,#2) -- ({#1-#4},{0.5*(#2+#3)-#4});
\draw[line width=1pt, blue] ({#1-#4},{0.5*(#2+#3)-#4}) -- ({#1+#4},{0.5*(#2+#3)+#4});
\draw[>=angle 60,->,line width=1pt, blue] ({#1+#4},{0.5*(#2+#3)+#4}) -- (#1,#3);
}


% \navierAxes{xmin}{ymin}{xmax}{ymax}{yaxe}{sigaxe}
\newcommand{\navierAxes}[6]%
{
\draw[>=latex,->,thin, black] (#1,#5) -- (#3,#5);
\draw[>=latex,->,thin, black] (#6,#2) -- (#6,#4);
\draw (#3,#5) node[right]{$\sigma$};
\draw (#6,#4) node[above]{$Y$};
}

% \navierStress{x0}{y0}{x1}{y1}{sigaxe}
\newcommand{\navierStress}[5]%
{
\draw[line width=1pt, blue] (#1,#2) -- (#3,#4);
\draw[>=latex,->,line width=0.5pt, blue] (#5,#2) --(#1,#2);
\draw[>=latex,->,line width=0.5pt, blue] (#5,{#2+(#4-#2)*0.25}) -- ({#1+(#3-#1)*0.25},{#2+(#4-#2)*0.25});
\draw[>=latex,->,line width=0.5pt, blue] (#5,{#2+(#4-#2)*0.75}) -- ({#1+(#3-#1)*0.75},{#2+(#4-#2)*0.75});
\draw[>=latex,->,line width=0.5pt, blue] (#5,#4) -- (#3,#4);
}

% \navierStress{x0}{y0}{x1}{y1}{sigaxe}{nbFleches}
\newcommand{\navierStressCustom}[6]%
{
\draw[line width=1pt, blue] (#1,#2) -- (#3,#4);
\foreach \i in {1,...,#6}
{
    \draw[>=latex,->,line width=0.5pt, blue] (#5,{#2+(\i-1)*(#4-#2)/(#6-1)}) -- ({#1+(\i-1)*(#3-#1)/(#6-1)},{#2+(\i-1)*(#4-#2)/(#6-1)});
}
\draw[>=latex,->,line width=0.5pt, blue] (#5,#4) -- (#3,#4);
}

\newcommand{\navierStressOpt}[6]%
{
\draw[line width=1pt, line cap=round, #6] (#1,#2) -- (#3,#4);
\draw[>=latex,->,line width=0.5pt, #6] (#5,#2) --(#1,#2);
\draw[>=latex,->,line width=0.5pt, #6] (#5,{#2+(#4-#2)*0.25}) -- ({#1+(#3-#1)*0.25},{#2+(#4-#2)*0.25});
\draw[>=latex,->,line width=0.5pt, #6] (#5,{#2+(#4-#2)*0.75}) -- ({#1+(#3-#1)*0.75},{#2+(#4-#2)*0.75});
\draw[>=latex,->,line width=0.5pt, #6] (#5,#4) -- (#3,#4);
}

% \navierStressMid{x0}{y0}{x1}{y1}{sigaxe}
\newcommand{\navierStressMid}[5]%
{
\draw[line width=1pt, blue] (#1,#2) -- (#3,#4);
\draw[>=latex,->,line width=0.5pt, blue] (#5,#2) --(#1,#2);
\draw[>=latex,->,line width=0.5pt, blue] (#5,{#2+(#4-#2)*0.25}) -- ({#1+(#3-#1)*0.25},{#2+(#4-#2)*0.25});
\draw[>=latex,->,line width=0.5pt, blue] (#5,{#2+(#4-#2)*0.5}) -- ({#1+(#3-#1)*0.5},{#2+(#4-#2)*0.5});
\draw[>=latex,->,line width=0.5pt, blue] (#5,{#2+(#4-#2)*0.75}) -- ({#1+(#3-#1)*0.75},{#2+(#4-#2)*0.75});
\draw[>=latex,->,line width=0.5pt, blue] (#5,#4) -- (#3,#4);
}

\newcommand{\navierStressXup}[5]%
{
\draw[line width=1pt, line cap=round, blue] (#1,#2) -- (#3,#4);
\draw[>=latex,->,line width=0.5pt, blue] (#5,#2) -- (#1,#2);
%\draw[>=latex,->,line width=0.5pt, blue] (#5,{#2+(#4-#2)*0.25}) -- ({#1+(#3-#1)*0.25},{#2+(#4-#2)*0.25});
\draw[>=latex,->,line width=0.5pt, blue] (#5,{#2+(#4-#2)*0.5}) -- ({#1+(#3-#1)*0.5},{#2+(#4-#2)*0.5});
\draw[>=latex,->,line width=0.5pt, blue] (#5,{#2+(#4-#2)*0.75}) -- ({#1+(#3-#1)*0.75},{#2+(#4-#2)*0.75});
\draw[>=latex,->,line width=0.5pt, blue] (#5,#4) -- (#3,#4);
}

\newcommand{\navierStressXdown}[5]%
{
\draw[line width=1pt, line cap=round, blue] (#1,#2) -- (#3,#4);
\draw[>=latex,->,line width=0.5pt, blue] (#5,#2) -- (#1,#2);
\draw[>=latex,->,line width=0.5pt, blue] (#5,{#2+(#4-#2)*0.25}) -- ({#1+(#3-#1)*0.25},{#2+(#4-#2)*0.25});
\draw[>=latex,->,line width=0.5pt, blue] (#5,{#2+(#4-#2)*0.5}) -- ({#1+(#3-#1)*0.5},{#2+(#4-#2)*0.5});
%\draw[>=latex,->,line width=0.5pt, blue] (#5,{#2+(#4-#2)*0.75}) -- ({#1+(#3-#1)*0.75},{#2+(#4-#2)*0.75});
\draw[>=latex,->,line width=0.5pt, blue] (#5,#4) -- (#3,#4);
}

%\repereglobal{x}{y}{rotation}{echelle}
\newcommand{\repereglobal}[4]%
{
\begin{scope}[xshift=#1cm, yshift=#2cm, rotate=#3, scale=#4]
\draw[>=latex,->,line width=1pt, black] (0,0) -- (1,0);
\draw[>=latex,->,line width=1pt, black] (0,0) -- (0,1);
\draw[>=latex,->,line width=1pt, black] (0.3535,-0.3535) arc (-45:135:0.5);
\draw[black] (0.2,0.2) node[above right]{$+$};
\draw[black] (1,0) node[below]{$\vec{X}$};
\draw[black] (0,1) node[left]{$\vec{Y}$};
\end{scope}
}

\newcommand{\grandrepere}[5]%
{
\begin{scope}[xshift=#1cm, yshift=#2cm, rotate=#3, scale=#4]
\draw[>=latex,->,line width=1pt,#5] (0,0) -- (1,0);
\draw[>=latex,->,line width=1pt,#5] (0,0) -- (0,1);
\draw[>=latex,->,line width=1pt,#5] (0.3535,-0.3535) arc (-45:135:0.5);
\draw[#5] (0.2,0.2) node[above right]{$+$};
\draw[#5] (1,0) node[below]{$\vec{X}$};
\draw[#5] (0,1) node[left]{$\vec{Y}$};
\end{scope}
}

%\repereZY{x}{y}{rotation}{echelle}
\newcommand{\repereZY}[4]%
{
\begin{scope}[xshift=#1cm, yshift=#2cm, rotate=#3, scale=#4]
\draw[>=latex,->,thin, black] (1,0) -- (-1,0);
\draw[>=latex,->,thin, black] (0,-1) -- (0,1);
\draw (-1,0) node[left]{$Z$};
\draw (0,1) node[above]{$Y$};
\end{scope}
}

%\repereZY{x}{y}{rotation}{echelle}{aspect ratio}
\newcommand{\repereZYAR}[5]%
{
\begin{scope}[xshift=#1cm, yshift=#2cm, rotate=#3, scale=#4]
\draw[>=latex,->,thin, black] (1,0) -- (-1,0);
\draw[>=latex,->,thin, black] (0,-#5) -- (0,#5);
\draw (-1,0) node[left]{$Z$};
\draw (0,#5) node[above]{$Y$};
\end{scope}
}

%\reperelocaltext{x}{y}{rotation}{echelle}
\newcommand{\reperelocaltext}[4]%
{
\begin{scope}[xshift=#1cm, yshift=#2cm, rotate=#3, scale=#4]
\draw[>=latex,->,thin, black] (0,0) -- (1,0);
\draw[>=latex,->,thin, black] (0,0) -- (0,1);
\draw (1,0) node[xshift=2pt]{\tiny$x$};
\draw (0,1) node[xshift=3pt]{\tiny$y$};
\end{scope}
}

%\reperelocal{x}{y}{rotation}{echelle}
\newcommand{\reperelocal}[4]%
{
\begin{scope}[xshift=#1cm, yshift=#2cm, rotate=#3, scale=#4]
\draw[>=latex,->,thin, black] (0,0) -- (1,0);
\draw[>=latex,->,thin, black] (0,0) -- (0,1);
\end{scope}
}

\newcommand{\reperelocalShiftOnly}[3]%
{
\begin{scope}[xshift=#1cm, yshift=#2cm, rotate=#3]
\draw[>=latex,->,thin, shift only, black] (0,0) -- (0.5,0);
\draw[>=latex,->,thin, shift only, black] (0,0) -- (0,0.5);
\end{scope}
}

\newcommand{\poutre}[4]%
{
\draw[line width=2.5pt, line cap=round, black] (#1,#2) -- (#3,#4);
}

\newcommand{\poutreOpt}[5]%
{
\draw[line width=2.5pt, line cap=round, #5] (#1,#2) -- (#3,#4);
}

\newcommand{\poutreComp}[4]%
{
\draw[line width=3.0pt, line cap=round, red] (#1,#2) -- (#3,#4);
}

\newcommand{\poutreTens}[4]%
{
\draw[line width=1.0pt, line cap=round, blue] (#1,#2) -- (#3,#4);
}

\newcommand{\poutreZero}[4]%
{
\draw[line width=1.0pt, line cap=round, dotted, black] (#1,#2) -- (#3,#4);
}

\newcommand{\force}[5]%
{
\draw[>=angle 60,->,line width=2.2pt, line cap=round,#5] (#1,#2) -- (#3,#4);
}

% \forceup{x}{y0}{y1}{color}
\newcommand{\forceup}[4]%
{
\draw[line width=1pt, line cap=round,#4] (#1-0.05,#2) -- (#1-0.05,{#3-0.05});
\draw[line width=1pt, line cap=round,dotted,white] (#1,#2) -- (#1,#3);
\draw[line width=1pt, line cap=round,#4] (#1+0.05,#2) -- (#1+0.05,{#3-0.05});
\draw[line width=1pt, line cap=round,#4] (#1,#3) -- ({#1-0.2},{#3-0.2});
\draw[line width=1pt, line cap=round,#4] (#1,#3) -- ({#1+0.2},{#3-0.2});
}

% \forceup{x}{y0}{y1}{color}{pos_ratio}{textpos}{text}
\newcommand{\forceupMore}[7]%
{
\draw[line width=1pt, line cap=round,#4] (#1-0.05,#2) -- (#1-0.05,{#3-0.05});
\draw[line width=1pt, line cap=round,dotted,white] (#1,#2) -- (#1,#3);
\draw[line width=1pt, line cap=round,#4] (#1+0.05,#2) -- (#1+0.05,{#3-0.05});
\draw[line width=1pt, line cap=round,#4] (#1,#3) -- ({#1-0.2},{#3-0.2});
\draw[line width=1pt, line cap=round,#4] (#1,#3) -- ({#1+0.2},{#3-0.2});
\draw (#1,{#2+#5*(#3-#2)}) node[#4,#6]{#7};
}

% \forcedown{x}{y0}{y1}{color}
\newcommand{\forcedown}[4]%
{
\draw[line width=1pt, line cap=round,#4] (#1-0.05,#2) -- (#1-0.05,{#3+0.05});
\draw[line width=1pt, line cap=round,dotted,white] (#1,#2) -- (#1,#3);
\draw[line width=1pt, line cap=round,#4] (#1+0.05,#2) -- (#1+0.05,{#3+0.05});
\draw[line width=1pt, line cap=round,#4] (#1,#3) -- ({#1-0.2},{#3+0.2});
\draw[line width=1pt, line cap=round,#4] (#1,#3) -- ({#1+0.2},{#3+0.2});
}

\newcommand{\forcedownMore}[7]%
{
\draw[line width=1pt, line cap=round,#4] (#1-0.05,#2) -- (#1-0.05,{#3+0.05});
\draw[line width=1pt, line cap=round,dotted,white] (#1,#2) -- (#1,#3);
\draw[line width=1pt, line cap=round,#4] (#1+0.05,#2) -- (#1+0.05,{#3+0.05});
\draw[line width=1pt, line cap=round,#4] (#1,#3) -- ({#1-0.2},{#3+0.2});
\draw[line width=1pt, line cap=round,#4] (#1,#3) -- ({#1+0.2},{#3+0.2});
\draw (#1,{#2-#5*(#2-#3)}) node[#4,#6]{#7};
}

% \moment{x}{y}{rayon}{angleDeb}{angleFin}{couleur}
\newcommand{\moment}[6]%
{
\draw[>=angle 60, ->, line width=2pt, line cap=round, #6] ({#1+#3*cos(#4)},{#2+#3*sin(#4)}) arc (#4:#5:#3);
}

\newcommand{\momentSO}[6]%
{
\draw[>=angle 60, ->, line width=2pt, line cap=round, shift only, #6] ({#1+#3*cos(#4)},{#2+#3*sin(#4)}) arc (#4:#5:#3);
}

% \point{x}{y}
\newcommand{\point}[2]%
{
\draw [orange, line width=1pt] (#1,#2) circle [radius=0.1414];
%\draw[orange, line width=1pt] ({#1-0.1},{#2-0.1}) -- ({#1+0.1},{#2+0.1});
%\draw[orange, line width=1pt] ({#1-0.1},{#2+0.1}) -- ({#1+0.1},{#2-0.1});
}

% \croix{x}{y}
\newcommand{\croix}[2]%
{
\draw[orange, line width=1.5pt, line cap=round] ({#1-0.1},{#2-0.1}) -- ({#1+0.1},{#2+0.1});
\draw[orange, line width=1.5pt, line cap=round] ({#1-0.1},{#2+0.1}) -- ({#1+0.1},{#2-0.1});
\draw[white, line width=0.5pt, line cap=round] ({#1-0.09},{#2-0.09}) -- ({#1+0.09},{#2+0.09});
\draw[white, line width=0.5pt, line cap=round] ({#1-0.09},{#2+0.09}) -- ({#1+0.09},{#2-0.09});
}

% \croixDiv{x}{y}
\newcommand{\croixDiv}[2]%
{
\draw[orange, line width=1pt] ({#1-0.066},{#2-0.1}) -- ({#1+0.066},{#2+0.1});
\draw[orange, line width=1pt] ({#1-0.066},{#2+0.1}) -- ({#1+0.066},{#2-0.1});
}

\newcommand{\jauge}[4]%
{
\def\dy{(#4-#2)/5}
 % Dessiner le rectangle
\draw[draw=black, line width=0.5pt] (#1,#2) rectangle (#3, #4);

% Dessiner la ligne avec des allers-retours et des virages arrondis
\draw[draw=black, line width=0.2pt] ({#1-2*\dy}, {#2+\dy}) -- ({#3-\dy}, {#2+\dy});
\draw[draw=black, line width=0.2pt] ({#3-\dy}, {#2+\dy}) arc (-90:90:{0.5*\dy});
\draw[draw=black, line width=0.2pt] ({#3-\dy}, {#2+2*\dy}) -- ({#1+\dy}, {#2+2*\dy});
\draw[draw=black, line width=0.2pt] ({#1+\dy}, {#2+3*\dy}) arc (90:270:{0.5*\dy});
\draw[draw=black, line width=0.2pt] ({#1+\dy}, {#2+3*\dy}) -- ({#3-\dy}, {#2+3*\dy});
\draw[draw=black, line width=0.2pt] ({#3-\dy}, {#2+3*\dy}) arc (-90:90:{0.5*\dy});
\draw[draw=black, line width=0.2pt] ({#1-2*\dy}, {#2+4*\dy}) -- ({#3-\dy}, {#2+4*\dy});
}

% \cote{x1}{y1}{x2}{y2}{texte}
\newcommand{\cote}[5]%
{
\draw[>=angle 45, <->, line width=1pt, darkgray] (#1,#2) -- (#3,#4);
\draw ({(#3+#1)/2},{#2}) node[above]{#5};
}

\newcommand{\cotep}[6]%
{
\draw[>=angle 45,<->,line width=1pt,darkgray] (#1,#2) -- (#3,#4);
\draw({#1+(#3-#1)/2},{#2+(#4-#2)/2}) node[#5]{#6};
}

% cote horizontale respectant les règles de dessin GC
% \coteH{x1}{y1}{x2}{y2}{texte}
\newcommand{\coteH}[5]%
{
\draw[line width=0.5pt,black] (#1,#2) -- (#1,#4);
\draw[line width=0.5pt,black] (#3,#2) -- (#3,#4);
\draw[>=angle 45,<->,line width=1pt,darkgray] (#1,#4) -- (#3,#4);
\draw ({(#3+#1)/2},{#4}) node[above]{#5};
}

% \coteHext{x1}{y1}{x2}{y2}{texte}
\newcommand{\coteHext}[5]%
{
\draw[line width=0.5pt,black] (#1,#2) -- (#1,#4);
\draw[line width=0.5pt,black] (#3,#2) -- (#3,#4);
\draw[>=angle 45,->,line width=1pt,darkgray] ({#1-0.25},#4) -- (#1,#4);
%\draw[line width=1pt,darkgray] (#1,#4) -- (#3,#4);
\draw[>=angle 45,->,line width=1pt,darkgray] ({#3+0.4},#4) -- (#3,#4);
\draw ({#3+0.3},{#4}) node[right]{#5};
}

% Pour les cotes par rapport au debut de tronçon
% \halfcoteH{x1}{y1}{x2}{y2}{texte}
\newcommand{\halfcoteH}[5]%
{
\draw[line width=0.5pt,black] (#1,#2) -- (#1,#4);
\draw[line width=0.5pt,black] (#3,#2) -- (#3,#4);
\draw[line width=2pt,darkgray, line cap=round] (#1,#4-0.1) -- (#1,#4+0.1);
\draw[>=angle 45,->,line width=1pt,darkgray] (#1,#4) -- (#3,#4);
\draw ({(#3+#1)/2},{#4}) node[above]{#5};
}

% \halfcoteV{x1}{y1}{x2}{y2}{texte}
\newcommand{\halfcoteV}[5]%
{
\draw[line width=0.5pt,black] (#1,#2) -- (#3,#2);
\draw[line width=0.5pt,black] (#1,#4) -- (#3,#4);
\draw[line width=2pt,darkgray, line cap=round] (#3-0.1,#2) -- (#3+0.1,#2);
\draw[>=angle 45,->,line width=1pt,darkgray] (#3,#2) -- (#3,#4);
\draw ({#3},{(#4+#2)/2}) node[rotate=90, above]{#5};
}

% cote horizontale respectant régle de dessin GC
% \coteV{x1}{y1}{x2}{y2}{texte}
\newcommand{\coteV}[5]%
{
\draw[line width=0.5pt,black] (#1,#2) -- (#3,#2);
\draw[line width=0.5pt,black] (#1,#4) -- (#3,#4);
\draw[>=angle 45,<->,line width=1pt,darkgray] (#3,#2) -- (#3,#4);
\draw ({#3},{(#4+#2)/2}) node[rotate=90, above]{#5};
}

% cote d'angle
% \coteA{xc}{yc}{angleInit}{angleOuverture}{rayon}{texte}
\newcommand{\coteA}[6]
{
\draw[line width=0.5pt] ({#1+#5*cos(#3)},{#2+#5*sin(#3)}) arc (#3:{#3+#4}:#5);
\draw[] ({#1+(#5+0.35)*cos(#3+0.5*#4)},{#2+(#5+0.35)*sin(#3+0.5*#4)}) node{#6};
}

\newcommand{\coteASO}[6]
{
\draw[line width=0.5pt, shift only] ({#1+#5*cos(#3)},{#2+#5*sin(#3)}) arc (#3:{#3+#4}:#5);
\draw[shift only] ({#1+(#5+0.35)*cos(#3+0.5*#4)},{#2+(#5+0.35)*sin(#3+0.5*#4)}) node{#6};
}

% \appuibati{x}{y}{rotation}{echelle}
\newcommand{\appuibati}[4]%
{
\begin{scope} [xshift=#1cm,yshift=#2cm,rotate=#3,scale=#4]
% triangle
\draw[black] (0,0) -- (0.5,-0.866);
\draw[black] (0.5,-0.866) -- (-0.5,-0.866);
\draw[black] (-0.5,-0.866) -- (0,0);
% roues
\draw [black] (-0.35,-1.041) circle [radius=0.175];
\draw [black] (0,-1.041) circle [radius=0.175];
\draw [black] (0.35,-1.041) circle [radius=0.175];
% sol
\draw[black] (-0.5,-1.216) -- (0.5,-1.216);
% hachures
\draw[black] (-0.5,-1.216) -- (-0.7,-1.416);
\draw[black] (-0.3,-1.216) -- (-0.5,-1.416);
\draw[black] (-0.1,-1.216) -- (-0.3,-1.416);
\draw[black] (0.1,-1.216) -- (-0.1,-1.416);
\draw[black] (0.3,-1.216) -- (0.1,-1.416);
\draw[black] (0.5,-1.216) -- (0.3,-1.416);
\end{scope}
}

% \appuiContinu{x}{y}{echelle}
\newcommand{\appuiContinu}[3]%
{
\begin{scope} [xshift=#1cm,yshift=#2cm,scale=#3]
% triangle
\draw[black] (0,0) -- (0.5,-0.866);
\draw[black] (0.5,-0.866) -- (-0.5,-0.866);
\draw[black] (-0.5,-0.866) -- (0,0);
% hachures
\draw[black] (-0.5,-0.866) -- (-0.7,-1.066);
\draw[black] (-0.3,-0.866) -- (-0.5,-1.066);
\draw[black] (-0.1,-0.866) -- (-0.3,-1.066);
\draw[black] (0.1,-0.866) -- (-0.1,-1.066);
\draw[black] (0.3,-0.866) -- (0.1,-1.066);
\draw[black] (0.5,-0.866) -- (0.3,-1.066);
\end{scope}
}

% \articulbati{x}{y}{rotation}{echelle}
\newcommand{\articulbati}[4]%
{
\begin{scope} [xshift=#1cm,yshift=#2cm,rotate=#3,scale=#4]
% cercle
\draw [black] (0,-0.5) circle [radius=0.5];
% sol
\draw[black] (-0.5,-1) -- (0.5,-1);
% hachures
\draw[black] (-0.5,-1) -- (-0.7,-1.2);
\draw[black] (-0.3,-1) -- (-0.5,-1.2);
\draw[black] (-0.1,-1) -- (-0.3,-1.2);
\draw[black] (0.1,-1) -- (-0.1,-1.2);
\draw[black] (0.3,-1) -- (0.1,-1.2);
\draw[black] (0.5,-1) -- (0.3,-1.2);
\end{scope}
}

% \articul{x}{y}{echelle}
\newcommand{\articul}[3]%
{
\begin{scope} [xshift=#1cm,yshift=#2cm,scale=#3]
% cercle
\draw [black] (0,0) circle [radius=0.5];
\end{scope}
}

% \encastrbati{x}{y}{rotation}{echelle}
\newcommand{\encastrbati}[4]%
{
\begin{scope} [xshift=#1cm,yshift=#2cm,rotate=#3,scale=#4]
% sol
\draw[thick,black] (0,0) -- (0,-0.1);
\draw[thick,black] (-0.5,-0.1) -- (0.5,-0.1);
% hachures
\draw[black] (-0.5,-0.1) -- (-0.7,-0.3);
\draw[black] (-0.3,-0.1) -- (-0.5,-0.3);
\draw[black] (-0.1,-0.1) -- (-0.3,-0.3);
\draw[black] (0.1,-0.1) -- (-0.1,-0.3);
\draw[black] (0.3,-0.1) -- (0.1,-0.3);
\draw[black] (0.5,-0.1) -- (0.3,-0.3);
\end{scope}
}

% \velo{x}{y}{rotation}{echelle}
\newcommand{\velo}[4]%
{
\begin{scope} [xshift=#1cm,yshift=#2cm,rotate=#3,scale=#4]
\draw [line width=1.5pt] (-1.2,1) circle (0.9);
\draw [line width=1.5pt] (1.2,1) circle (0.9);
\draw [->,line width=1pt] (2.5,0) -- (-3,0);
\draw [line cap=round,line join=round,line width=1.5pt] (-1.2,0.9)-- (-1,1.3);
\draw [line cap=round,line join=round,line width=1.5pt] (-1,1.3)-- (-0.6,2.4);
\draw [line cap=round,line join=round,line width=2pt] (0.1,1.)-- (-0.1,2.2);
\draw [line cap=round,line join=round,line width=2pt] (-0.1,2.2)-- (1.,2.3);
\draw [line cap=round,line join=round,line width=2pt] (1.,2.3)-- (0.5,2.8);
\draw [line width=2pt] (0.5,2.8)-- (-0.2,3.1);
\draw [line cap=round,line join=round,fill=black] (-0.6,3.2) circle (0.3);
\end{scope}
}
