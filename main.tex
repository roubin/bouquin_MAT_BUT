\documentclass[a4paper, french, 11pt, openright, fleqn]{book} 

%%%%%%%%%%%%%%%%%%%%%%%%%%%
%%%%%%%%%%%%%%%%%%%%%%%%%%%
\usepackage[utf8]{inputenc}
\usepackage[T1]{fontenc}
\usepackage[francais]{babel}

\usepackage[
  paperwidth = 190mm,
  paperheight = 240mm,
  includehead,
  textwidth = 130mm,
  %textheight = 205mm,
  top = 17.5mm,
  bottom = 17.5mm,
  marginparsep = 0pt,
  marginparwidth = 0pt,
  showframe = false,
  showcrop = true
]{geometry}

\usepackage[cross, center, a4, cam, noinfo]{crop}

\usepackage[fleqn]{mathtools}
\usepackage{framed}       % pour les environnement d'exo et sol
\usepackage{amsthm}       % pour spaceabove par exemple
\usepackage{thmtools}     % utilisé dans l'environnement des exo / sol
\usepackage{fancybox}     % utilisé dans l'environnement des exo / sol
\usepackage{multicol}     % passe en 2 colonnes quelques fois
\usepackage{cancel}       % pour barrer des termes dans les équations
\usepackage[fleqn]{mathtools}
\usepackage{wrapfig}      %
\usepackage{tikz}         % pour faire tous les schémas
\usepackage{pgfplots}     %
\usetikzlibrary{patterns} % UTILE ?????
\usetikzlibrary{tikzmark} % les flèches de relation dans quelques équations
\usetikzlibrary{arrows.meta} % grosse flèche du schéma en préamble
% Usage :
% \repereglobal{x}{y}{rotation}{echelle}
% \poutre{x1}{y1}{x2}{y2}
% \force{x1}{y1}{x2}{y2}{couleur}
% \moment{x}{y}{rayon}{angleDeb}{angleFin}{couleur}
% \point{x}{y}
% \cote{x1}{y1}{x2}{y2}{label}
% \cotep{x1}{y1}{x2}{y2}{position}{label}
% \appuibati{x}{y}{rotation}{echelle}
% \articulbati{x}{y}{rotation}{echelle}
% \encastrbati{x}{y}{rotation}{echelle}
% \upe{x}{y}{rotation}{echelle}
% \ipe{x}{y}{rotation}{echelle}
% \forceRep{xleft}{yleft}{xright}{yright}{text}
% \navierAxes{xmin}{ymin}{xmax}{ymax}{yaxe}{sigaxe}
% \navierStress{x0}{y0}{x1}{y1}{sigaxe}
% \navierStress{x0}{y0}{x1}{y1}{sigaxe}
% \velo{x}{y}{rotation}{echelle}
% ======================================================

\usetikzlibrary{arrows}
\usetikzlibrary{bending}
\usetikzlibrary{shadings}
\usetikzlibrary{decorations.pathmorphing}

\pgfdeclarepatternformonly{spaced north east lines}{\pgfqpoint{-1pt}{-1pt}}{\pgfqpoint{10pt}{10pt}}{\pgfqpoint{9pt}{9pt}}%
{
    \pgfsetlinewidth{1pt}
    \pgfpathmoveto{\pgfqpoint{0pt}{0pt}}
    \pgfpathlineto{\pgfqpoint{9.1pt}{9.1pt}}
    \pgfusepath{stroke}
}

\pgfdeclarepatternformonly{clean north east lines}{\pgfqpoint{-1pt}{-1pt}}{\pgfqpoint{5pt}{5pt}}{\pgfqpoint{5pt}{5pt}}%
{
    \pgfsetlinewidth{0.6pt}
    \pgfpathmoveto{\pgfqpoint{0pt}{0pt}}
    \pgfpathlineto{\pgfqpoint{5pt}{5pt}}
    \pgfusepath{stroke}
}

%\cercletrigo{rayon}
\newcommand{\cercletrigo}[2]%
{
\draw [black,line width=1pt,] (0,0) circle [radius=#1];
\draw[>=angle 60,->,line width=1pt, black, line cap=round] ({-#1-#2},0) -- ({#1+#2},0);
\draw[>=angle 60,->,line width=1pt, black, line cap=round] (0,{-#1-#2}) -- (0,{#1+#2});
}


% \SymbolVq{x0}{y0}{L}{H}{r}{color}{$L$}{$q\times L$}
\newcommand{\SymbolVq}[8]%
{
\draw ({#1-#3/2},#2) node[fill=white,above]{#7};
\draw ({#1-#3},{#2-#4/2}) node[fill=white,right]{#8};
\draw[>=angle 60,->,dashed,line width=2.5pt,line cap=round, #6] (#1, #2) -- ++({-(#3-#5)},0) arc (90:180:#5) -- ++(0,{-(#4-#5)});
}

\newcommand{\SymbolVqSmaller}[8]%
{
\draw ({#1-#3/2},#2) node[fill=white,above]{\tiny$\leftarrow$ #7, $\downarrow$ #8};
\draw[>=angle 60,->,dashed,line width=2.5pt,line cap=round, #6] (#1, #2) -- ++({-(#3-#5)},0) arc (90:180:#5) -- ++(0,{-(#4-#5)});
}

% \forceEquiv{x0}{y0}{x1}{y1}{textforce}{textpos}
\newcommand{\forceEquiv}[6]%
{
% boîte avec des bords ondulés
\draw[line width=1pt,line cap=round, decorate, decoration={random steps, amplitude=0.05 cm, segment length=0.4 cm}, gray, dotted]
({#1-0.6}, {#2-0.2}) -- ++({#3-#1+1.2}, 0) -- ++(0, {#4-#2+0.4}) -- ++({-#3+#1-1.2}, 0) -- cycle;
\draw[>=angle 60,->,line width=1.5pt, line cap=round, blue, dotted] ({(#1+#3)*0.5},{#2+1}) -- ({(#1+#3)*0.5},#2);
\draw ({(#1+#3)/2},{#2+1}) node[above]{#5};
%
\draw[line width=0.5pt,gray] (#1,{#2+0.2}) -- (#1,{#2+0.6});
\draw[>=angle 45,<->,line width=1pt,gray] (#1,{#2+0.4}) -- ({(#1+#3)/2},{#2+0.4});
\draw ({#1+(#3-#1)/4},{#2+0.4}) node[above]{#6};
%
\draw[line width=0.5pt,gray] (#3,{#2+0.2}) -- (#3,{#2+0.6});
\draw[>=angle 45,<->,line width=1pt,gray] ({(#1+#3)/2},{#2+0.4}) -- (#3,{#2+0.4});
\draw ({#1+3*(#3-#1)/4},{#2+0.4}) node[above]{#6};
%
\draw ({#3+0.8}, {#2-0.2}) node[fill=white, left]{\small Equivalence};
}

% \forceEquivSmaller{x0}{y0}{x1}{y1}{textforce}{textpos}
\newcommand{\forceEquivSmaller}[6]%
{
% boîte avec des bords ondulés
\draw[line width=1pt,line cap=round, decorate, decoration={random steps, amplitude=0.05 cm, segment length=0.4 cm}, gray, dotted]
({#1-0.6}, {#2-0.2}) -- ++({#3-#1+1.2}, 0) -- ++(0, {#4-#2+0.2}) -- ++({-#3+#1-1.2}, 0) -- cycle;
\draw[>=angle 60,->,line width=1.5pt, line cap=round, blue, dotted] ({(#1+#3)*0.5},{#2+1}) -- ({(#1+#3)*0.5},#2);
\draw ({(#1+#3)/2},{#2+1}) node[above]{#5};
%
\draw[line width=0.5pt,gray] (#1,{#2+0.2}) -- (#1,{#2+0.6});
\draw[>=angle 45,<->,line width=1pt,gray] (#1,{#2+0.4}) -- ({(#1+#3)/2},{#2+0.4});
\draw ({#1+(#3-#1)/4},{#2+0.4}) node[above]{#6};
%
\draw[line width=0.5pt,gray] (#3,{#2+0.2}) -- (#3,{#2+0.6});
\draw[>=angle 45,<->,line width=1pt,gray] ({(#1+#3)/2},{#2+0.4}) -- (#3,{#2+0.4});
\draw ({#1+3*(#3-#1)/4},{#2+0.4}) node[above]{#6};
%
\draw ({#3+0.8}, {#2-0.2}) node[fill=white, left]{\small Equiv.};
}

\newcommand{\debfin}[4]%
{
\draw[decorate, decoration={snake, pre length=5mm, post length=5mm}, line width=1pt] (#1,#2) -- (#3,#4);
\draw [black, fill=darkgray] (#1,#2) circle [radius=0.1];
\draw [black, fill=gray] (#3,#4) circle [radius=0.1];
}

% \upe{x}{y}{rotation}{echelle}
\newcommand{\upe}[4]%
{
\begin{scope}[xshift=#1cm, yshift=#2cm, rotate=#3, scale=#4]
\draw[draw=black, line width=1] (0,0) -- (2,0);
\draw[draw=black, line width=1] (0,0) -- (0,1);
\draw[draw=black, line width=1] (0,1) -- (0.2,1);
\draw[draw=black, line width=1] (0.2,1) -- (0.2,0.4);
\draw[draw=black, line width=1] (0.2,0.4) arc (180:270:0.2);
\draw[draw=black, line width=1] (0.4,0.2) -- (1.6,0.2);
\draw[draw=black, line width=1] (1.6,0.2) arc (270:360:0.2);
\draw[draw=black, line width=1] (1.8,0.4) -- (1.8,1);
\draw[draw=black, line width=1] (1.8,1) -- (2,1);
\draw[draw=black, line width=1] (2,1) -- (2,0);
\end{scope}
}

% \ipe{x}{y}{rotation}{echelle}
\newcommand{\ipe}[4]%
{
\begin{scope}[xshift=#1cm, yshift=#2cm, rotate=#3, scale=#4]
\draw[draw=black, line width=1] (0,0) -- (1,0);
\draw[draw=black, line width=1] (0,0) -- (0,0.2);
\draw[draw=black, line width=1] (0,0.2) -- (0.2,0.2);
\draw[draw=black, line width=1] (0.2,0.2) arc (270:360:0.2);
\draw[draw=black, line width=1] (0.4,0.4) -- (0.4,1.6);
\draw[draw=black, line width=1] (0.4,1.6) arc (0:90:0.2);
\draw[draw=black, line width=1] (0.2,1.8) -- (0,1.8);
\draw[draw=black, line width=1] (0,1.8) -- (0,2);
\draw[draw=black, line width=1] (0,2) -- (1,2);
\draw[draw=black, line width=1] (1,2) -- (1,1.8);
\draw[draw=black, line width=1] (1,1.8) -- (0.8,1.8);
\draw[draw=black, line width=1] (0.8,1.8) arc (90:180:0.2);
\draw[draw=black, line width=1] (0.6,1.6) -- (0.6,0.4);
\draw[draw=black, line width=1] (0.6,0.4) arc (180:270:0.2);
\draw[draw=black, line width=1] (0.8,0.2) -- (1,0.2);
\draw[draw=black, line width=1] (1,0.2) -- (1,0);
\end{scope}
}

% \forceRep{xleft}{yleft}{xright}{yright}{text}
\newcommand{\forceRep}[5]%
{
\draw[line width=1pt, blue] (#1,#4) -- (#3,#4);
\draw[>=angle 60,->,line width=1pt, blue, line cap=round] (#1,#4) -- (#1,#2);
\draw[>=angle 60,->,line width=1pt, blue, line cap=round] ({#1+(#3-#1)/4},#4) -- ({#1+(#3-#1)/4},#2);
\draw[>=angle 60,->,line width=1pt, blue, line cap=round] ({(#1+#3)/2},#4) -- ({(#1+#3)/2},#2);
\draw[>=angle 60,->,line width=1pt, blue, line cap=round] ({#3-(#3-#1)/4},#4) -- ({#3-(#3-#1)/4},#2);
\draw[>=angle 60,->,line width=1pt, blue, line cap=round] (#3,#4) -- (#3,#2);
\draw ({(#1+#3)/2},#4) node[above]{#5};
}

\newcommand{\forceRepMini}[5]%
{
\draw[line width=1pt, blue] (#1,#4) -- (#3,#4);
\draw[>=angle 60,->,line width=1pt, blue, line cap=round] (#1,#4) -- (#1,#2);
\draw[>=angle 60,->,line width=1pt, blue, line cap=round] ({#1+(#3-#1)/3},#4) -- ({#1+(#3-#1)/3},#2);
\draw[>=angle 60,->,line width=1pt, blue, line cap=round] ({#3-(#3-#1)/3},#4) -- ({#3-(#3-#1)/3},#2);
\draw[>=angle 60,->,line width=1pt, blue, line cap=round] (#3,#4) -- (#3,#2);
\draw ({(#1+#3)/2},#4) node[above]{#5};
}

% \flashForce{x}{y1}{y2}{w}
\newcommand{\flashForce}[4]%
{
\draw[line width=1pt, blue] (#1,#2) -- ({#1-#4},{0.5*(#2+#3)-#4});
\draw[line width=1pt, blue] ({#1-#4},{0.5*(#2+#3)-#4}) -- ({#1+#4},{0.5*(#2+#3)+#4});
\draw[>=angle 60,->,line width=1pt, blue] ({#1+#4},{0.5*(#2+#3)+#4}) -- (#1,#3);
}


% \navierAxes{xmin}{ymin}{xmax}{ymax}{yaxe}{sigaxe}
\newcommand{\navierAxes}[6]%
{
\draw[>=latex,->,thin, black] (#1,#5) -- (#3,#5);
\draw[>=latex,->,thin, black] (#6,#2) -- (#6,#4);
\draw (#3,#5) node[right]{$\sigma$};
\draw (#6,#4) node[above]{$Y$};
}

% \navierStress{x0}{y0}{x1}{y1}{sigaxe}
\newcommand{\navierStress}[5]%
{
\draw[line width=1pt, blue] (#1,#2) -- (#3,#4);
\draw[>=latex,->,line width=0.5pt, blue] (#5,#2) --(#1,#2);
\draw[>=latex,->,line width=0.5pt, blue] (#5,{#2+(#4-#2)*0.25}) -- ({#1+(#3-#1)*0.25},{#2+(#4-#2)*0.25});
\draw[>=latex,->,line width=0.5pt, blue] (#5,{#2+(#4-#2)*0.75}) -- ({#1+(#3-#1)*0.75},{#2+(#4-#2)*0.75});
\draw[>=latex,->,line width=0.5pt, blue] (#5,#4) -- (#3,#4);
}

% \navierStress{x0}{y0}{x1}{y1}{sigaxe}{nbFleches}
\newcommand{\navierStressCustom}[6]%
{
\draw[line width=1pt, blue] (#1,#2) -- (#3,#4);
\foreach \i in {1,...,#6}
{
    \draw[>=latex,->,line width=0.5pt, blue] (#5,{#2+(\i-1)*(#4-#2)/(#6-1)}) -- ({#1+(\i-1)*(#3-#1)/(#6-1)},{#2+(\i-1)*(#4-#2)/(#6-1)});
}
\draw[>=latex,->,line width=0.5pt, blue] (#5,#4) -- (#3,#4);
}

\newcommand{\navierStressOpt}[6]%
{
\draw[line width=1pt, line cap=round, #6] (#1,#2) -- (#3,#4);
\draw[>=latex,->,line width=0.5pt, #6] (#5,#2) --(#1,#2);
\draw[>=latex,->,line width=0.5pt, #6] (#5,{#2+(#4-#2)*0.25}) -- ({#1+(#3-#1)*0.25},{#2+(#4-#2)*0.25});
\draw[>=latex,->,line width=0.5pt, #6] (#5,{#2+(#4-#2)*0.75}) -- ({#1+(#3-#1)*0.75},{#2+(#4-#2)*0.75});
\draw[>=latex,->,line width=0.5pt, #6] (#5,#4) -- (#3,#4);
}

% \navierStressMid{x0}{y0}{x1}{y1}{sigaxe}
\newcommand{\navierStressMid}[5]%
{
\draw[line width=1pt, blue] (#1,#2) -- (#3,#4);
\draw[>=latex,->,line width=0.5pt, blue] (#5,#2) --(#1,#2);
\draw[>=latex,->,line width=0.5pt, blue] (#5,{#2+(#4-#2)*0.25}) -- ({#1+(#3-#1)*0.25},{#2+(#4-#2)*0.25});
\draw[>=latex,->,line width=0.5pt, blue] (#5,{#2+(#4-#2)*0.5}) -- ({#1+(#3-#1)*0.5},{#2+(#4-#2)*0.5});
\draw[>=latex,->,line width=0.5pt, blue] (#5,{#2+(#4-#2)*0.75}) -- ({#1+(#3-#1)*0.75},{#2+(#4-#2)*0.75});
\draw[>=latex,->,line width=0.5pt, blue] (#5,#4) -- (#3,#4);
}

\newcommand{\navierStressXup}[5]%
{
\draw[line width=1pt, line cap=round, blue] (#1,#2) -- (#3,#4);
\draw[>=latex,->,line width=0.5pt, blue] (#5,#2) -- (#1,#2);
%\draw[>=latex,->,line width=0.5pt, blue] (#5,{#2+(#4-#2)*0.25}) -- ({#1+(#3-#1)*0.25},{#2+(#4-#2)*0.25});
\draw[>=latex,->,line width=0.5pt, blue] (#5,{#2+(#4-#2)*0.5}) -- ({#1+(#3-#1)*0.5},{#2+(#4-#2)*0.5});
\draw[>=latex,->,line width=0.5pt, blue] (#5,{#2+(#4-#2)*0.75}) -- ({#1+(#3-#1)*0.75},{#2+(#4-#2)*0.75});
\draw[>=latex,->,line width=0.5pt, blue] (#5,#4) -- (#3,#4);
}

\newcommand{\navierStressXdown}[5]%
{
\draw[line width=1pt, line cap=round, blue] (#1,#2) -- (#3,#4);
\draw[>=latex,->,line width=0.5pt, blue] (#5,#2) -- (#1,#2);
\draw[>=latex,->,line width=0.5pt, blue] (#5,{#2+(#4-#2)*0.25}) -- ({#1+(#3-#1)*0.25},{#2+(#4-#2)*0.25});
\draw[>=latex,->,line width=0.5pt, blue] (#5,{#2+(#4-#2)*0.5}) -- ({#1+(#3-#1)*0.5},{#2+(#4-#2)*0.5});
%\draw[>=latex,->,line width=0.5pt, blue] (#5,{#2+(#4-#2)*0.75}) -- ({#1+(#3-#1)*0.75},{#2+(#4-#2)*0.75});
\draw[>=latex,->,line width=0.5pt, blue] (#5,#4) -- (#3,#4);
}

%\repereglobal{x}{y}{rotation}{echelle}
\newcommand{\repereglobal}[4]%
{
\begin{scope}[xshift=#1cm, yshift=#2cm, rotate=#3, scale=#4]
\draw[>=latex,->,line width=1pt, black] (0,0) -- (1,0);
\draw[>=latex,->,line width=1pt, black] (0,0) -- (0,1);
\draw[>=latex,->,line width=1pt, black] (0.3535,-0.3535) arc (-45:135:0.5);
\draw[black] (0.2,0.2) node[above right]{$+$};
\draw[black] (1,0) node[below]{$\vec{X}$};
\draw[black] (0,1) node[left]{$\vec{Y}$};
\end{scope}
}

\newcommand{\grandrepere}[5]%
{
\begin{scope}[xshift=#1cm, yshift=#2cm, rotate=#3, scale=#4]
\draw[>=latex,->,line width=1pt,#5] (0,0) -- (1,0);
\draw[>=latex,->,line width=1pt,#5] (0,0) -- (0,1);
\draw[>=latex,->,line width=1pt,#5] (0.3535,-0.3535) arc (-45:135:0.5);
\draw[#5] (0.2,0.2) node[above right]{$+$};
\draw[#5] (1,0) node[below]{$\vec{X}$};
\draw[#5] (0,1) node[left]{$\vec{Y}$};
\end{scope}
}

%\repereZY{x}{y}{rotation}{echelle}
\newcommand{\repereZY}[4]%
{
\begin{scope}[xshift=#1cm, yshift=#2cm, rotate=#3, scale=#4]
\draw[>=latex,->,thin, black] (1,0) -- (-1,0);
\draw[>=latex,->,thin, black] (0,-1) -- (0,1);
\draw (-1,0) node[left]{$Z$};
\draw (0,1) node[above]{$Y$};
\end{scope}
}

%\repereZY{x}{y}{rotation}{echelle}{aspect ratio}
\newcommand{\repereZYAR}[5]%
{
\begin{scope}[xshift=#1cm, yshift=#2cm, rotate=#3, scale=#4]
\draw[>=latex,->,thin, black] (1,0) -- (-1,0);
\draw[>=latex,->,thin, black] (0,-#5) -- (0,#5);
\draw (-1,0) node[left]{$Z$};
\draw (0,#5) node[above]{$Y$};
\end{scope}
}

%\reperelocaltext{x}{y}{rotation}{echelle}
\newcommand{\reperelocaltext}[4]%
{
\begin{scope}[xshift=#1cm, yshift=#2cm, rotate=#3, scale=#4]
\draw[>=latex,->,thin, black] (0,0) -- (1,0);
\draw[>=latex,->,thin, black] (0,0) -- (0,1);
\draw (1,0) node[xshift=2pt]{\tiny$x$};
\draw (0,1) node[xshift=3pt]{\tiny$y$};
\end{scope}
}

%\reperelocal{x}{y}{rotation}{echelle}
\newcommand{\reperelocal}[4]%
{
\begin{scope}[xshift=#1cm, yshift=#2cm, rotate=#3, scale=#4]
\draw[>=latex,->,thin, black] (0,0) -- (1,0);
\draw[>=latex,->,thin, black] (0,0) -- (0,1);
\end{scope}
}

\newcommand{\reperelocalShiftOnly}[3]%
{
\begin{scope}[xshift=#1cm, yshift=#2cm, rotate=#3]
\draw[>=latex,->,thin, shift only, black] (0,0) -- (0.5,0);
\draw[>=latex,->,thin, shift only, black] (0,0) -- (0,0.5);
\end{scope}
}

\newcommand{\poutre}[4]%
{
\draw[line width=2.5pt, line cap=round, black] (#1,#2) -- (#3,#4);
}

\newcommand{\poutreOpt}[5]%
{
\draw[line width=2.5pt, line cap=round, #5] (#1,#2) -- (#3,#4);
}

\newcommand{\poutreComp}[4]%
{
\draw[line width=3.0pt, line cap=round, red] (#1,#2) -- (#3,#4);
}

\newcommand{\poutreTens}[4]%
{
\draw[line width=1.0pt, line cap=round, blue] (#1,#2) -- (#3,#4);
}

\newcommand{\poutreZero}[4]%
{
\draw[line width=1.0pt, line cap=round, dotted, black] (#1,#2) -- (#3,#4);
}

\newcommand{\force}[5]%
{
\draw[>=angle 60,->,line width=2.2pt, line cap=round,#5] (#1,#2) -- (#3,#4);
}

% \forceup{x}{y0}{y1}{color}
\newcommand{\forceup}[4]%
{
\draw[line width=1pt, line cap=round,#4] (#1-0.05,#2) -- (#1-0.05,{#3-0.05});
\draw[line width=1pt, line cap=round,dotted,white] (#1,#2) -- (#1,#3);
\draw[line width=1pt, line cap=round,#4] (#1+0.05,#2) -- (#1+0.05,{#3-0.05});
\draw[line width=1pt, line cap=round,#4] (#1,#3) -- ({#1-0.2},{#3-0.2});
\draw[line width=1pt, line cap=round,#4] (#1,#3) -- ({#1+0.2},{#3-0.2});
}

% \forceup{x}{y0}{y1}{color}{pos_ratio}{textpos}{text}
\newcommand{\forceupMore}[7]%
{
\draw[line width=1pt, line cap=round,#4] (#1-0.05,#2) -- (#1-0.05,{#3-0.05});
\draw[line width=1pt, line cap=round,dotted,white] (#1,#2) -- (#1,#3);
\draw[line width=1pt, line cap=round,#4] (#1+0.05,#2) -- (#1+0.05,{#3-0.05});
\draw[line width=1pt, line cap=round,#4] (#1,#3) -- ({#1-0.2},{#3-0.2});
\draw[line width=1pt, line cap=round,#4] (#1,#3) -- ({#1+0.2},{#3-0.2});
\draw (#1,{#2+#5*(#3-#2)}) node[#4,#6]{#7};
}

% \forcedown{x}{y0}{y1}{color}
\newcommand{\forcedown}[4]%
{
\draw[line width=1pt, line cap=round,#4] (#1-0.05,#2) -- (#1-0.05,{#3+0.05});
\draw[line width=1pt, line cap=round,dotted,white] (#1,#2) -- (#1,#3);
\draw[line width=1pt, line cap=round,#4] (#1+0.05,#2) -- (#1+0.05,{#3+0.05});
\draw[line width=1pt, line cap=round,#4] (#1,#3) -- ({#1-0.2},{#3+0.2});
\draw[line width=1pt, line cap=round,#4] (#1,#3) -- ({#1+0.2},{#3+0.2});
}

\newcommand{\forcedownMore}[7]%
{
\draw[line width=1pt, line cap=round,#4] (#1-0.05,#2) -- (#1-0.05,{#3+0.05});
\draw[line width=1pt, line cap=round,dotted,white] (#1,#2) -- (#1,#3);
\draw[line width=1pt, line cap=round,#4] (#1+0.05,#2) -- (#1+0.05,{#3+0.05});
\draw[line width=1pt, line cap=round,#4] (#1,#3) -- ({#1-0.2},{#3+0.2});
\draw[line width=1pt, line cap=round,#4] (#1,#3) -- ({#1+0.2},{#3+0.2});
\draw (#1,{#2-#5*(#2-#3)}) node[#4,#6]{#7};
}

% \moment{x}{y}{rayon}{angleDeb}{angleFin}{couleur}
\newcommand{\moment}[6]%
{
\draw[>=angle 60, ->, line width=2pt, line cap=round, #6] ({#1+#3*cos(#4)},{#2+#3*sin(#4)}) arc (#4:#5:#3);
}

\newcommand{\momentSO}[6]%
{
\draw[>=angle 60, ->, line width=2pt, line cap=round, shift only, #6] ({#1+#3*cos(#4)},{#2+#3*sin(#4)}) arc (#4:#5:#3);
}

% \point{x}{y}
\newcommand{\point}[2]%
{
\draw [orange, line width=1pt] (#1,#2) circle [radius=0.1414];
%\draw[orange, line width=1pt] ({#1-0.1},{#2-0.1}) -- ({#1+0.1},{#2+0.1});
%\draw[orange, line width=1pt] ({#1-0.1},{#2+0.1}) -- ({#1+0.1},{#2-0.1});
}

% \croix{x}{y}
\newcommand{\croix}[2]%
{
\draw[orange, line width=1.5pt, line cap=round] ({#1-0.1},{#2-0.1}) -- ({#1+0.1},{#2+0.1});
\draw[orange, line width=1.5pt, line cap=round] ({#1-0.1},{#2+0.1}) -- ({#1+0.1},{#2-0.1});
\draw[white, line width=0.5pt, line cap=round] ({#1-0.09},{#2-0.09}) -- ({#1+0.09},{#2+0.09});
\draw[white, line width=0.5pt, line cap=round] ({#1-0.09},{#2+0.09}) -- ({#1+0.09},{#2-0.09});
}

% \croixDiv{x}{y}
\newcommand{\croixDiv}[2]%
{
\draw[orange, line width=1pt] ({#1-0.066},{#2-0.1}) -- ({#1+0.066},{#2+0.1});
\draw[orange, line width=1pt] ({#1-0.066},{#2+0.1}) -- ({#1+0.066},{#2-0.1});
}

\newcommand{\jauge}[4]%
{
\def\dy{(#4-#2)/5}
 % Dessiner le rectangle
\draw[draw=black, line width=0.5pt] (#1,#2) rectangle (#3, #4);

% Dessiner la ligne avec des allers-retours et des virages arrondis
\draw[draw=black, line width=0.2pt] ({#1-2*\dy}, {#2+\dy}) -- ({#3-\dy}, {#2+\dy});
\draw[draw=black, line width=0.2pt] ({#3-\dy}, {#2+\dy}) arc (-90:90:{0.5*\dy});
\draw[draw=black, line width=0.2pt] ({#3-\dy}, {#2+2*\dy}) -- ({#1+\dy}, {#2+2*\dy});
\draw[draw=black, line width=0.2pt] ({#1+\dy}, {#2+3*\dy}) arc (90:270:{0.5*\dy});
\draw[draw=black, line width=0.2pt] ({#1+\dy}, {#2+3*\dy}) -- ({#3-\dy}, {#2+3*\dy});
\draw[draw=black, line width=0.2pt] ({#3-\dy}, {#2+3*\dy}) arc (-90:90:{0.5*\dy});
\draw[draw=black, line width=0.2pt] ({#1-2*\dy}, {#2+4*\dy}) -- ({#3-\dy}, {#2+4*\dy});
}

% \cote{x1}{y1}{x2}{y2}{texte}
\newcommand{\cote}[5]%
{
\draw[>=angle 45, <->, line width=1pt, darkgray] (#1,#2) -- (#3,#4);
\draw ({(#3+#1)/2},{#2}) node[above]{#5};
}

\newcommand{\cotep}[6]%
{
\draw[>=angle 45,<->,line width=1pt,darkgray] (#1,#2) -- (#3,#4);
\draw({#1+(#3-#1)/2},{#2+(#4-#2)/2}) node[#5]{#6};
}

% cote horizontale respectant les règles de dessin GC
% \coteH{x1}{y1}{x2}{y2}{texte}
\newcommand{\coteH}[5]%
{
\draw[line width=0.5pt,black] (#1,#2) -- (#1,#4);
\draw[line width=0.5pt,black] (#3,#2) -- (#3,#4);
\draw[>=angle 45,<->,line width=1pt,darkgray] (#1,#4) -- (#3,#4);
\draw ({(#3+#1)/2},{#4}) node[above]{#5};
}

% \coteHext{x1}{y1}{x2}{y2}{texte}
\newcommand{\coteHext}[5]%
{
\draw[line width=0.5pt,black] (#1,#2) -- (#1,#4);
\draw[line width=0.5pt,black] (#3,#2) -- (#3,#4);
\draw[>=angle 45,->,line width=1pt,darkgray] ({#1-0.25},#4) -- (#1,#4);
%\draw[line width=1pt,darkgray] (#1,#4) -- (#3,#4);
\draw[>=angle 45,->,line width=1pt,darkgray] ({#3+0.4},#4) -- (#3,#4);
\draw ({#3+0.3},{#4}) node[right]{#5};
}

% Pour les cotes par rapport au debut de tronçon
% \halfcoteH{x1}{y1}{x2}{y2}{texte}
\newcommand{\halfcoteH}[5]%
{
\draw[line width=0.5pt,black] (#1,#2) -- (#1,#4);
\draw[line width=0.5pt,black] (#3,#2) -- (#3,#4);
\draw[line width=2pt,darkgray, line cap=round] (#1,#4-0.1) -- (#1,#4+0.1);
\draw[>=angle 45,->,line width=1pt,darkgray] (#1,#4) -- (#3,#4);
\draw ({(#3+#1)/2},{#4}) node[above]{#5};
}

% \halfcoteV{x1}{y1}{x2}{y2}{texte}
\newcommand{\halfcoteV}[5]%
{
\draw[line width=0.5pt,black] (#1,#2) -- (#3,#2);
\draw[line width=0.5pt,black] (#1,#4) -- (#3,#4);
\draw[line width=2pt,darkgray, line cap=round] (#3-0.1,#2) -- (#3+0.1,#2);
\draw[>=angle 45,->,line width=1pt,darkgray] (#3,#2) -- (#3,#4);
\draw ({#3},{(#4+#2)/2}) node[rotate=90, above]{#5};
}

% cote horizontale respectant régle de dessin GC
% \coteV{x1}{y1}{x2}{y2}{texte}
\newcommand{\coteV}[5]%
{
\draw[line width=0.5pt,black] (#1,#2) -- (#3,#2);
\draw[line width=0.5pt,black] (#1,#4) -- (#3,#4);
\draw[>=angle 45,<->,line width=1pt,darkgray] (#3,#2) -- (#3,#4);
\draw ({#3},{(#4+#2)/2}) node[rotate=90, above]{#5};
}

% cote d'angle
% \coteA{xc}{yc}{angleInit}{angleOuverture}{rayon}{texte}
\newcommand{\coteA}[6]
{
\draw[line width=0.5pt] ({#1+#5*cos(#3)},{#2+#5*sin(#3)}) arc (#3:{#3+#4}:#5);
\draw[] ({#1+(#5+0.35)*cos(#3+0.5*#4)},{#2+(#5+0.35)*sin(#3+0.5*#4)}) node{#6};
}

\newcommand{\coteASO}[6]
{
\draw[line width=0.5pt, shift only] ({#1+#5*cos(#3)},{#2+#5*sin(#3)}) arc (#3:{#3+#4}:#5);
\draw[shift only] ({#1+(#5+0.35)*cos(#3+0.5*#4)},{#2+(#5+0.35)*sin(#3+0.5*#4)}) node{#6};
}

% \appuibati{x}{y}{rotation}{echelle}
\newcommand{\appuibati}[4]%
{
\begin{scope} [xshift=#1cm,yshift=#2cm,rotate=#3,scale=#4]
% triangle
\draw[black] (0,0) -- (0.5,-0.866);
\draw[black] (0.5,-0.866) -- (-0.5,-0.866);
\draw[black] (-0.5,-0.866) -- (0,0);
% roues
\draw [black] (-0.35,-1.041) circle [radius=0.175];
\draw [black] (0,-1.041) circle [radius=0.175];
\draw [black] (0.35,-1.041) circle [radius=0.175];
% sol
\draw[black] (-0.5,-1.216) -- (0.5,-1.216);
% hachures
\draw[black] (-0.5,-1.216) -- (-0.7,-1.416);
\draw[black] (-0.3,-1.216) -- (-0.5,-1.416);
\draw[black] (-0.1,-1.216) -- (-0.3,-1.416);
\draw[black] (0.1,-1.216) -- (-0.1,-1.416);
\draw[black] (0.3,-1.216) -- (0.1,-1.416);
\draw[black] (0.5,-1.216) -- (0.3,-1.416);
\end{scope}
}

% \appuiContinu{x}{y}{echelle}
\newcommand{\appuiContinu}[3]%
{
\begin{scope} [xshift=#1cm,yshift=#2cm,scale=#3]
% triangle
\draw[black] (0,0) -- (0.5,-0.866);
\draw[black] (0.5,-0.866) -- (-0.5,-0.866);
\draw[black] (-0.5,-0.866) -- (0,0);
% hachures
\draw[black] (-0.5,-0.866) -- (-0.7,-1.066);
\draw[black] (-0.3,-0.866) -- (-0.5,-1.066);
\draw[black] (-0.1,-0.866) -- (-0.3,-1.066);
\draw[black] (0.1,-0.866) -- (-0.1,-1.066);
\draw[black] (0.3,-0.866) -- (0.1,-1.066);
\draw[black] (0.5,-0.866) -- (0.3,-1.066);
\end{scope}
}

% \articulbati{x}{y}{rotation}{echelle}
\newcommand{\articulbati}[4]%
{
\begin{scope} [xshift=#1cm,yshift=#2cm,rotate=#3,scale=#4]
% cercle
\draw [black] (0,-0.5) circle [radius=0.5];
% sol
\draw[black] (-0.5,-1) -- (0.5,-1);
% hachures
\draw[black] (-0.5,-1) -- (-0.7,-1.2);
\draw[black] (-0.3,-1) -- (-0.5,-1.2);
\draw[black] (-0.1,-1) -- (-0.3,-1.2);
\draw[black] (0.1,-1) -- (-0.1,-1.2);
\draw[black] (0.3,-1) -- (0.1,-1.2);
\draw[black] (0.5,-1) -- (0.3,-1.2);
\end{scope}
}

% \articul{x}{y}{echelle}
\newcommand{\articul}[3]%
{
\begin{scope} [xshift=#1cm,yshift=#2cm,scale=#3]
% cercle
\draw [black] (0,0) circle [radius=0.5];
\end{scope}
}

% \encastrbati{x}{y}{rotation}{echelle}
\newcommand{\encastrbati}[4]%
{
\begin{scope} [xshift=#1cm,yshift=#2cm,rotate=#3,scale=#4]
% sol
\draw[thick,black] (0,0) -- (0,-0.1);
\draw[thick,black] (-0.5,-0.1) -- (0.5,-0.1);
% hachures
\draw[black] (-0.5,-0.1) -- (-0.7,-0.3);
\draw[black] (-0.3,-0.1) -- (-0.5,-0.3);
\draw[black] (-0.1,-0.1) -- (-0.3,-0.3);
\draw[black] (0.1,-0.1) -- (-0.1,-0.3);
\draw[black] (0.3,-0.1) -- (0.1,-0.3);
\draw[black] (0.5,-0.1) -- (0.3,-0.3);
\end{scope}
}

% \velo{x}{y}{rotation}{echelle}
\newcommand{\velo}[4]%
{
\begin{scope} [xshift=#1cm,yshift=#2cm,rotate=#3,scale=#4]
\draw [line width=1.5pt] (-1.2,1) circle (0.9);
\draw [line width=1.5pt] (1.2,1) circle (0.9);
\draw [->,line width=1pt] (2.5,0) -- (-3,0);
\draw [line cap=round,line join=round,line width=1.5pt] (-1.2,0.9)-- (-1,1.3);
\draw [line cap=round,line join=round,line width=1.5pt] (-1,1.3)-- (-0.6,2.4);
\draw [line cap=round,line join=round,line width=2pt] (0.1,1.)-- (-0.1,2.2);
\draw [line cap=round,line join=round,line width=2pt] (-0.1,2.2)-- (1.,2.3);
\draw [line cap=round,line join=round,line width=2pt] (1.,2.3)-- (0.5,2.8);
\draw [line width=2pt] (0.5,2.8)-- (-0.2,3.1);
\draw [line cap=round,line join=round,fill=black] (-0.6,3.2) circle (0.3);
\end{scope}
}
       % des outils pour faire les schémas en mécanique des structures
\usepackage{enumitem}     % régler certain espacements dans les listes
\usepackage{adjustbox}    % pour mieux centrer (verticalement)
\usepackage[tikz]{bclogo} % les encadrés avec attention ou lampe
\usepackage{graphicx}     %
%\usepackage{amsmath}	  % already load by mathtools
\usepackage{amssymb}      % pour curvearrowright par exemple
\usepackage{bm}           % certain symboles en gras
\usepackage{array}        % pour pouvoir center les cellules de tabular
\usepackage{vwcol}        %
\usepackage{booktabs}     %
\usepackage{tasks}        % liste de tâches pour les exercices
\usepackage{needspace}    % ajustement des changements de page
\usepackage{pifont}       % des symbols graphiques
%\usepackage{fontawesome5} % d'autres symbols
\usepackage{siunitx}      % système d'unité pour écrire les nombres avec leurs unités
\usepackage{pdfpages}     % insertion des catalogues en annexe
\usepackage{mathabx}      % pour widecheck
\usepackage{comment}      % nécessaire pendant la phase de rédaction
\usepackage{footnote}     % nécessaire pour le 'savenotes'
\usepackage{titlepic}     % juste pour avoir une image sur la page de garde
\usepackage{mathabx}      % certain symbols
\usepackage[normalem]{ulem} % souligner
\usepackage{ragged2e}       % gestion des espaces
\usepackage{emptypage}    % Pour supprimer les en-têtes et pieds de page des pages blanches
\usepackage{fancyhdr}
\usepackage{imakeidx}
\usepackage{esvect}

%%%%%%%%%%%%%%%%%%%%%%%%%%%
%%%%%%%%%%%%%%%%%%%%%%%%%%%

\pagestyle{fancy}
\fancyhf{} % Supprime les en-têtes et pieds de page par défaut
\fancyhead[LE,RO]{\thepage} % Numéro de page à gauche sur les pages paires, à droite sur les pages impaires
\fancyhead[RE]{\leftmark} % Titre du chapitre à droite sur les pages paires
\fancyhead[LO]{\rightmark} % Titre de la section à gauche sur les pages impaires
\fancyfoot{} % Supprime tout contenu dans le pied de page
\renewcommand{\headrulewidth}{0.4pt} % Ligne sous l'en-tête
\renewcommand{\footrulewidth}{0pt} % Pas de ligne au-dessus du pied de page

% Redéfinir le style plain pour supprimer les numéros de page sur les premières pages de chapitre
\fancypagestyle{plain}{
    \fancyhf{} % Supprime les en-têtes et pieds de page par défaut
    \fancyfoot{} % Supprime tout contenu dans le pied de page
    \renewcommand{\headrulewidth}{0pt} % Pas de ligne sous l'en-tête
    \renewcommand{\footrulewidth}{0pt} % Pas de ligne au-dessus du pied de page
}

%%%%%%%%%%%%%%%%%%%%%%%%%%%
%%%%%%%%%%%%%%%%%%%%%%%%%%%

\usepackage{pgfplots}
\pgfplotsset{width=6cm, compat = newest}
\usepgfplotslibrary{fillbetween}

% Style personnalisé pour les graphiques (2D)
\pgfplotsset{
    mathbook/.style={
        width=0.8\textwidth, % Largeur du graphique
        height=6cm, % Hauteur du graphique
        axis lines=middle, % Axes centrés
        axis line style={->,>=stealth}, % Style des flèches
        xlabel={$x$}, % Label de l'axe x
        ylabel={$y$}, % Label de l'axe y
        xlabel style={below right}, % Position du label x
        ylabel style={above left}, % Position du label y
        grid=both, % Afficher la grille
        grid style={dashed,gray!30}, % Style de la grille
        tick style={black,thick}, % Style des graduations
        legend style={at={(1,1)},anchor=north east}, % Position de la légende
        samples=200, % Nombre de points pour le tracé
        thick, % Épaisseur des courbes
        smooth, % Lissage des courbes
        clip=false, % Permet d'afficher les labels hors de la zone de tracé
    }
}

%%%%%%%%%%%%%%%%%%%%%%%%%%%
%%%%%%%%%%%%%%%%%%%%%%%%%%%

%\newcommand{\repereDebut}{\raisebox{-0.3ex}{\ding{229}~}}
%\newcommand{\xx}{\mathbf{x}}

\newcommand{\centeredcircled}[1]{%
  \raisebox{.5pt}{\textcircled{\raisebox{-.9pt}{#1}}}%
}

\renewcommand{\thefootnote}{\fnsymbol{footnote}}

\sisetup{
output-decimal-marker = {,},
unit-mode = text,
%inter-unit-product = \ensuremath{{\!}$\cdot${\!}},
inter-unit-product = \ensuremath{\mathbin{\cdot}},
retain-explicit-plus
}

%%%%%%%
\makeatletter
\newcommand*{\underarrow}{\def\@underarrow{\relax}\@ifstar{\@@underarrow}{\def\@underarrow{\hidewidth}\@@underarrow}}
\newcommand*{\@@underarrow}[2][]{\underset{\@underarrow\substack{\uparrow\if\relax\detokenize{#1}\relax\else\\#1\fi}\@underarrow}{#2}}

\newcommand*{\overarrow}{\def\@overarrow{\relax}\@ifstar{\@@overarrow}{\def\@overarrow{\hidewidth}\@@overarrow}}
\newcommand*{\@@overarrow}[2][]{\overset{\@overarrow\substack{\if\relax\detokenize{#1}\relax\else#1\\\fi\downarrow}\@overarrow}{#2}}
\makeatother
%%%%%%%

\usepackage{hyperref}
\hypersetup{
    bookmarks=true,
    bookmarksopen=false,
    colorlinks=true,
    linkcolor=blue,
    urlcolor=blue,
    linktocpage=true,
    pdftitle={Outils Mathématiques},
    pdfpagemode=FullScreen,
    pdfauthor={V. Richefeu},
    pdfstartpage=1,
}

\newcommand{\dd}{\mathrm{d}} % le "d droit" pour les différentielles
\newcommand{\superpo}{\stackrel{\scriptscriptstyle\text{sup}}{\oplus}}

\definecolor{mygreen}{rgb}{0.0, 0.5, 0.0}

%%%%%%%%%
\newenvironment{myleftbar}{%
\def\FrameCommand{\hspace{-0.2em}\vrule width 3pt\hspace{0.2em}}%
\MakeFramed{\advance\hsize-\width \FrameRestore}}%
{\endMakeFramed}

\newenvironment{myleftbarGray}{%
\def\FrameCommand{{\color{black!50}\hspace{-0.2em}\vrule width 3pt\hspace{0.2em}}}%
\MakeFramed{\advance\hsize-\width \FrameRestore}}%
{\endMakeFramed}

\newenvironment{myleftbarSol}{%
\def\FrameCommand{{\color{black!50}\hspace{-0.2em}\vrule width 2pt\hspace{0.2em}}}%
\MakeFramed{\advance\hsize-\width \FrameRestore}}%
{\endMakeFramed}

\declaretheoremstyle[
spaceabove = 6pt,
spacebelow = 6pt,
headfont = \normalfont\bfseries,
postheadspace = 0pt,
headindent = 0pt,
headpunct = {},
headformat = {\cornersize*{2pt}\ovalbox{\NAME~\NUMBER\ifstrequal{\NOTE}{}{\relax}{\NOTE}}},
%headformat={\cornersize*{2pt}\ovalbox{\NAME~\NUMBER\ifstrequal{\NOTE}{}{\relax}{\NOTE}}\hspace{1em}}, 
bodyfont = \normalfont,
]{exobreak}

\declaretheorem[style=exobreak, name={%
\raisebox{-2mm}{\includegraphics[height=7mm]{figs/hard-working-icon-6.jpg}}%
\hspace*{2mm}Exercice},%
postheadhook = \leavevmode\myleftbar,%
prefoothook  = \endmyleftbar]{exo}

\declaretheorem[style=exobreak, name={%
\raisebox{-2mm}{\includegraphics[height=7mm]{figs/solution-icon-png-5.jpg}}%
\hspace*{2mm}Correction de l'exercice},%
postheadhook = \leavevmode\myleftbarGray,%
prefoothook = \endmyleftbarGray]{sol}

%%%%%%%%%%%%%%%%%%%%%%%%%%%%%%%%%%%%%%%%%%%%%%%%%%%%%%%%%%

\parskip=2pt % petit espace vertical entre les paragraphes

\title{Outils et méthodes en mathématiques\\
\vspace*{1cm}
\large Un guide pratique à l’usage des étudiants du BUT GCCD}
\author{Vincent~\textsc{Richefeu}, Emmanuel~\textsc{Roubin}}
\date{\today}
%\titlepic{\includegraphics[width=0.99\textwidth]{figs/progression_contrasted.jpg}}

\setcounter{tocdepth}{1}

\begin{document}

% Activer le mode raggedbottom pour permettre un espacement flexible en bas de page
\raggedbottom

\maketitle

\frontmatter

%!TEX root = main.tex

\chapter*{Avant propos}
\label{chapter:avant-propos}

Ce manuel de mathématiques a été conçu à partir des enseignements dispensés à l'IUT de Grenoble dans le cadre du BUT Génie Civil -- Construction Durable (GCCD). Il s'adresse avant tout aux étudiants de cette formation, mais pourra également servir de référence à toute personne souhaitant acquérir ou consolider les bases mathématiques nécessaires à la compréhension des phénomènes rencontrés en génie civil.

Les chapitres réunis dans cet ouvrage couvrent un ensemble de notions essentielles au programme du BUT. Certains d'entre eux entretiennent un lien direct avec les applications du génie civil : la topographie, la mécanique des structures ou encore la modélisation de phénomènes physiques y trouvent naturellement leur place. D'autres, en revanche, traitent de concepts plus abstraits ou dont la relation avec le domaine professionnel est moins immédiate et/ou évidente. Leur présence répond à un autre objectif tout aussi important : préparer certains  étudiants à la poursuite d'études et à la maîtrise des outils mathématiques indispensables à la compréhension de modèles plus complexes.

L'apprentissage des mathématiques ne se limite pas à l'application de formules. C'est une discipline de raisonnement et de méthode, qui demande rigueur, curiosité et persévérance. C'est pourquoi chaque chapitre comporte des exercices de nature variée. Certains sont purement mathématiques : ils permettent de se familiariser avec les notions fondamentales, d'acquérir les automatismes et de développer la logique nécessaire à toute démarche scientifique. D'autres proposent un ancrage dans des situations concrètes, souvent inspirées du génie civil, pour montrer que les mathématiques ne sont pas une fin en soi mais un outil puissant de compréhension et de conception.

L'objectif de ce manuel est donc double : offrir une formation solide et structurée en mathématiques, et en même temps, aider les étudiants à percevoir la portée et l'utilité de cette discipline dans leur futur métier. Nous espérons qu'il accompagnera chacun dans la découverte, parfois exigeante mais toujours formatrice, du langage mathématique qui sous-tend les sciences de l'ingénierie et de la construction.


% Il faudra entrer dans le détail du contenu chapitre par chapitre



\tableofcontents

\mainmatter

% Une moyenne de 30 pages par chapitre

% MAT1
%!TEX root = main.tex

\chapter{Trigonométrie}
\label{chapter:trigo}


\section{Introduction}

La trigonométrie est une branche des mathématiques qui étudie les relations entre les angles et les côtés des triangles. Bien que souvent perçue comme une discipline théorique, elle est en réalité un outil fondamental dans de nombreux domaines techniques, notamment en génie civil. Que ce soit pour concevoir des infrastructures, réaliser des levés topographiques ou calculer des pentes, la trigonométrie permet aux ingénieurs de résoudre des problèmes concrets avec précision et efficacité.

En guise d'exemple concret, imaginons un ingénieur  voulant mesurer la hauteur d'un bâtiment en utilisant un dispositif de mesure des angles par rapport à l'horizontale (théodolite). Voici comment la trigonométrie entre en jeu. L'ingénieur place le théodolite à une distance $d$ du bâtiment et mesure l'angle $\theta$ entre le sol et le sommet du bâtiment. En utilisant la fonction tangente, qui relie l'angle à la hauteur $h$ et à la distance $d$, la trigonométrie donne la relation $\tan(\theta) = \frac{h}{d}$. La hauteur $h$ du bâtiment peut alors être calculée comme $h = d \, \tan(\theta)$.

% ICI shéma d'illustration

Ce chapitre vise à vous fournir une compréhension des principaux concepts trigonométriques.  
Vous serez  familiarisé avec les fonctions trigonométriques essentielles, notamment le sinus, le cosinus et la tangente.
L'objectif principal est d'être capable d'appliquer les relations trigonométriques pour résoudre des problèmes concrets, comme le calcul de distances, de hauteurs ou de pentes.
En combinant théorie et pratique, ce chapitre vous préparera à utiliser la trigonométrie comme un outil puissant et polyvalent dans votre future carrière en génie civil.


\section{Un point sur les unités d'angle}


La mesure des angles est essentielle dans de nombreux domaines, notamment en mathématiques, en physique, en ingénierie et en topographie. Il existe plusieurs unités pour exprimer les angles, chacune ayant ses avantages et ses applications spécifiques. Nous allons explorer les trois principales unités : les \textbf{radians}, les \textbf{degrés} et les \textbf{grads} (ou \textbf{gons}), en mettant en lumière leur utilisation, en particulier en topographie.


\textbullet\ Le \textbf{radian} est l'unité\footnote{Pour être tout à fait exact, le radian n'est pas considérée comme une unité en physique car il définit u rapport de longueur. Ici, on s'autorisera cet abus de language.} de mesure d'angle dans le système international (SI). Un radian est défini comme l'angle sous-tendu par un arc de cercle dont la longueur est égale au rayon du cercle. Autrement dit, si un arc de longueur $R$ est tracé sur un cercle de même rayon, l'angle correspondant au centre du cercle est de 1 radian.
Un cercle complet correspond à un angle de $2\pi$ radians, car la circonférence d'un cercle est $2\pi R$. Ainsi,
$360^\circ$ représente $2\pi$~radians.
On voit alors que l'unité de radian pour un angle est naturel pour les calculs mathématiques, les radians simplifient les formules en analyse mathématique, notamment pour les dérivées et les intégrales des fonctions trigonométriques.


\textbullet\  Le \textbf{degré} est une unité de mesure d'angle largement répandue. Un degré est défini comme $\frac{1}{360}$ d'un cercle complet. Ainsi, un cercle complet mesure $360^\circ$.
%
Par exemple, en navigation ou cartographie, les degrés sont utilisés pour exprimer les latitudes et longitudes. De façon générale, les degrés sont souvent utilisés dans les cours de géométrie pour leur simplicité et leur \og familiarité \fg.
Pour convertir des degrés en radians, il suffit de multiplier par $\pi/180$.
Par exemple, $180^\circ = \pi$ radians.


\textbullet\   Le \textbf{grad} (ou \textbf{gon} du grec \textit{gônia} qui signifie angle) est une unité de mesure d'angle moins courante, mais très utile en topographie. Un gon est défini comme \( \frac{1}{400} \) d'un cercle complet. 
Les gons permettent des calculs plus précis pour les mesures angulaires sur le terrain.
Comme ils divisent le cercle en 400 unités, ils sont compatibles avec le système métrique décimal, ce qui peut faciliter les calculs et les conversions.

Pour convertir des gons en degrés ou en radians, on utilise les relations suivantes :
%
\begin{equation*}
1 \text{ gon} = 0.9^\circ = \frac{\pi}{200} \text{ radians}
\end{equation*}
%
Ainsi, un angle droit mesure  \num{100} grades ou $\frac{\pi}{2}$ radians.


\begin{bclogo}[arrondi=0.1,barre=snake,noborder=true,logo=\bclampe]{Astuce}
Pour utiliser correctement les fonctions trigonométriques sur une calculatrice de collège, 
il faut vérifier le mode d'angle sélectionné. Les calculatrices de collège proposent au moins deux modes~: les degrés (\textsc{Deg}) et les radians (\textsc{Rad}). Certaines calculatrices offrent aussi le mode des grades (\textsc{Grad}).
%
Une erreur de mode peut donner des résultats totalement faux~: par exemple, $\sin(30)$ affichera \texttt{0.5} en mode \textsc{Deg} (pour $30^\circ$), mais environ \texttt{-0.988} en mode \textsc{Rad} (pour $30$~radians). Pensez toujours à vérifier ce réglage avant de commencer vos calculs !
\end{bclogo}


\begin{exo}
Convertir : ...
\end{exo}

\begin{sol}
blabla
\end{sol}



\section{Les fonctions sinus, cosinus et tangente}



Les fonctions trigonométriques sont des outils fondamentaux en mathématiques, particulièrement en géométrie et en analyse. Elles permettent de décrire les relations entre les angles et les côtés des triangles, ainsi que de modéliser des phénomènes périodiques comme les ondes ou les mouvements circulaires.


\begin{bclogo}[arrondi=0.1,barre=snake,noborder=true,logo=\bcattention]{Parenthèses}
Parfois, on peut écrire une fonction trigonométrique sans mettre de parenthèses, à condition qu'il n'y ait pas de risque de confusion. Par exemple, il est courant d'écrire $\cos \theta$ plutôt que $\cos(\theta)$, car on comprend facilement que l'angle est $\theta$.
En revanche, on évite d'écrire $\cos \omega t$ à la place de $\cos(\omega t)$, car cela pourrait prêter à confusion : on pourrait croire qu'il s'agit de $\cos\omega$ multiplié par $t$.
\end{bclogo}

\begin{bclogo}[arrondi=0.1,barre=snake,noborder=true,logo=\bcattention]{Carré du cosinus}
Toujours dans l'idée d'éviter les confusions, on utilise une notation particulière pour écrire le carré (ou une autre puissance) d'une fonction trigonométrique. Par exemple, on écrit généralement $\cos^2(x)$ plutôt que $\cos(x)^2$.
Cette façon d'écrire est plus courte, mais elle veut dire exactement la même chose : $\cos^2(x)$ signifie le carré du cosinus de $x$, c'est-à-dire $(\cos(x))^2$.
Attention à ne pas confondre $\cos^2(x)$ avec $\cos(x^2)$ ! Dans le premier cas, on élève le résultat du cosinus au carré, alors que dans le second, on calcule le cosinus du carré de $x$. Ce n'est pas du tout la même chose.
\end{bclogo}

\textbullet\  \textbf{La fonction sinus} associe à un angle $\theta$ (exprimé dans l'unité de son choix) le rapport entre la longueur du côté opposé à cet angle et la longueur de l'hypoténuse dans un triangle rectangle. Mathématiquement, pour un angle $\theta$ dans un triangle rectangle, on a :
\begin{equation*}
\sin(\theta) = \frac{\text{côté opposé}}{\text{hypoténuse}}
\end{equation*}

La fonction sinus est  \textbf{périodique} de période $2\pi$, \textbf{impaire} (symétrique par rapport à l'origine), et définie pour tout réel $x \in \mathbb{R}$. Elle prend ses valeurs dans l'intervalle $[-1,\,1]$. 

\begin{center}
    \begin{tikzpicture}
        \begin{axis}[
            mathbook,
            width=\textwidth,
            height=6cm,
            xmin=-3*pi-1, xmax=3*pi+1,
            ymin=-1.2, ymax=1.2,
            xtick={-7.85, -4.71, -1.57, 0, 1.57, 4.71, 7.85},
            xticklabels={$-\frac{5\pi}{2}$, $-\frac{3\pi}{2}$, $-\frac{\pi}{2}$, $0$, $\frac{\pi}{2}$, $\frac{3\pi}{2}$, $\frac{5\pi}{2}$},
        ]
            \addplot[black,line width = 1.5pt,domain=-3*pi:3*pi]{sin(deg(x))};
        \end{axis}
    \end{tikzpicture}
\end{center}

Sa courbe, appelée \textit{sinusoïde}, oscille entre \num{-1} et \num{1} et passe par des \textit{maxima} en $x = \frac{\pi}{2} + 2k\pi$ (valeur \num{1}), des \textit{minima} en $x = \frac{3\pi}{2} + 2k\pi$ (valeur \num{-1}), et s'annule en $x = k\pi$ ($k \in \mathbb{Z}$). Elle est croissante sur $\left[-\frac{\pi}{2}, \frac{\pi}{2}\right]$ et décroissante sur $\left[\frac{\pi}{2}, \frac{3\pi}{2}\right]$, reflétant sa dérivée $\sin'(x) = \cos(x)$.


\textbullet\  \textbf{La fonction cosinus} associe à un angle $\theta$ le rapport entre la longueur du côté adjacent à cet angle et la longueur de l'hypoténuse dans un triangle rectangle. Mathématiquement, on écrit :
%
\begin{equation*}
\cos(\theta) = \frac{\text{côté adjacent}}{\text{hypoténuse}}
\end{equation*}
%
Comme la fonction sinus, la fonction cosinus est  \textbf{périodique} de période $2\pi$, \textbf{paire} (symétrique par rapport à l'axe des ordonnées), et définie pour tout réel $x$ de $\mathbb{R}$. Elle prend ses valeurs dans l'intervalle $[-1,\,1]$.

\begin{center}
    \begin{tikzpicture}
        \begin{axis}[
            mathbook,
            width=\textwidth,
            height=6cm,
            xmin=-3*pi-1, xmax=3*pi+1,
            ymin=-1.2, ymax=1.2,
            xtick={-9.42, -6.28, -3.14, 0, 3.14, 6.28, 9.42},
            xticklabels={$-3\pi$, $-2\pi$, $-\pi$, $0$, $\pi$, $2\pi$, $3\pi$},
        ]
            \addplot[black,line width = 1.5pt, domain=-3*pi:3*pi]{cos(deg(x))};
        \end{axis}
    \end{tikzpicture}
\end{center}

Sa courbe oscille entre -1 et 1 et passe par des \textit{maxima} en $x = 2k\pi$ (valeur \num{1}), des \textit{minima} en $x = \pi + 2k\pi$ (valeur \num{-1}), et s'annule en $x = \frac{\pi}{2} + k\pi$ ($k \in \mathbb{Z} $). Elle est décroissante sur $[0, \pi]$ et croissante sur $[\pi, 2\pi]$, reflétant sa dérivée $\cos'(x) = -\sin(x)$.


\textbullet\  \textbf{La fonction tangente} est définie comme le rapport entre le sinus  et le cosinus :
\begin{equation*}
\tan(\theta) = \frac{\sin(\theta)}{\cos(\theta)}
\end{equation*}
%
Elle représente également le rapport entre le côté opposé et le côté adjacent dans un triangle rectangle :
\begin{equation*}
\tan(\theta) = \frac{\text{côté opposé}}{\text{côté adjacent}}
\end{equation*}
%
Contrairement aux fonctions sinus et cosinus, la fonction tangente n'est pas bornée et présente des asymptotes verticales aux angles $x$ où $\cos(x) = 0$, donc pour $x = \frac{\pi}{2} + k\pi \quad \forall k \in \mathbb{Z}$.


\begin{center}
    \begin{tikzpicture}
        \begin{axis}[
            mathbook,
            width=\textwidth,
            height=6cm,
            xmin=-3*pi-1, xmax=3*pi+1,
            ymin=-4, ymax=4,
            xtick={-9.42, -6.28, -3.14, 0, 3.14, 6.28, 9.42},
            xticklabels={$-3\pi$, $-2\pi$, $-\pi$, $0$, $\pi$, $2\pi$, $3\pi$},
            tick label style={inner sep=1pt, font=\scriptsize},
            xticklabel style={yshift=-8pt},
            restrict y to domain=-4:4, % Limite l'affichage de la tangente
        ]
            \addplot[black,line width = 1.5pt,domain=-3*pi:3*pi,samples=5000]{tan(deg(x))};
            
        \end{axis}
    \end{tikzpicture}
\end{center}

La fonction est \textbf{périodique} de période $\pi$, \textbf{impaire} (symétrique par rapport à l'origine), et définie pour tout réel $x \neq \frac{\pi}{2} + k\pi$ ($k \in \mathbb{Z}$). Sa courbe présente des \textbf{asymptotes verticales} aux points $x = \frac{\pi}{2} + k\pi$ et oscille entre $-\infty$ et $+\infty$. Elle s'annule en $x = k\pi$ ($k \in \mathbb{Z}$) et est croissante sur chaque intervalle de son domaine de définition. Sa dérivée est donnée par $\tan'(x) = 1 + \tan^2(x)$.


% conseil
\begin{bclogo}[arrondi=0.1,barre=snake,noborder=true,logo=\bclampe]{Astuce}
Plutôt que de s'appuyer systématiquement sur le moyen mnémotechnique \og SOH~CAH~TOA \fg, il est préférable de travailler à mémoriser et comprendre les relations fondamentales dans un triangle rectangle.
Avec de la pratique, ces relations deviennent naturelles et leur utilisation intuitive et rapide. De façon générale, les astuces apprises avant le bac peuvent être très utiles pour débuter, mais l'objectif est de s'en détacher progressivement pour gagner en aisance et en profondeur mathématique, même si l'objectif n'est pas de devenir un grand mathématicien.
\end{bclogo}

\begin{exo}
Exercices dans des triangles rectangles
\end{exo}

\begin{sol}
Correction des exercices dans des triangles rectangles
\end{sol}

\begin{exo}

\begin{enumerate}
\item La fonction $x(t) = A \cos(\omega t + \phi)$ est-elle paire, impaire, ou ni l'une ni l'autre ? Justifier.

% pour formule d'addition
%\item Montrer que $\cos(\omega t + \phi)$ peut s'écrire comme une combinaison linéaire de $\sin(\omega t)$ et  $\cos(\omega t)$.

\item Quelle est la période $T$ de $x(t) = 3 \cos(4t + \pi/3)$ ?

\item La fonction $z(t) = \cos(t)+\sin(2t)$ est-elle périodique ? Si oui, quelle est sa période ? Calculer $z(\tfrac{\pi}{2})$ et $z(\pi)$.

\item Quelle est la période de $\tan(\omega t)$ ? La fonction est-elle paire, impaire, ou ni l'une ni l'autre ? Justifier.
\end{enumerate}

\end{exo}

\begin{sol}

\begin{enumerate}
\item La fonction $x(t) = A \cos(\omega t + \phi)$ n'est ni paire ni impaire en général, sauf dans des cas particuliers. Pour le montrer de façon purement graphique [...]

\item On veut trouver la plus petite durée $T > 0$ telle que $x(t+T) = x(t)\ \forall t$. En particulier, pour $t=0$, $x(T) = x(0)$ et donc :
\begin{equation*}
\cos(4T + \tfrac{\pi}{3}) = \cos(\tfrac{\pi}{3} + 2k\pi) \implies 4T = 2k\pi  \implies T = \frac{k\pi}{2}
\end{equation*}
La plus petite valeur strictement positive est $T=\tfrac{\pi}{2}$ en prenant $k=1$.
On retiendra que, de façon générale, la période de $\cos(\omega t + \phi)$ ou de $\sin(\omega t + \phi)$ est $T = \frac{2 \pi}{\omega}$. 

\item On sait que la période de $\cos(t)$ est  $T_1= 2 \pi$, et que la période de $\sin(2t)$ est $T_2=\frac{2\pi}{2} = \pi$. La période $T_1$ est un multiples de $T_2$, $T_1 = 2 T_2$. Ce rapport d'exactement 2, un nombre entier, rend la fonction $z(t)$ également périodique, de période $2\pi$ (plus petit multiple commun à $T_1$ et $T_2$).

$z(\tfrac{\pi}{2}) = \cos(\tfrac{\pi}{2}) + \sin(2 \tfrac{\pi}{2}) = 0 + \sin(\pi) = 0$

$z(\pi) = \cos(\pi) + \sin(2 \pi) = (-1) + 0 = -1$

\item On sait que la fonction tangente est périodique de période $\pi$, donc $\tan(x+\pi) = \tan(x)$. Par conséquent, en remplaçant $x$ par $\omega t$, on obtient $\tan(\omega t+\pi) = \tan(\omega t)$. Cherchons, $T>0$ tel que $\tan(\omega (t+T)) = \tan(\omega t+\omega T) = \tan(\omega t)$. On identifie alors que $\omega T = \pi$, donc $T=\frac{\pi}{\omega}$.

La fonction $\tan(\omega t)$ est impaire comme l'est la fonction tangente. 
\end{enumerate}


\end{sol}


\section{Le cercle trigonométrique} % est un très bon ami !


Le cercle trigonométrique est un outil fondamental pour relier les notions d'angles et de fonctions trigonométriques. 
Il constitue un point de passage essentiel entre la géométrie et l'analyse, et joue un rôle clé dans de nombreuses applications comme  la mesure d'angles, le calcul de pentes, la topographie, et la modélisation de phénomènes périodiques (oscillatoires).


On appelle \textbf{cercle trigonométrique} le cercle centré à l'origine d'un repère orthonormé et de rayon unitaire. 
Le sens positif de rotation est le sens \textbf{anti-horaire}.  
Chaque point $M$ de ce cercle correspond à un angle orienté $\theta$ dont  les coordonnées  sont directement liées aux valeurs du cosinus et du sinus comme illustré ci-après.

\begin{center}
\begin{tikzpicture}

\cercletrigo{2cm}{0.9cm}
\draw[line width=1pt, black, line cap=round] (0,0) -- (53:2);
\draw[>=angle 60, ->, line width=1pt, line cap=round] (0:0.8) arc (0:53:0.8);
\draw ({53/2}:1) node[]{$\theta$};
\draw ({cos(53)*2},0) node[below]{$\cos \theta$};
\draw (0,{sin(53)*2}) node[left]{$\sin \theta$};

\draw (3,0) node[right]{\og cosinus \fg};
\draw (0,3) node[above]{\og sinus \fg};

\draw (-2,0) node[fill=white, left]{-1};
\draw (2,0) node[fill=white, right]{1};
\draw (0,-2) node[fill=white, below]{-1};
\draw (0,2) node[fill=white, above]{1};

\draw[line width=1pt, black, line cap=round, dotted] ({cos(53)*2},{sin(53)*2}) -- ({cos(53)*2},0);
\draw[line width=1pt, black, line cap=round, dotted] ({cos(53)*2},{sin(53)*2}) -- (0,{sin(53)*2});

\draw[>=angle 60, ->, line width=1pt, line cap=round] (20:2.8) arc (20:60:2.8);
\draw[>=angle 60, ->, line width=1pt, line cap=round] (-20:2.8) arc (-20:-60:2.8);
\draw (40:3.2) node[]{\textcircled{+}};
\draw (-40:3.2) node[]{\textcircled{-}};

\end{tikzpicture}
\end{center}

% il faut commenter le cercle

Le cercle trigonométrique doit être considéré comme un \og très bon ami \fg\ parce qu'il permet de visualiser intuitivement les relations entre les fonctions trigonométriques et les angles.  
Grâce à lui, on comprend d'un simple coup d'œil pourquoi $\sin^2\theta + \cos^2\theta = 1$, 
ou encore comment les signes de $\sin\theta$ et $\cos\theta$ varient selon le quadrant considéré.  
Dans de nombreuses situations, revenir au cercle trigonométrique 
permet de vérifier rapidement la cohérence d'un calcul ou d'une mesure angulaire.


\section{Angles remarquables}

Les angles remarquables sont des angles pour lesquels les valeurs des fonctions trigonométriques $\sin$, $\cos$ et $\tan$ peuvent être déterminées exactement, souvent à l'aide de triangles particuliers ou de propriétés géométriques. Voici les angles les plus courants et leurs valeurs associées :


Ces angles ($0^\circ$, $30^\circ$, $45^\circ$, $60^\circ$ et $90^\circ$) sont souvent étudiés dans le contexte d'un triangle rectangle ou d'un cercle trigonométrique. Leurs valeurs trigonométriques peuvent être déterminées à l'aide des triangles suivants :

% AJOUTER ????


\begin{center}
\begin{tabular}{ccccc}
\toprule
\multicolumn{2}{c}{Angle $\theta$}  & & & \\
\cmidrule{1-2}
(degré) & (radian) & $\sin\theta$ & $\cos\theta$ & $\tan\theta$ \\
\midrule
$0^\circ$ & 0 & 0 & 1 & 0 \\[2mm]
$30^\circ$ & $\dfrac{\pi}{6}$ & $\dfrac{1}{2}$ & $\dfrac{\sqrt{3}}{2}$ & $\dfrac{1}{\sqrt{3}}$ \\[2mm]
$45^\circ$ & $\dfrac{\pi}{4}$ & $\dfrac{\sqrt{2}}{2}$ & $\dfrac{\sqrt{2}}{2}$ & 1 \\[2mm]
$60^\circ$ & $\dfrac{\pi}{3}$ & $\dfrac{\sqrt{3}}{2}$ & $\dfrac{1}{2}$ & $\sqrt{3}$ \\[2mm]
$90^\circ$ & $\dfrac{\pi}{2}$ & 1 & 0 & -- \\[2mm]
\bottomrule
\end{tabular}
\end{center}


Ces valeurs sont souvent mémorisées pour faciliter les calculs et sont essentielles pour comprendre les propriétés des fonctions trigonométriques.
Ces angles remarquables peuvent être utiles pour simplifier les calculs trigonométriques, résoudre des équations trigonométriques, ou encore déterminer des valeurs exactes dans des problèmes géométriques ou physiques.


% astuces ne pas retenir bêtement, mais remarquer qu'il n'y a que peu de valeurs

\begin{center}
\begin{tikzpicture}

\cercletrigo{1.5cm}{0.4cm}
\draw[line width=1pt, black, line cap=round] (0,0) -- (45:1.5);
%\draw[line width=1pt, black, line cap=round] (0,1.5) -- (45:1.5);
\draw (45:1.8) node[]{$\tfrac{\pi}{4}$};
\draw[line width=1pt, black, line cap=round,dotted] (0,{sin(45)*1.5}) -- (45:1.5);
\draw[line width=1pt, black, line cap=round,dotted] ({cos(45)*1.5},0) -- (45:1.5);
\draw ({cos(45)*1.5},0)  node[below]{$\tfrac{1}{\sqrt{2}}$};
\draw (0,{sin(45)*1.5})  node[left]{$\tfrac{1}{\sqrt{2}}$};

\begin{scope}[xshift=4.5cm]
\cercletrigo{1.5cm}{0.4cm}
\draw[line width=1pt, black, line cap=round] (0,0) -- (30:1.5);
\draw[line width=0.5pt, black, line cap=round] (0,1.5) -- (30:1.5);
\draw (30:1.8) node[]{$\tfrac{\pi}{6}$};
\draw[line width=1pt, black, line cap=round,dotted] (0,{sin(30)*1.5}) -- (30:1.5);
\draw[line width=1pt, black, line cap=round,dotted] ({cos(30)*1.5},0) -- (30:1.5);
\draw ({cos(30)*1.5 - 0.1},0)  node[below]{$\tfrac{\sqrt{3}}{2}$};
\draw (0,{sin(30)*1.5})  node[left]{$\tfrac{1}{2}$};
\end{scope}

\begin{scope}[xshift=9cm]
\cercletrigo{1.5cm}{0.4cm}
\draw[line width=1pt, black, line cap=round] (0,0) -- (60:1.5);
\draw[line width=0.5pt, black, line cap=round] (1.5,0) -- (60:1.5);
\draw (60:1.8) node[]{$\tfrac{\pi}{3}$};
\draw[line width=1pt, black, line cap=round,dotted] (0,{sin(60)*1.5}) -- (60:1.5);
\draw[line width=1pt, black, line cap=round,dotted] ({cos(60)*1.5},0) -- (60:1.5);
\draw ({cos(60)*1.5},0)  node[below]{$\tfrac{1}{2}$};
\draw (0,{sin(60)*1.5-0.2})  node[left]{$\tfrac{\sqrt{3}}{2}$};
\end{scope}

\end{tikzpicture}
\end{center}

%\subsection{Autres angles remarquables}

D'autres angles, comme $18^\circ\,(\frac{\pi}{10})$, $36^\circ\,(\frac{\pi}{5})$, $54^\circ\,(\frac{3\pi}{10})$ et $72^\circ\,(\frac{2\pi}{5})$, sont également remarquables et peuvent être étudiés à l'aide de triangles d'or ou de pentagones réguliers. Par exemple :
%
\begin{itemize}
    \item[\ding{42}] $\sin(18^\circ) = \frac{\sqrt{5} - 1}{4}, \quad \cos(18^\circ) = \frac{\sqrt{10 + 2\sqrt{5}}}{4}$
    \item[\ding{42}] $\sin(36^\circ) = \frac{\sqrt{10 - 2\sqrt{5}}}{4}, \quad \cos(36^\circ) = \frac{\sqrt{5} + 1}{4} $
\end{itemize}
%
Bien entendu, il n'est absolument pas nécessaire de mémoriser toutes ces formules, et nous verrons plus loin comment ils peuvent être établis.

%\subsection{Utilisation des angles remarquables}

\begin{exo}
\noindent \ding{45} Donner la valeur exacte des expressions suivantes :

\begin{tasks}(3)
\task $\sin(30^\circ)$
\task $ \cos(\tfrac{\pi}{3})$
\task $\sin(30^\circ) + \cos(60^\circ)$
\task $\dfrac{\sin(60^\circ)}{\cos(30^\circ)}$
\task $\cos(\pi + \tfrac{\pi}{3})$
\task $\cos^2(30^\circ) + \sin^2(30^\circ)$
\end{tasks}

\end{exo}

\begin{sol}

\begin{tasks}(2)
\task $\sin(30^\circ) = \dfrac{\sqrt{3}}{2}$
\task $ \cos(\tfrac{\pi}{3}) = \dfrac{1}{2}$
\task $\sin(30^\circ) + \cos(60^\circ) = \dfrac{1}{2}+\dfrac{1}{2}= 1$
\task $\dfrac{\sin(60^\circ)}{\cos(30^\circ)}$
\task $\cos(\pi + \tfrac{\pi}{3})$
\task $\cos^2(30^\circ) + \sin^2(30^\circ)$
\end{tasks}

\end{sol}


\section{Angles associés}

En trigonométrie, les angles associés permettent de relier les valeurs des fonctions trigonométriques pour différents angles en utilisant des propriétés géométriques et algébriques. Ces relations sont particulièrement utiles pour simplifier des expressions ou résoudre des équations trigonométriques.
%
Par exemple, en utilisant la relation de complémentarité, l'expression $\sin(\tfrac{\pi}{2} - x) + \cos(x)$ peut se simplifier en $ \cos(x) + \cos(x) = 2\cos(x)$.

%\subsection{Relations fondamentales entre angles associés}

Les angles associés sont souvent obtenus par des transformations géométriques comme la complémentarité, la supplémentarité, l'opposition, la périodicité ou la symétrie (cette dernière notion étant liée à la parité). Voici quelques relations parmi les plus importantes :

\begin{center}
\begin{tabular}{ccc}
\toprule
 sinus  & cosinus & tangent \\
\midrule
 $\sin(\tfrac{\pi}{2}-a) = \cos(a)$ & $\cos(\tfrac{\pi}{2}-a)=\sin(a)$ & $\tan(\tfrac{\pi}{2}-a) = \cot(a)$ \\
 $\sin(\pi - a) = \sin(a)$ & $\cos(\pi - a) = -\cos(a)$ & $\tan(\pi - a) = -\tan(a)$ \\
 $\sin(\pi + a) = -\sin(a)$ & $\cos(\pi + a) = -\cos(a)$ & $\tan(\pi + a) = \tan(a)$ \\
 $\sin(a + 2\pi) = \sin(a)$ & $\cos(a + 2\pi) = \cos(a)$ & $\tan(a + \pi) = \tan(a)$ \\
 $\sin(-a) = -\sin(a)$ & $\cos(-a) = \cos(a)$ & $\tan(-a) = -\tan(a)$ \\
\bottomrule
\end{tabular}
\end{center}


\begin{bclogo}[arrondi=0.1,barre=snake,noborder=true,logo=\bcattention]{Cotangente}
La fonction cotangente, notée $\cot(x)$, correspond à l'inverse de la tangente : $\cot(x) = \tfrac{1}{\tan(x)}$. Il ne faut pas la confondre avec $\tan^{-1}(x)$, qui ne désigne pas une puissance inverse mais la fonction réciproque de la tangente, appelée aussi $\arctan(x)$. Autrement dit, $\cot(x)$ représente l'inverse de la tangente de $x$, tandis que $\tan^{-1}(x)$ (ou $\arctan(x)$) désigne l'angle dont la tangente vaut $x$. La fonction $\arctan$ sera étudiée plus loin.
\end{bclogo}

Dans le tableau précédent, les transformations sont données dans l'ordre suivant : complémentarité,  supplémentarité, opposition, périodicité et symétrie.
Ces termes sont bien sûr utiles, mais plutôt que de s'enfermer dans des définitions, explorons comment le cercle trigonométrique peut révéler, d'un seul regard, toutes ces correspondances.



%\subsection{Visualisation sur le cercle unité}

Les relations des angles associés peuvent être visualisées sur le cercle unité. Par exemple, l'angle \(\frac{\pi}{2} - a\) est le reflet de l'angle \(a\) par rapport à l'axe vertical du cercle unité, ce qui explique pourquoi \(\cos\left(\frac{\pi}{2} - a\right) = \sin(a)\).
[...]

\begin{comment}
\subsection{Cas particuliers}

- **Angles négatifs** : Grâce à la symétrie, on peut exprimer les fonctions trigonométriques d'angles négatifs en termes d'angles positifs.
- **Angles supérieurs à \(2\pi\)** : La périodicité des fonctions trigonométriques permet de ramener l'étude de ces angles à celle d'angles dans l'intervalle \([0, 2\pi]\).
\end{comment}

\begin{exo}
\noindent Simplifier les expressions suivantes

\begin{tasks}(2)
\task $\dfrac{\sin(\frac{\pi}{2} - \theta)}{\cos(\pi+\theta)}$
\task $\dfrac{\sin(\pi- x)}{\cos(\pi+x)}$
\task $\dfrac{\sin(\frac{3\pi}{2} + x) \cos(\frac{\pi}{2})}{\tan(\pi- x) \sin(-x)}$
\task $\sin(\tfrac{5\pi}{2} - x) \cos(-x)$
\end{tasks}

\end{exo}


\begin{sol}

\begin{tasks}(2)
\task $\dfrac{\sin(\frac{\pi}{2} - \theta)}{\cos(\pi+\theta)} = -1$
\task $\dfrac{\sin(\pi- x)}{\cos(\pi+x)}  = -\tan(x)$
\task $\dfrac{\sin(\frac{3\pi}{2} + x) \cos(\frac{\pi}{2})}{\tan(\pi- x) \sin(-x)} = \dfrac{-\cos^2 (x)}{\sin(x)}$
\task $\sin(\tfrac{5\pi}{2} - x) \cos(-x) = \cos^2(x)$
\end{tasks}

\end{sol}




\section{Relations dans un triangle quelconque}


\subsection{Relation des sinus}

% dessin

\begin{equation*}
\dfrac{\sin (\alpha)}{a} = \dfrac{\sin (\beta)}{b} = \dfrac{\sin (\gamma)}{c} = \dfrac{2S}{abc}  
\end{equation*}


\subsection{Théorème d'Al Kashi}

% dessin

\begin{equation*}
\begin{cases}
 c^2 &= a^2 + b^2 - 2 a b \cos (\gamma) \\
 a^2 &= b^2 + c^2 - 2 b c \cos (\alpha) \\
 b^2 &= c^2 + d^2 - 2 c d \cos (\beta) 
\end{cases}  
\end{equation*}


\section{Principales formules trigonométriques}

\subsection{Pythagore dans le cercle trigonométrique}



\begin{equation*}
\cos^2(x) + \sin^2(x) = 1
\end{equation*}



\subsection{Formules d'addition}

\begin{equation*}
\vv{OA} \cdot \vv{OB} = \cos (b-a) = ...
\end{equation*}



\subsection{Formules de duplication}

À partir de formule d'addition
%
\begin{equation*}
\cos(2x) = \cos(x+x) = \cos(x)\cos(x) - \sin(x)\sin(x) 
\end{equation*}
%
d'où
%
\begin{equation*}
\boxed{\cos(2x)= \cos^2(x) - \sin^2(x)}
\end{equation*}
%
En introduisant le fait que $\cos^2(x) + \sin^2(x)=1$ on peut obtenir deux autres expressions :
\begin{equation*}
\cos(2x)= \cos^2(x) - \sin^2(x) + 1 - 1 = \cos^2(x) - \sin^2(x) + \left( \cos^2(x) + \sin^2(x) \right) - 1 
\end{equation*}
%
Ce qui conduit à :
%
\begin{equation*}
\boxed{
\cos(2x)=  2\cos^2(x)   - 1 
\quad\text{ou}\quad
\cos(2x)=  1 +  2\sin^2(x)
}
\end{equation*}


Blabla sin(2x) à partir de formule d'addition
%
\begin{equation*}
\sin(2x) = \sin(x+x) = \sin(x)\cos(x) + \sin(x)\cos(x) 
\end{equation*}
%
d'où
%
\begin{equation*}
\sin(2x)= 2\sin(x)\cos(x)
\end{equation*}

Ces formules sont très utiles pour simplifier des expressions.


% ICI tableau qui resume les formules



\begin{center} 
\begin{tabular}{l} 
\toprule 
Addition \\ 
\midrule 
$\sin(a+b) = \sin(a)\cos(b) + \cos(a)\sin(b)$ \\ [2mm] 
$\cos(a+b) = \cos(a)\cos(b) - \sin(a)\sin(b)$ \\[2mm]
\midrule 
Soustraction \\ 
\midrule 
$\sin(a-b) = \sin(a)\cos(b) - \cos(a)\sin(b)$ \\[2mm]  
$\cos(a-b) = \cos(a)\cos(b) + \sin(a)\sin(b)$ \\[2mm]
 \midrule 
 Duplication \\ 
 \midrule 
 $\sin(2a) = 2\sin(a)\cos(a)$ \\ [2mm] 
 $\cos(2a) = \begin{cases}
 \cos^2(a) - \sin^2(a) \\
  2\cos^2(a) - 1 \\
   1 - 2\sin^2(a)
   \end{cases}$ \\[2mm] 
 \midrule 
 Réduction \\ 
 \midrule 
 $\sin(\tfrac{a}{2}) = 
\langle \sin(\tfrac{a}{2})\rangle ^\pm \, 
 \sqrt{\dfrac{1 - \cos(a)}{2}}$ \\ [2mm]
 $\cos\!\left(\tfrac{a}{2}\right) = 
 \langle \sin(\tfrac{a}{2})\rangle ^\pm \,
 \sqrt{\dfrac{1 + \cos(a)}{2}}$ \\ 
 \bottomrule 
 \end{tabular} 
 \end{center}




\begin{exo}
\noindent  En utilisant les différentes formules trigonométriques :
%
\begin{itemize}
\item[\ding{42}]  Soit un angle $x$ tel que 
$\tan(x) = \frac{\sin(x)}{\cos(x)} = \frac{3}{4}$. 
Calculer la valeur de $\sin^2(x) - \cos^2(x)$.

\item[\ding{42}]  Sachant que $\sin(x) = \tfrac{2}{3}$, calculer la valeur de exacte de $\cos^2(x)$ sans utiliser de calculatrice.

\item[\ding{42}]  En utilisant la formule de duplication, simplifie l'expression suivante : $\cos(2x)$:
$\cos^2(3x) -  \sin^2(3x)$

\item[\ding{42}]  À partir de la valeur exacte du sinus et du cosinus de l'angle $\tfrac{\pi}{6}$, et en utilisant les formules de duplication, calculer les
valeurs exactes de $\cos(\tfrac{\pi}{12})$ et $\sin(\tfrac{\pi}{12})$.

\item[\ding{42}]  Calculer la valeur exacte de $\sin(2x)$ si $\sin(x) = \frac{3}{5}$ et $\cos(x) = \frac{4}{5}$ 
\end{itemize}
\end{exo}

\begin{sol}

TODO


\end{sol}



\section{Équations trigonométriques : cas des égalités de sinus, cosinus et tangente}



Pour résoudre l'équation $\cos(a) = \cos(b)$, les solutions générales sont données par :
\begin{equation*}
\begin{cases}
a &= b + 2k\pi \\ 
\text{ou}& \\
a &= -b + 2k\pi 
\end{cases}
\qquad \forall k \in \mathbb{Z}
\end{equation*}


Pour résoudre l'équation $\sin(a) = \sin(b)$, les solutions générales sont données par :
\begin{equation*}
\begin{cases}
a &= b + 2k\pi \\ 
\text{ou}& \\
a &= (\pi - b) + 2k\pi 
\end{cases}
\qquad  \forall k \in \mathbb{Z}
\end{equation*}


\begin{exo}
\end{exo}

\begin{sol}
\end{sol}


\section{Inégalités trigonométriques}

% méthode graphique basée sur le cercle trigo




\section{Fonctions trigonométriques réciproques}






 % en cours d'écriture
%!TEX root = main.tex


\newcommand{\veccol}[2]{%
  \begin{pmatrix}
    #1 \\
    #2
  \end{pmatrix}
}



\chapter{Vecteurs dans le plan}
\label{chapter:vecteurs}


\section{Introduction}

Les vecteurs sont des outils mathématiques fondamentaux qui permettent de représenter des grandeurs ayant à la fois une \textbf{direction}, un \textbf{sens} et une \textbf{intensité}. Dans le plan, les vecteurs sont utilisés pour modéliser des forces, des déplacements, des vitesses et bien d'autres phénomènes physiques. Pour les ingénieurs civils, la maîtrise des vecteurs est essentielle, car ils interviennent dans presque tous les aspects de la conception, de l'analyse et de la construction des infrastructures.

Que ce soit pour calculer les forces agissant sur un pont, déterminer les contraintes dans une structure ou optimiser la disposition des éléments d'un bâtiment, les vecteurs offrent une méthode claire et précise pour analyser et résoudre des problèmes complexes.
Illustrons cela avec un pont suspendu : ses câbles ont pour rôle de supporter à la fois le poids de la chaussée et celui des véhicules qui y circulent. Pour étudier les forces mises en jeu, les ingénieurs utilisent la notion de vecteurs. Chaque câble exerce une force de tension, représentée par un vecteur dirigé selon l'axe du câble.
En appliquant les fonctions trigonométriques (abordées au chapitre précédent), cette force de tension peut être décomposée en deux composantes : une composante horizontale et une composante verticale. Pour assurer la stabilité du pont, il est indispensable que la résultante de l’ensemble des forces vectorielles, incluant notamment le poids du tablier, soit nulle (même si d’autres critères entrent en compte, nous nous concentrons ici sur cet équilibre). Cette condition permet alors de déterminer avec précision la tension nécessaire dans chaque câble pour contrebalancer le poids de l’ensemble de la structure.


Ce chapitre a pour objectif de vous offrir une bonne compréhension des vecteurs dans le plan pour une application concrète en génie civil. 
Vous y découvrirez comment définir et représenter des vecteurs avec ses coordonnées cartésiennes ou polaires, et vous aborderez les opérations fondamentales qui leur sont associées, comme l’addition, la soustraction, la multiplication par un scalaire, ou encore le calcul de normes et de distances. Vous explorerez également deux outils clés : le produit scalaire, utile pour déterminer des angles ou des projections, et le produit vectoriel (dans le plan, lié à la notion de déterminant), qui permet d'évaluer des aires ou des orientations par exemple. 
Nous aborderons plus loin dans cet ouvrage une étude approfondie des vecteurs en trois dimensions, où ces notions prendront toute leur ampleur.


\section{Composantes cartésiennes}

Un vecteur $\vv{v}$ dans le plan peut être représenté par une paire de valeurs $x$ et  $y$ que l'on appel des \textbf{composantes} cartésiennes. On note généralement ces composantes en colonne 
%
\begin{equation*}
\vv{v} = \veccol{v_x}{v_y} \qquad \text{ou bien en ligne} \quad \vv{v} = (v_x\ v_y)^T
\end{equation*}
%
On verra au chapitre XX qu'il s'agit en fait d'une \textbf{transposition}, mais prenons le pour le moment comme une astuce de notation permettant une écriture plus compacte.

La notation $\vv{AB}$ d'un vecteur spécifie explicitement que ce vecteur part du point $A$ vers le point $B$. 
Pour identifier la position spatiale d'un point $P$, on peut utiliser les composantes du vecteur $\vv{OP}$. Ce point $O$ est généralement considéré comme l'origine. Il est courant de confondre la notion de coordonnées avec celle de composantes, mais ce n'est pas une grande faute, car cela n'apporte pas de confusion, ainsi la coordonnée ($x$, $y$) se notera :
%
\begin{equation*}
\vv{OP} = \veccol{x}{y}  \quad \text{ou de façon équivalente} \quad P \veccol{x}{y}
\end{equation*}


% extrémité - origine



\section{Somme, norme, produit scalaire et produit vectoriel}

La \textbf{somme} de deux vecteurs $\vv{v}$ et $\vv{w}$ est le vecteur dont les composantes cartésiennes sont la somme des composantes respectives de $\vv{v}$ et $\vv{w}$. Par exemple :
%
\begin{equation*}
\vv{v} + \vv{w} = \veccol{v_x}{v_y}  + \veccol{w_x}{w_y} = \veccol{v_x + w_x}{v_y + w_y}
\end{equation*}
%
Ceci est assez intuitif et bien entendu valable pour une soustraction.

% Norme




Le \textbf{produit scalaire} de deux vecteurs $\vv{v}$ et $\vv{w}$ est un nombre réel obtenu en multipliant les composantes respectives de $\vv{v}$ et $\vv{w}$ : 
%
\begin{equation*}
\vv{v} \cdot \vv{w} = \veccol{v_x}{v_y}  \cdot \veccol{w_x}{w_y} = v_x \times w_x + v_y \times w_y
\end{equation*}

% Bof
Ce produit scalaire peut être vu comme une mesure du déplacement d'un vecteur dans le sens opposé à l'autre. Plus le produit scalaire est grand, plus les vecteurs sont directs ; plus il est nul ou inversement, plus les vecteurs sont orthogonaux.



% produit vectoriel
Le \textbf{produit vectoriel} de deux vecteurs $\vec{v}$ et $\vec{w}$ du plan est défini comme un vecteur perpendiculaire au plan, dont la norme correspond à l’aire du parallélogramme formé par $\vec{v}$ et $\vec{w}$, et dont le sens dépend de l’ordre des vecteurs.
%
On définit :
%
\begin{equation*}
\vv{v} \wedge \vv{w} =  \veccol{v_x}{v_y}  \wedge \veccol{w_x}{w_y} =  (v_x w_y - v_y w_x)\, \vv{k}
\end{equation*}
%
où $\vv{k}$ désigne le vecteur unitaire sortant perpendiculaire au plan.
Ainsi,
le vecteur $\vv{v} \wedge \vv{w}$ pointe dans le sens de $\vv{k}$ si la rotation de $\vv{v}$ vers $\vv{w}$ est \textbf{anti-horaire} ;
il pointe dans le sens opposé si la rotation est \textbf{horaire} ;
et il est nul si $\vv{v}$ et $\vv{w}$ sont colinéaires (nous reviendrons sur ce point un peu plus loin).


La norme de ce produit vectoriel est donc 
$\Vert \vv{v} \wedge \vv{w} \Vert =  \vert v_x w_y - v_y w_x \vert$
ce qui correspond à l’aire du parallélogramme construit sur $\vv{v}$ et $\vv{w}$.




\section{Composantes polaires}



Exprimer des composante cartésiennes à partir de composantes polaires :

\begin{equation*}
\begin{cases}
x = r \cos(\theta) \\
y = r \sin(\theta) 
\end{cases}
\end{equation*}

Exprimer des composante polaires à partir de composantes cartésiennes :

\begin{equation*}
\begin{cases}
r = \sqrt{x^2 + y^2} \\
\theta  = \text{arctan}\left(\dfrac{y}{x} \right) + 
\begin{array}{|c|c|} \hline
\pi & 0 \\ \hline
-\pi & 0 \\ \hline
\end{array}
\qquad (x \ne 0 \text{ et } y \ne 0)
\end{cases}
\end{equation*}
%
où
%
\begin{equation*} 
\begin{array}{|c|c|} \hline
\pi & 0 \\ \hline
-\pi & 0 \\ \hline
\end{array}
= 
\begin{cases}
\pi & \text{si } x < 0 \text{ et } y > 0 \\
0 & \text{si } x > 0 \text{ et } y > 0 \\
0 & \text{si } x > 0 \text{ et } y < 0 \\
-\pi & \text{si } x < 0 \text{ et } y < 0 \\
\end{cases}
\end{equation*}

Pour déterminer l'angle polaire $\theta$ lorsque $x=0$ ou $y=0$, il suffit de déduire intuitivement l'angle en se représentant le vecteur par la pensée :
%
\begin{equation*} 
\begin{cases}
\theta = 0 & \text{si } x > 0 \text{ et } y = 0 \\
\theta = \frac{\pi}{2}  & \text{si } x = 0 \text{ et } y > 0 \\
\theta = \frac{\pi}{2}  & \text{si } x = 0 \text{ et } y < 0 \\
\theta = \pm \pi & \text{si } x < 0 \text{ et } y = 0 \\
\end{cases}
\end{equation*}




\section{Droites dans le plan}

Pour définir une droite $(\Delta)$, il faut un point $A$ de cette droite et une direction. La position du point $A$ peut être donnée par un vecteur $\vv{OA}$, et la direction par un \textbf{vecteur directeur} $\vv{d}$. Ainsi, la coordonnée ($x$, $y$) de tout point $M$ appartenant  à la droite $(\Delta)$ s'obtient avec :
\begin{equation*}
\vv{OM} = \vv{OA} + k \vv{d}
\quad\text{où}\quad k \in \mathbb{R}
\end{equation*}





Intersection de 2 droites (il y a différentes solutions, on en ):

s'assurer que le point d'intersection existe
prendre l'équation cartésienne d'une première droite
Utiliser l'équation paramétrique de la seconde droite pour exprimer  x et y en fonction de k.
trouver k
réinjecter cette valeur dans l'équation paramétrique de la seconde droite pour obtenir les coordonnées du point d'intersection


\section{Projections orthogonales}






 % en cours d'écriture
%!TEX root = main.tex

\chapter{Géométrie analytique tridimensionnelle}
\label{chapter:geospace}


\section{Introduction}



%%%%%%%%%%%%%%%%%%%%%%%%%%%%%%%%%%%%%%%%%%%
\section{Produits scalaire et vectoriel (rappels et extension tridimensionnelle)}


Il existe différentes solutions pour définir un vecteur. La solution la plus couramment rencontrée est la connaissance de ses composantes dans un système cartésien bidimensionnel (2D) ou tridimensionnel (3D). 
En physique, ces composantes auront la même unité que le vecteur ; par exemple, des mètres, des newtons, des pascals, etc.
Une autre solution, plutôt utilisée en configuration 2D, est de se donner \uuline{la norme}, \uline{le sens} et l'\uwave{inclinaison} du vecteur. Par exemple, on peut définir un vecteur force ainsi : ``une force \uwave{verticale} de \uuline{2~kN} \uline{vers le bas}''.


Le \textbf{produit scalaire} est une opération de multiplication entre deux vecteurs dont le résultat est un scalaire (c'est-à-dire un simple nombre). Le symbole de cette multiplication est un point.
%
\begin{enumerate}
\item Si les composantes sont connues
%
\begin{equation*}
\vv{a}\cdot \vv{b} = \begin{pmatrix} a_x\\a_y\\a_z \end{pmatrix} \cdot \begin{pmatrix} b_x\\b_y\\b_z \end{pmatrix}
= a_x \times b_x + a_y \times b_y + a_z \times b_z
\end{equation*}

\item Si les normes et les orientations des deux vecteurs sont connues
%
\begin{equation*}
\vv{a} \cdot \vv{b} = \Vert\vv{a}\Vert \times \Vert\vv{b}\Vert \times \cos(\vv{a},\vv{b})
\end{equation*}
\end{enumerate}
%
Le produit scalaire est commutatif : $\vv{a} \cdot \vv{b} = \vv{b} \cdot \vv{a}$. Il est utile en géométrie pour réaliser des projections. La valeur projetée du vecteur $\vv{a}$ sur une direction donnée par le vecteur unitaire $\vv{u}$ est $\vv{a} \cdot \vv{u}$.

La \textbf{norme} d'un vecteur est définie, de façon similaire aux simples nombres réels, comme ``la racine carré de son carré'' : 
\begin{equation*}
\Vert \vv{a} \Vert = \sqrt{\vv{a}^{\, 2}} = \sqrt{\vv{a}\cdot\vv{a}} = \sqrt{a_x^2+a_y^2+a_z^2}
\end{equation*}


\begin{exo}

\noindent \ding{45} Calculer la norme $\Vert \vv{v} \Vert = \sqrt{v_x^2 + v_y^2 + v_z^2}$ des vecteurs suivants pour en déduire le vecteur $\vv{u}_v$ de norme unitaire associé (on dira qu'on ``normalise'' le vecteur $\vv{v} \rightarrow \vv{u}_v = \vv{v}/\Vert\vv{v}\Vert$.

\begin{tasks}(3)
\task $\vv{v}=\begin{pmatrix} 3 \\ 2 \end{pmatrix}$
\task $\vv{v}=\begin{pmatrix} 2 \\ 4 \end{pmatrix}$
\task $\vv{v}=\begin{pmatrix} 2 \\ 2 \\ 2 \end{pmatrix}$
\task $\vv{v}=\begin{pmatrix} 3 \\ 2 \\ 1 \end{pmatrix}$
\task $\vv{v}=\begin{pmatrix} \pi \\ 2\pi \end{pmatrix}$
\end{tasks}

\end{exo}


\begin{sol}
\begin{tasks}(1)
\task $\left \Vert \begin{pmatrix} 3 \\ 2 \end{pmatrix} \right \Vert = \sqrt{9+4} = \sqrt{13}$ se normalise en $\begin{pmatrix} 3/\sqrt{13} \\ 2/\sqrt{13} \end{pmatrix}$
\task $\left \Vert\begin{pmatrix} 2 \\ 4 \end{pmatrix}\right \Vert = \sqrt{4+16} = \sqrt{20} = 2\sqrt{5}$ se normalise en $\begin{pmatrix} 1/\sqrt{5} \\ 2/\sqrt{5} \end{pmatrix}$
\task $\left \Vert\begin{pmatrix} 2 \\ 2 \\ 2 \end{pmatrix}\right \Vert = \sqrt{16} = 4$ se normalise en $\begin{pmatrix} 1/2 \\ 1/2 \\ 1/2 \end{pmatrix}$
\task $\left \Vert\begin{pmatrix} 3 \\ 2 \\ 1 \end{pmatrix}\right \Vert = \sqrt{9+4+1} = \sqrt{14}$ se normalise en $\begin{pmatrix} 3/\sqrt{14} \\ 2/\sqrt{14} \\ 1/\sqrt{14} \end{pmatrix}$
\task $\left \Vert\begin{pmatrix} \pi \\ 2\pi \end{pmatrix}\right \Vert = \sqrt{\pi^2+4\pi^2} = \pi\sqrt{5}$ se normalise en $\begin{pmatrix} 1/\sqrt{5} \\ 2/\sqrt{5} \end{pmatrix}$
\end{tasks}
\end{sol}



Pour calculer la valeur projetée $a_u$ d'un vecteur $\vv{a}$ sur une droite de vecteur directeur unitaire $\vv{u}$, il suffit de faire une multiplication scalaire : $a_u = \vv{a}\cdot\vv{u}$.

Si le vecteur directeur $\vv{d}$ de la droite sur laquelle on veut projeter n'est pas unitaire (c'est-à-dire que sa norme n'est pas égale à 1), alors il faut préalablement le normaliser : 
%
\begin{equation*}
a_d = \vv{a}\cdot \frac{\vv{d}}{\Vert \vv{d} \Vert} = \frac{\vv{a}\cdot\vv{d}}{\Vert \vv{d} \Vert}
\end{equation*}

Pour calculer le vecteur projeté $\vv{a}_u$ d'un vecteur $\vv{a}$ sur une droite de vecteur directeur unitaire $\vv{u}$, il suffit de multiplier le vecteur unitaire $\vv{u}$ par la valeur projetée $a_u$ : $\vv{a}_u = a_u \vv{u}$.

Si le vecteur directeur $\vv{d}$ de la droite sur laquelle on veut projeter n'est pas unitaire, alors il faut préalablement le normaliser (le rendre unitaire) :
%
\begin{equation*}
\displaystyle \vv{a}_d = a_u \times \frac{\vv{d}}{\Vert \vv{d} \Vert} = \frac{\vv{a}\cdot\vv{d}}{\Vert \vv{d} \Vert} \times \frac{\vv{d}}{\Vert \vv{d} \Vert} = \frac{(\vv{a}\cdot\vv{d})}{\Vert \vv{d} \Vert^2} \times \vv{d}
\end{equation*}


\begin{exo}

\noindent \ding{45} Déterminer la valeur projetée $v_{AB}$ du vecteur $\vv{v}$ sur la droite $(AB)$ à l'aide des données suivantes.

\begin{tasks}[item-indent=4pt, column-sep=2em](2)
\task $\vv{v} = \begin{pmatrix} 3 \\ 2 \end{pmatrix}$ et $\vv{AB} = \begin{pmatrix} -1 \\ 1 \end{pmatrix}$
\task $\vv{v} = \begin{pmatrix} -2 \\ 2 \end{pmatrix}$ et $\vv{AB} = \begin{pmatrix} 10 \\ 0 \end{pmatrix}$
\task $\vv{v} = \begin{pmatrix} 2 \\ 3 \\ 1 \end{pmatrix}$  et $\vv{AB} = \begin{pmatrix} 1 \\ 0 \\ -1 \end{pmatrix}$
\task $\vv{v} = \begin{pmatrix} 2 \\ 1 \\ 3 \end{pmatrix}$, $\vv{OA} = \begin{pmatrix} 1 \\ 1 \\ 1 \end{pmatrix}$ et $\vv{OB} = \begin{pmatrix} 1 \\ 0 \\ -1 \end{pmatrix}$
\end{tasks}

\end{exo}


\begin{sol}
\begin{tasks}(1)
\task $\displaystyle v_{AB}=\begin{pmatrix} 3 \\ 2 \end{pmatrix} \cdot \begin{pmatrix} -1/\sqrt{2} \\ 1/\sqrt{2} \end{pmatrix} = \frac{-3}{\sqrt{2}} +  \frac{2}{\sqrt{2}} =  \frac{-1}{\sqrt{2}}$
\task $\displaystyle v_{AB}=\begin{pmatrix} -2 \\ 2 \end{pmatrix} \cdot \begin{pmatrix} 1 \\ 0 \end{pmatrix} = -2$
\task $\displaystyle v_{AB}=\begin{pmatrix} 2 \\ 3 \\ 1 \end{pmatrix} \cdot \begin{pmatrix} 1/\sqrt{2} \\ 0  \\ -1/\sqrt{2} \end{pmatrix} = \frac{2}{\sqrt{2}} - \frac{1}{\sqrt{2}} = \frac{1}{\sqrt{2}} $
\task $\displaystyle v_{AB}=\begin{pmatrix} 2 \\ 1 \\ 3 \end{pmatrix} \cdot \begin{pmatrix} 0 \\ -1/\sqrt{5}  \\ -2/\sqrt{5} \end{pmatrix} = \frac{-1}{\sqrt{5}} - \frac{6}{\sqrt{5}} = \frac{-7}{\sqrt{5}} $
\end{tasks}
\end{sol}

\needspace{4cm}

\begin{exo}

\noindent \ding{45} Déterminer la force projetée $\vv{F}_{AB}$ de la force $\vv{F}$ sur la droite $(AB)$ à l'aide des données suivantes.

\begin{tasks}(2)
\task $\vv{F} = \begin{pmatrix} 2 \\ 3 \end{pmatrix}$ et $\vv{AB} = \begin{pmatrix} 2 \\ -1 \end{pmatrix}$
\task $\vv{F} = \begin{pmatrix} qa \\ T \end{pmatrix}$ et $\vv{AB} = \begin{pmatrix} 1/\sqrt{2} \\ 1/\sqrt{2} \end{pmatrix}$
\task $\vv{F} = \begin{pmatrix} 1 \\ 2 \\ 2 \end{pmatrix}$  et $\vv{AB} = \begin{pmatrix} -1 \\ 1 \\ 0 \end{pmatrix}$
\task $\vv{F} = \begin{pmatrix} -1 \\ 2 \\ 1 \end{pmatrix}$  et $\vv{AB} = \begin{pmatrix} 1 \\ -1 \\ 2 \end{pmatrix}$
\task*(2) $\vv{F} = \begin{pmatrix} qL \\ T \\ T\cos(\alpha) \end{pmatrix}$, $\vv{OA} = \begin{pmatrix} 1 \\ 3 \\ 3 \end{pmatrix}$ et $\vv{OB} = \begin{pmatrix} 2 \\ 3 \\ 2 \end{pmatrix}$
\end{tasks}

\end{exo}


\begin{sol}
\begin{tasks}(1)
\task $\displaystyle \vv{F}_{AB} = \left \{ \begin{pmatrix} 2 \\ 3 \end{pmatrix}\cdot \begin{pmatrix} 2/\sqrt{5} \\ -1/\sqrt{5} \end{pmatrix} \right \} \begin{pmatrix} 2/\sqrt{5} \\ -1/\sqrt{5} \end{pmatrix} = \frac{4-3}{\sqrt{5}} \begin{pmatrix} 2/\sqrt{5} \\ -1/\sqrt{5} \end{pmatrix} = \begin{pmatrix} 2/5 \\ -1/5 \end{pmatrix} $
\task $\displaystyle \vv{F}_{AB} = \left \{ \begin{pmatrix} qa \\ T \end{pmatrix}\cdot \begin{pmatrix} 1/\sqrt{2} \\ 1/\sqrt{2} \end{pmatrix} \right \} \begin{pmatrix} 1/\sqrt{2} \\ 1/\sqrt{2} \end{pmatrix} = \frac{qa+T}{\sqrt{2}} \begin{pmatrix} 1/\sqrt{2} \\ 1/\sqrt{2} \end{pmatrix} = \begin{pmatrix} (qa+T)/2 \\ (qa+T)/2 \end{pmatrix} = \frac{qa+T}{2} \begin{pmatrix} 1 \\ 1 \end{pmatrix}$
\task $\displaystyle \vv{F}_{AB} = \left \{ \begin{pmatrix} 1 \\ 2 \\ 2 \end{pmatrix}\cdot \begin{pmatrix} -1/\sqrt{2} \\ 1/\sqrt{2} \\ 0 \end{pmatrix} \right \} \begin{pmatrix} -1/\sqrt{2} \\ 1/\sqrt{2} \\ 0 \end{pmatrix} = \frac{-1+2}{\sqrt{5}} \begin{pmatrix} -1/\sqrt{2} \\ 1/\sqrt{2} \\ 0 \end{pmatrix} = \begin{pmatrix} -1/2 \\ 1/2 \\ 0 \end{pmatrix} $
\task $\displaystyle \vv{F}_{AB} = \left \{ \begin{pmatrix} -1 \\ 2 \\ 1 \end{pmatrix}\cdot \begin{pmatrix} 1/\sqrt{6} \\ -1/\sqrt{6} \\ 2/\sqrt{6} \end{pmatrix} \right \} \begin{pmatrix} 1/\sqrt{6} \\ -1/\sqrt{6} \\ 2/\sqrt{6} \end{pmatrix} = \frac{-1}{\sqrt{6}}\begin{pmatrix} 1/\sqrt{6} \\ -1/\sqrt{6} \\ 2/\sqrt{6} \end{pmatrix} = \begin{pmatrix} -1/6 \\ 1/6 \\ -2/6 \end{pmatrix} $
\task $\displaystyle \vv{F}_{AB} = \left \{ \begin{pmatrix} qL \\ T \\ T\cos(\alpha) \end{pmatrix}\cdot \begin{pmatrix} 1/\sqrt{2} \\ 0 \\ -1/\sqrt{2} \end{pmatrix} \right \} \begin{pmatrix} 1/\sqrt{2} \\ 0 \\ -1/\sqrt{2} \end{pmatrix} = \frac{qL-T\cos(\alpha)}{2}\begin{pmatrix} 1 \\ 0 \\ -1 \end{pmatrix} $
\end{tasks}
\end{sol}


Le \textbf{produit vectoriel} est une autre opération de multiplication entre deux vecteurs. Elle est très utile en géométrie (mais pas uniquement). Le résultat d'un produit vectoriel est un vecteur, de direction perpendiculaire à chacun des vecteurs multipliés, et dont le sens est donné par la ``règle de la main droite'' : 
%
\begin{center}
\includegraphics[width=4cm]{figs/right_hand.jpg}
\end{center}
%
Dans l'ordre dans lequel on dit ``$\,\vv{u}$ vectoriel $\vv{v}$ égal $\vv{u}\wedge\vv{v}\ $'', on pense ``$1\wedge2 = 3$'', ce qui correspond à l'ordre des doigts de la main \textbf{droite} : pouce, index, majeur. La direction du majeur est alors la direction de $\vv{u}\wedge\vv{v}$. La norme de $\vv{u}\wedge\vv{v}$ est :
%
\begin{equation*}
\Vert \vv{u}\wedge\vv{v} \Vert = \Vert \vv{u} \Vert \times \Vert \vv{v} \Vert \times \vert \sin(\vv{u}, \vv{v}) \vert
\end{equation*}

Il est également possible de faire l'opération de produit vectoriel à partir des composantes des vecteurs :
%
\begin{equation*}
\vv{a} \times \vv{b} = \vv{a} \wedge \vv{b} =
\begin{pmatrix} a_x \\ a_y \\ a_z \end{pmatrix} \wedge
\begin{pmatrix} b_x \\ b_y \\ b_z \end{pmatrix}
=
\begin{pmatrix} a_y b_z - a_z b_y  \\ a_z b_x - a_x b_z \\ a_x b_y - a_y b_x \end{pmatrix}
\end{equation*}

\needspace{5cm}
Nous verrons en TD comment ``retenir'' cette formule. Remarquons que le symbol utilisé en france est $\wedge$, alors que la croix $\times$ est plus communément utilisée dans le reste du monde. 


\begin{exo}

\noindent \ding{45} Calculer les produits vectoriels suivants

\begin{tasks}(2)
\task $\begin{pmatrix} 1 \\ 2 \\ 3 \end{pmatrix} \wedge
\begin{pmatrix} -1 \\ 2 \\ 1 \end{pmatrix} $
\task $\begin{pmatrix} -1 \\ 4 \\ 2 \end{pmatrix} \wedge
\begin{pmatrix} 1 \\ 0 \\ 5 \end{pmatrix} $
\task $\begin{pmatrix} 1 \\ 1 \\ 5 \end{pmatrix} \wedge
\begin{pmatrix} -1 \\ -2 \\ 2 \end{pmatrix} $
\task $\begin{pmatrix} -5 \\ 2 \\ 1 \end{pmatrix} \wedge
\begin{pmatrix} -1 \\ -2 \\ 3 \end{pmatrix} $
\end{tasks}
\end{exo}


\begin{sol}
\begin{tasks}(1)
\task $\begin{pmatrix} 1 \\ 2 \\ 3 \end{pmatrix} \wedge
\begin{pmatrix} -1 \\ 2 \\ 1 \end{pmatrix} =  \begin{pmatrix} 2-6 \\ -3-1 \\ 2-(-2) \end{pmatrix} =  \begin{pmatrix} -4 \\ -4 \\ 4 \end{pmatrix} $
\task $\begin{pmatrix} -1 \\ 4 \\ 2 \end{pmatrix} \wedge
\begin{pmatrix} 1 \\ 0 \\ 5 \end{pmatrix} =  \begin{pmatrix} 20-0 \\ 2-(-5) \\ 0-4 \end{pmatrix}=  \begin{pmatrix} 20 \\ 7 \\ -4 \end{pmatrix}$
\task*(2) $\begin{pmatrix} 1 \\ 1 \\ 5 \end{pmatrix} \wedge
\begin{pmatrix} -1 \\ -2 \\ 2 \end{pmatrix} =  \begin{pmatrix} 2-(-10) \\ -5-2 \\ -2-(-1) \end{pmatrix}=  \begin{pmatrix} 12 \\ -7 \\ -1 \end{pmatrix}$
\task*(2) $\begin{pmatrix} -5 \\ 2 \\ 1 \end{pmatrix} \wedge
\begin{pmatrix} -1 \\ -2 \\ 3 \end{pmatrix} =  \begin{pmatrix} 6 -(-2)\\ -1-(-15) \\ 10-(-2) \end{pmatrix}=  \begin{pmatrix} 8 \\ 14 \\ 12 \end{pmatrix}$
\end{tasks}
\end{sol}


La surface (l'aire) d'un triangle peut être déterminée comme la moitié de la norme du produit vectoriel des vecteurs formés par deux de ses cotés. 
La surface $S$ du triangle formé par les vecteurs $\vv{C}_1$ et $\vv{C}_2$ est :
%
\begin{equation*}
S = \frac{1}{2} \Vert \vv{C}_1 \wedge \vv{C}_2 \Vert
\end{equation*}

En particulier, l'aire d'un triangle $ABC$ peut se calculer avec l'une des formules suivantes (et bien d'autres).

\begin{equation*}
S = \frac{1}{2} \Vert \vv{AB} \wedge \vv{AC} \Vert = \frac{1}{2} \Vert \vv{BA} \wedge \vv{BC} \Vert
= \frac{1}{2} \Vert \vv{CA} \wedge \vv{CB} \Vert
\end{equation*}


\begin{comment}
Schéma d'interprétation (à faire avec l'enseignant)
% 
\ifthenelse{\equal{\showDrawings}{1}}{
\begin{center}
\includegraphics[height=4cm]{figs/SurfTriangle.png}
\end{center}
}{\vspace*{5cm}}
\end{tcolorbox}

\needspace{8cm}
\end{comment}


\begin{exo}

\noindent \ding{45} Déterminer l'expression \textbf{littérale} de la surface $S$ du triangle $ABC$ sachant que...

\begin{tasks}(1)
\task $\vv{AB} = \vv{U} = \begin{pmatrix} U_x \\ U_y \end{pmatrix}$ et $\vv{BC} = \vv{V} = \begin{pmatrix} V_x \\ V_y \end{pmatrix}$
\task $\vv{OA} = \begin{pmatrix} a_x \\ a_y \end{pmatrix}$ ; $\vv{OB} = \begin{pmatrix} b_x \\ b_y \end{pmatrix}$ ; $\vv{OC} = \begin{pmatrix} c_x \\ c_y \end{pmatrix}$
\task $\vv{AB} = \begin{pmatrix} L \\ 0 \\ 1 \end{pmatrix}$ et $\vv{AC} = \begin{pmatrix} D \cos(\theta) \\ D \sin(\theta) \\ 0 \end{pmatrix}$
\end{tasks}
\end{exo}

\begin{sol}
\begin{tasks}(1)
\task $S = \frac{1}{2} \left\Vert\begin{pmatrix} U_x \\ U_y \end{pmatrix} \wedge \begin{pmatrix} V_x \\ V_y \end{pmatrix} \right\Vert = \frac{1}{2} \left\vert U_x V_y - U_y V_x \right\vert $
\task $S  =  \frac{1}{2} \left\Vert\begin{pmatrix} b_x-a_x \\ b_y-a_y \end{pmatrix} \wedge \begin{pmatrix} c_x-a_x \\ c_y-a_y \end{pmatrix} \right\Vert = \frac{1}{2} \left\vert (b_x-a_x) (c_y-a_y) - (b_y-a_y) (c_x-a_x)\right\vert$
\task $S = \frac{1}{2} \left\Vert \begin{pmatrix} L \\ 0 \\ 1 \end{pmatrix} \wedge \begin{pmatrix} D \cos(\theta) \\ D \sin(\theta) \\ 0 \end{pmatrix} \right\Vert = \left\Vert \begin{pmatrix} -D\sin(\theta) \\ D\cos(\theta) \\ LD\sin(\theta) \end{pmatrix}\right\Vert = \sqrt{ D^2 \big(\cos^2(\theta) + (1 + L^2) \sin^2 (\theta) \big)}$
\end{tasks}
\end{sol}

Rappelons, si cela est utile, que 
%
\begin{equation*}
\vv{a} \cdot \vv{b} = 0 \Rightarrow \text{les vecteur sont orthogonaux}
\end{equation*}
%
\begin{equation*}
\vv{a} \wedge \vv{b} = \vv{0} \Rightarrow \text{les vecteur sont colinéaires}
\end{equation*}

Si les vecteurs $\vv{a}$ et $\vv{b}$ portent des droites, l'application pratique sera de déterminer si ces droites sont perpendiculaires ou parallèles. 



%%%%%%%%%%%%%%%%%%%%%%%%%%%%%%%%%%%%%%%%%%%%%%%%%%%%%%%%%%%%%%%%%%%%%%%%%%%%%%%%
\section{Equation cartésienne d'un plan}


Soit un plan ($P$) de normale $\vv{n} = (n_x,\ n_y,\ n_z)^T$ et passant par le point $A$, de coordonnée $\vv{OA} = (A_x,\ A_y,\ A_z)^T$. Pour qu'un point $M$, de coordonnées cartésiennes $\vv{OM} = (x,\ y,\ z)^T$ appartienne au plan ($P$), il est nécessaire que :
%
\begin{equation*}
\vv{AM} \cdot \vv{n} = 0
\Rightarrow
(\vv{OM}-\vv{OA}) \cdot \vv{n} = 0
\Rightarrow
\begin{pmatrix} x-A_x \\ y-A_y \\ z-A_z \end{pmatrix}
\cdot
\begin{pmatrix} n_x \\ n_y \\ n_z \end{pmatrix} = 0
\end{equation*}
%
Ceci se développe en
%
\begin{equation*}
(x-A_x)n_x + (y-A_y)n_y + (z-A_z)n_z = 0
\end{equation*}
%
\begin{equation*}
\Rightarrow n_x x  + n_y y  + n_z z - (A_x n_x + A_y n_y + A_z n_z) = 0
\end{equation*}
%
On voit donc que l'équation cartésienne d'un plan passant par $A$ et de normale $\vv{n}$ (pas nécessairement de norme unitaire) est :
\begin{equation*}
a x  + b y  + c z + d = 0
\end{equation*}
%
avec $a=n_x$, $b=n_y$, $c=n_z$ et $d=-\vv{OA}\cdot \vv{n}$.



\begin{comment}
Schéma d'interprétation (à faire avec l'enseignant)
%
\ifthenelse{\equal{\showDrawings}{1}}{
\begin{center}
\includegraphics[height=6cm]{figs/eqPlan.png}
\end{center}
}{\vspace*{6cm}}
\end{comment}

\needspace{8cm}
\begin{exo}

\noindent \ding{45} Déterminer l'équation cartésienne du plan...

\begin{tasks}(1)
\task passant par le point $A= \begin{pmatrix} -2 \\ 0 \\ 5\end{pmatrix}$ et de normale $\vv{n} = \begin{pmatrix} 2 \\ -6 \\ 4 \end{pmatrix}$ 
\task passant par les points $A= \begin{pmatrix} 1 \\ -2 \\ 7\end{pmatrix}$, $B= \begin{pmatrix} 2 \\ 2 \\ 1\end{pmatrix}$ et $C= \begin{pmatrix} 1 \\ 1 \\ 5\end{pmatrix}$
\task tangent à une sphère de rayon $3\sqrt{3}$ centrée en $A= \begin{pmatrix} 1 \\ 2 \\ 3\end{pmatrix}$, et de normale $\vv{n} = \begin{pmatrix} 1 \\ 1 \\ -1 \end{pmatrix}$
\task tangent à une sphère de rayon $R$ centrée en $A= \begin{pmatrix} A_x \\ A_y \\ A_z\end{pmatrix}$, et de normale unitaire $\vv{n} = \begin{pmatrix} n_x \\ n_y \\ n_z \end{pmatrix}$
\end{tasks}
\end{exo}


\begin{sol}
\begin{tasks}(1)
\task $d = -\vv{OA}\cdot\vv{n} = -(-4 + 20) = -16$, équation du plan : $2x-6y+4z-16 = 0$, qu'on peut simplifier en $\boxed{x-3y+2z-8 = 0}$
\task $\vv{n} = \vv{AB} \wedge \vv{AC} = \begin{pmatrix} 1 \\ 4 \\ -6\end{pmatrix} \wedge \begin{pmatrix} 0 \\ 3 \\ -2\end{pmatrix} = \begin{pmatrix} 10 \\ 2 \\ 3\end{pmatrix}$, $d = -\vv{OA}\cdot\vv{n} = -(10  -4 + 21)  = -27 $, équation du plan : $\boxed{10x+2y+3z-27 = 0}$
\task Point du plan : $\vv{OA} + 3\sqrt{3}\frac{\vv{n}}{\sqrt{3}} = \begin{pmatrix} 1+3 \\ 2+3 \\ 3-3\end{pmatrix} = \begin{pmatrix} 4 \\ 5 \\ 0\end{pmatrix}$, $d=- \begin{pmatrix} 4 \\ 5 \\ 0\end{pmatrix}\cdot\begin{pmatrix} 1 \\ 1 \\ -1\end{pmatrix}= -(4+5)= -9$, équation du plan : $\boxed{x+y-z-9=0}$
\task Point du plan : $\vv{OA} + R\vv{n}$, $d=-(\vv{OA} + R\vv{n})\cdot\vv{n} = -(\vv{OA}\cdot\vv{n} + R\vv{n}\cdot\vv{n})$, équation du plan : $\boxed{n_x x+n_y y-n_z z -(\vv{OA}\cdot\vv{n} + R\vv{n}^2) =0}$
\end{tasks}
\end{sol}



%%%%%%%%%%%%%%%%%%%%%%%%%%%%%%%%%%%%%%%%%%%%%%%%%%%%%%%%%%%%%%%%%%%%%%%%%%%%%%%%
\needspace{8cm}
\section{Equations paramétrique et cartésienne d'une droite}



Soit une droite ($D$) de vecteur directeur $\vv{d} = (d_x,\ d_y,\ d_z)^T$ et passant par le point $A$, de coordonnée $\vv{OA} = (A_x,\ A_y,\ A_z)^T$. Pour qu'un point $M$, de coordonnées cartésiennes $\vv{OM} = (x,\ y,\ z)^T$ appartienne à la droite ($D$), il est nécessaire que :
%
\begin{equation*}
\vv{AM} = k \vv{d} \quad (\text{avec } k \in \mathbb{R})
\Rightarrow
\begin{pmatrix} x-A_x \\ y-A_y \\ z-A_z \end{pmatrix}
= k
\begin{pmatrix}  d_x \\  d_y \\  d_z \end{pmatrix} 
\Rightarrow
\begin{pmatrix} x\\ y \\ z \end{pmatrix}
=
\begin{pmatrix} A_x + k d_x \\ A_y + k d_y \\ A_z + k d_z \end{pmatrix}
\end{equation*}
%
Notons que le vecteur directeur $\vv{d}$ n'est pas obligatoirement unitaire. 
Les \textbf{équations paramétriques} de cette droite consistent simplement à réécrire ceci sous forme d'un système de trois équations utilisant le même paramètre $k$ :
%
\begin{equation*}
\begin{cases}
x & = A_x + k \times d_x \\
y & = A_y + k \times d_y \\
z & = A_z + k \times d_z
\end{cases}
\quad (k \in \mathbb{R})
\end{equation*}


\begin{comment}
Schéma d'interprétation (à faire avec l'enseignant)
\ifthenelse{\equal{\showDrawings}{1}}{
\begin{center}
\includegraphics[height=4cm]{figs/eqParamDroite.png}
\end{center}
}%
{\vspace*{5cm}}% if not set to 1
\end{comment}


\begin{exo}


\noindent \ding{45} Déterminer les équations paramétriques de la droite...

\begin{tasks}(1)
\task passant par le point $A= \begin{pmatrix} 2 \\ 1 \\ -3\end{pmatrix}$ et de vecteur directeur normal $\vv{u} = \begin{pmatrix} 1 \\ 1 \\ 1 \end{pmatrix}$
\task passant par les points $A= \begin{pmatrix} 2 \\ 0 \\ 1\end{pmatrix}$ et $B= \begin{pmatrix} 0 \\ 3 \\ -1\end{pmatrix}$
\end{tasks}
\end{exo}


\begin{sol}
\begin{tasks}(2)
\task $\begin{cases} x &= 2+k \\ y &= 1+k  \\ z &= -3+k \end{cases} \quad (k \in \mathbb{R})$ 
\task $\begin{cases} x &= 2-2k \\ y &= 0+3k  \\ z &= 1-2k \end{cases} \quad (k \in \mathbb{R})$ 
\end{tasks}
\end{sol}



\needspace{5cm}
En remarquant qu'il existe dans les équations paramétriques d'une droite, une valeur unique du paramètre $k$ pour un point donné sur cette droite, on peut calculer $k$ de trois façons :
%
\begin{equation*}
k = \frac{x - A_x}{d_x} = \frac{y - A_y}{d_y} = \frac{z - A_z}{d_z} 
\end{equation*}
%
Ceci constitue une façon alternative de définir une droite par ses \textbf{équations cartésiennes} :
%
\begin{equation*}
\begin{cases}
\displaystyle \frac{x-A_x}{d_x} & = \displaystyle \frac{y-A_y}{d_y}  \\
\displaystyle \frac{y-A_y}{d_y} & = \displaystyle \frac{z-A_z}{d_z}
\end{cases}
\end{equation*}
%
ou encore :
%
\begin{equation*}
\begin{cases}
d_y \times x - d_x \times y + (d_x \times A_y - d_y \times A_x)  & = 0  \\
d_z \times y - d_y \times z + (d_y \times A_z - d_z \times A_y) & = 0
\end{cases}
\end{equation*}
%
Enfin, il est intéressant d'interpréter l'équation cartésienne d'une droite comme l'intersection de deux plans :
\begin{equation*}
\begin{cases}
n_x \times x + n_y \times y + \text{\colorbox{lightgray}{$0 \times z$}} + d & = 0 \\
\text{\colorbox{lightgray}{$0 \times x$}} + n_y' \times y + n_z' \times z + d' & = 0
\end{cases}
\end{equation*}

%\needspace{10cm}
\begin{comment}
Schéma d'interprétation (à faire avec l'enseignant)
%
\ifthenelse{\equal{\showDrawings}{1}}{
\begin{center}
\includegraphics[height=6cm]{figs/eqCartesienneDroite3D.png}
\end{center}
}{\vspace*{6cm}}
\end{comment}


\begin{exo}

\noindent \ding{45} Déterminer les équations cartésiennes pour chacun des cas de l'exercice précédent rappelés ci-après :

\begin{tasks}(1)
\task passant par le point $A= \begin{pmatrix} 2 \\ 1 \\ -3\end{pmatrix}$ et de vecteur directeur normal $\vv{u} = \begin{pmatrix} 1 \\ 1 \\ 1 \end{pmatrix}$
\task passant par les points $A= \begin{pmatrix} 2 \\ 0 \\ 1\end{pmatrix}$ et $B= \begin{pmatrix} 0 \\ 3 \\ -1\end{pmatrix}$
\end{tasks}

\end{exo}

\begin{sol}
\begin{tasks}(1)
\task $\begin{cases} x &= 2+k \\ y &= 1+k  \\ z &= -3+k \end{cases} \Rightarrow x-2 = y-1 = z+3$.\\ 
La première égalité s'écrit $x-y-1=0$ (en transférant tous les termes vers la gauche) et la seconde égalité se ré-écrit $y-z-4=0$. \\
D'où les équations cartésiennes de la droite : $\boxed{\begin{cases}x-y-1&=0\\y-z-4&=0 \end{cases}}$   
\task $\displaystyle \begin{cases} x &= 2-2k \\ y &= 0+3k  \\ z &= 1-2k \end{cases} \Rightarrow \frac{x-2}{-2} = \frac{y}{3} = \frac{z-1}{-2}$.\\ 
D'où les équations cartésiennes de la droite : 
$\begin{cases}\displaystyle -\frac{x}{2}-\frac{y}{3}+1&=0\\[2mm] \displaystyle \frac{y}{3}+\frac{z}{2}-\frac{1}{2}&=0 \end{cases}$ \\
Il est interessant de simplifier l'écriture de ces équations en multipliant par une valeur bien choisie :\\[2mm]
$\begin{cases}\displaystyle -6\times \left( -\frac{x}{2}-\frac{y}{3}+1\right) &= -6\times0  \\[2mm] \displaystyle 6\times \left(\frac{y}{3}+\frac{z}{2}-\frac{1}{2}\right)&=6\times0 \end{cases} \Rightarrow \boxed{\begin{cases} 3x +2y -6 &=0\\ 2y + 3z -3 &=0 \end{cases}}$ 
\end{tasks}
\end{sol}




%%%%%%%%%%%%%%%%%%%%%%%%%%%%%%%%%%%%%%%%%%%%%%%%%%%%%%%%%%%%%%%%%%%%%%%%%%%%%%%%
\section{Intersections :\\ plan~$\cap$~plan (droite) et droite~$\cap$~plan (point)}


L'intersection de deux plans ($P$) et ($P'$) en 3D est une droite ($D$) si les plans se coupent et ne sont donc pas parallèles (c'est-à-dire que leurs normales ne sont pas colinéaires). Les coordonnées $x$, $y$ et $z$ des points de la droite d'intersection ($D$) appartiennent à la fois au plan ($P$) et au plan ($P'$). Elles satisfont donc aux deux équations cartésiennes de ces plans :
%
$$
\begin{cases}
a x + b y + c z + d  & = 0  \\
a' x + b' y + c' z + d'  & = 0 
\end{cases}
$$
%
Il s'agit ici d'un système de deux équations avec trois inconnues.
La méthode pour trouver les équations paramétriques de la droite ($D$) consiste à poser arbitrairement $x=k$ (tout autre choix comme $z=k$ serait aussi valable). Ceci ajoute une équation au système précédent, et en remplaçant $x$ par $k$ le premier système à deux équations n'aura plus que deux inconnues (ici $y$ et $z$) :   
%
$$
\begin{cases}
x & = k \\
\text{et} &
\begin{cases}
a x + b y + c z + d  & = 0  \\
a' x + b' y + c' z + d'  & = 0 
\end{cases}
\end{cases}
\quad\Rightarrow\quad
\begin{cases}
x & = k \\
\text{et} &
\begin{cases}
b y + c z   & = -d - a k  \\
b' y + c' z   & = -d' - a' k 
\end{cases}
\end{cases}
$$
%
En résolvant ce système d'équations avec la méthode de son choix, il devient possible de trouver $y$ et $z$, et de se ramener à un système d'équations de la forme suivante : 
%
$$
\begin{cases}
x & = 0 + 1 \times k \\
y & = A_y + d_y \times k  \\
z & = A_z + d_z \times k 
\end{cases}
\quad (k \in \mathbb{R})
$$
%
Il s'agit là des équations paramétriques de la droite d'intersection ($D$).


\begin{comment}
Schéma d'interprétation (à faire avec l'enseignant)
%
\ifthenelse{\equal{\showDrawings}{1}}{
\begin{center}
\includegraphics[height=6cm]{figs/intersectionPlans.png}
\end{center}
}{\vspace*{6cm}} 
\end{comment}


\begin{exo}

\noindent \ding{45} Déterminer les équations paramétriques de la droite, intersection des plans $P_1$ et $P_2$.

\begin{tasks}(1)
\task $P_1 : x+y+z=0\quad$ et $\quad P_2 : 2x+3y+z-4 = 0$
\task $P_1 : x-3y+2z=5\quad$ et $\quad P_2 : 2x+y+7z = 1$
\task $P_1 : x+y+z+1=0\quad$ et $\quad P_2 : 2x-y-z+3 = 1$
\end{tasks}

\end{exo}

\begin{sol}
\begin{tasks}(1)
\task $P_1 : x+y+z=0\quad$ et $\quad P_2 : 2x+3y+z-4 = 0$
$$
\begin{cases}
z & = k \\
\text{et} &
\begin{cases}
x+y+z  & = 0  \\
2x+3y+z-4  & = 0 
\end{cases}
\end{cases}
\Rightarrow
\begin{cases}
z & = k \\
\text{et} &
\begin{cases}
x+y   &= -k  \\
2x+3y   &= 4-k 
\end{cases}
\end{cases}
$$
$$
\Rightarrow
\begin{cases}
z & = k \\
\text{et} &
\begin{cases}
x   & = -k - y  \\
-2(k+y)+3y  & = 4-k 
\end{cases}
\end{cases}
\Rightarrow
\boxed{
\begin{cases}
x   &= -4 -2k  \\
y   &= 4 + k \\
z & = k 
\end{cases}
}
$$

\task $P_1 : x-3y+2z=5\quad$ et $\quad P_2 : 2x+y+7z = 1$
$$
\begin{cases}
x & = k \\
\text{et} &
\begin{cases}
k-3y+2z & = 5  \\
2k+y+7z  & = 1 
\end{cases}
\end{cases}
\Rightarrow
\begin{cases}
x & = k \\
\text{et} &
\begin{cases}
k-3y+2z   &= 5 - k  \\
y+7z   &= 1 - 2k 
\end{cases}
\end{cases}
$$
$$
\Rightarrow
\begin{cases}
x & = k \\
\text{et} &
\begin{cases}
y   & = -\frac{5}{3} + \frac{2}{3}(k+z) \\
z   & =  -\frac{7}{23}k  + \frac{8}{23}
\end{cases}
\end{cases}
\Rightarrow
\boxed{
\begin{cases}
x   &= k  \\
y   &= -\frac{33}{23} + \frac{3}{23}k  \\
z   &=  \frac{8}{23}  -\frac{7}{23}k  
\end{cases}
}
$$

\task $P_1 : x+y+z+1=0\quad$ et $\quad P_2 : 2x-y-z+3 = 1$
$$
\begin{cases}
y & = k \\
\text{et} &
\begin{cases}
x+k+z+1  & = 0  \\
2x-k-z+2  & = 0 
\end{cases}
\end{cases}
\Rightarrow
\begin{cases}
y & = k \\
\text{et} &
\begin{cases}
x   &= -k-1-z  \\
z   &= -k 
\end{cases}
\end{cases}
$$
$$
\Rightarrow
\begin{cases}
y & = k \\
\text{et} &
\begin{cases}
x  & = -1  \\
z  & = -k 
\end{cases}
\end{cases}
\Rightarrow
\boxed{
\begin{cases}
x   &= -1  \\
y   &= k \\
z & = -k 
\end{cases}
}
$$

\end{tasks}
\end{sol}



L'intersection d'un plan ($P$) et d'une droite ($D$) en 3D est un point si le plan et la droite se coupent (c'est-à-dire que la normale au plan et le vecteur directeur de la droite ne sont pas orthogonaux). Les coordonnées $x$, $y$ et $z$ du point d'intersection appartiennent à la fois au plan ($P$) et à la droite ($D$). Elles satisfont donc aux équations suivantes, respectivement pour la droite et le plan :
%
$$
\begin{cases}
x & = A_x + d_x \times k \\
y & = A_y + d_y \times k  \\
z & = A_z + d_z \times k 
\end{cases}
\quad (k \in \mathbb{R})
\qquad \text{et} \qquad
a x + b y + c z + d  = 0
$$
%
En insérant les équations paramétriques de la droite ($D$) dans l'équation cartésienne du plan ($P$), on obtient :
%
$$
a (A_x + d_x \times k) + b (A_y + d_y \times k) + c (A_z + d_z \times k) + d  = 0
$$
%
où la seule inconnue est le paramètre $k$ qui peut alors être déterminé :
%
$$
(a d_x + b d_y + c d_z )k + (a A_x + b A_y + c A_z + d) = 0 
$$
%
$$
\Rightarrow k = \frac{-(a A_x + b A_y + c A_z + d)}{(a d_x + b d_y + c d_z )} 
$$

% un mot pour dire qu'il ne faut pas obligatoirement prendre cette relation comme une formule, mais plutôt comprendre comment y arriver. Dans la pratique, ce sera plus simple

Une fois que la valeur de $k$ est trouvée, il suffit de l'utiliser dans les équations paramétriques de la droite pour finaliser le calcul des composante $x$, $y$ et $z$ du point d'intersection.


\begin{comment}
Schéma d'interprétation (à faire avec l'enseignant)
%
\ifthenelse{\equal{\showDrawings}{1}}{
\begin{center}
\includegraphics[height=6cm]{figs/intersectionDroitePlan.png}
\end{center}
}{\vspace*{6cm}} 
\end{comment}


\begin{exo}

\noindent \ding{45} Déterminer le point d'intersection de la droite $\Delta$ et du plan $P$

\begin{tasks}(1)
\task Droite $\Delta$ passant par $(1\ 2\ 3)^T$ de vecteur directeur $(0\ 1\ -1)^T$ et plan $P$ d'équation $z -2 = 0$
\task Droite $\Delta$ passant par $(0\ 0\ 0)^T$ de vecteur directeur $(1\ 1\ 2)^T$ et plan $P$ d'équation $3x+y-z -6= 0$
\end{tasks}
\end{exo}

\begin{sol}
\begin{tasks}(1)
\task Droite $\Delta$ passant par $(1\ 2\ 3)^T$ de vecteur directeur $(0\ 1\ -1)^T$ et plan $P$ d'équation $z -2 = 0$
$$
\begin{cases}
x & = 1 \\
y & = 2 +  k  \\
z & = 3 - k 
\end{cases}
\quad (k \in \mathbb{R})
\qquad \text{et} \qquad
z - 2  = 0
$$
$$
\Rightarrow (3-k) -2 = 0 \Rightarrow \boxed{k = 1} \Rightarrow \Delta \cap P = \begin{pmatrix} 1 \\ 3 \\ 2 \end{pmatrix}
$$

\task Droite $\Delta$ passant par $(0\ 0\ 0)^T$ de vecteur directeur $(1\ 1\ 2)^T$ et plan $P$ d'équation $3x+y-z -6 = 0$
$$
\begin{cases}
x & = k \\
y & = k  \\
z & = 2k 
\end{cases}
\quad (k \in \mathbb{R})
\qquad \text{et} \qquad
3x+y-z - 6=0
$$
$$
\Rightarrow 3k+k-2k = 6 \Rightarrow 2k = 6 \Rightarrow \boxed{k = 3} \Rightarrow \Delta \cap P = \begin{pmatrix} 3 \\ 3 \\ 6 \end{pmatrix}
$$

\end{tasks}
\end{sol}



 % dans MAT4 à Grenoble

% MAT2
%!TEX root = main.tex

\chapter{Etude de fonctions}
\label{chapter:fonctions}


 % à faire
%!TEX root = main.tex

\chapter{Integration}
\label{chapter:integration}


 % à faire

% MAT3
%!TEX root = main.tex

\chapter{Calcul matriciel}
\label{chapter:matrices}


 % sur la base du cours de Manu
%!TEX root = main.tex

\chapter{Equations Différentielles Ordinaires}
\label{chapter:EDO}



% A remplir à partir du cours de Manu


 % sur la base du cours de Manu

% MAT4
%!TEX root = main.tex

\chapter{Fonctions à plusieurs variables}
\label{chapter:fonction2var}


Jusqu'à présent, nous avons considéré les fonctions comme des \textbf{fonctions à une seule variable}.
Ceci signifie qu'à une valeur $x$ pris dans $\mathbb{R}$ on associe une valeur $y=f(x)$, également dans $\mathbb{R}$. 
Nous nous intéressons maintenant à des fonctions à deux variables qui associent un couple de valeurs $(x,y)$ pris dans $\mathbb{R}\times\mathbb{R} = \mathbb{R}^2$ à une valeur $f(x,\,y)$ dans $\mathbb{R}$. On notera :
%
$$
\begin{aligned}
f:\mathcal{D}_f \subset \mathbb{R}^2 & \longmapsto \mathbb{R} \\
 (x,y) & \longmapsto z = f(x,\,y)
\end{aligned}
$$


\begin{comment}
Schéma d'interprétation (à faire avec l'enseignant)
%
\ifthenelse{\equal{\showDrawings}{1}}{
\begin{center}
\includegraphics[height=6cm]{figs/surf3D.png}
\end{center}
}{\vspace*{6cm}}
\end{comment}


Le domaine de définition $\mathcal{D}_f$ de la fonction $f(x,\,y)$ est l'ensemble des couples de valeurs $(x,y)$ où la valeur $z=f(x,\,y)$ est définie (c'est-à-dire, ``existe''). On représente ce domaine sur le plan $(x,y)$



\begin{exo}

\noindent \ding{45} Pour chacune des fonctions $f(x,\,y)$ suivantes, représenter son domaine de définition dans le plan $Oxy$.

\begin{tasks}(2)
\task $f(x,\,y) = \displaystyle \frac{x^3y + xy^2}{x+y}$
\task $f(x,\,y) = \sqrt{xy}$
\task $f(x,\,y) = \displaystyle \ln \left(1+\frac{x}{y}  \right)$
\task $f(x,\,y) = \displaystyle \frac{\ln x}{x^2+y^2-9}$
\end{tasks}
\end{exo}


\begin{sol}

\begin{tasks}(1)

\task $f(x,\,y) = \displaystyle \frac{x^3y + xy^2}{x+y}$ n'est pas définie si $x+y=0$, donc si $y=-x$.
%
\begin{center}
\begin{tikzpicture}
\begin{axis}[xmin=-2,xmax=2,ymin=-2,ymax=2,axis background/.style={fill=green!20},xlabel=$x$,ylabel=$y$]
\draw (-2,0) -- (2,0);
\draw (0,-2) -- (0,2);
\draw[red,line width=2pt] (-2.0,2.0) -- (2.0,-2.0);
\end{axis}
\end{tikzpicture}
\end{center}


\task $f(x,\,y) = \sqrt{xy}$ est définie ssi $xy \geq 0$, donc si $x$ et $y$ sont de signes opposés.
%
\begin{center}
\begin{tikzpicture}
\begin{axis}[xmin=-2,xmax=2,ymin=-2,ymax=2,axis background/.style={fill=green!20},xlabel=$x$,ylabel=$y$]
\filldraw [fill=red!20, draw=none] (-2,2) rectangle (0,0);
\filldraw [fill=red!20, draw=none] (2,-2) rectangle (0,0);
\draw[green,line width=2pt] (-2,0) -- (2,0);
\draw[green,line width=2pt] (0,-2) -- (0,2);
\end{axis}
\end{tikzpicture}
\end{center}


% c
\task $f(x,\,y) = \displaystyle \ln \left(1+\frac{x}{y} \right)$ n'est définie que si $\frac{x}{y} > -1$ avec $y\ne0$.
%
\begin{center}
\begin{tikzpicture}
\begin{axis}[xmin=-2,xmax=2,ymin=-2,ymax=2,axis background/.style={fill=green!20},xlabel=$x$,ylabel=$y$]
\addplot[name path=B,domain=-2:2] {-x};
\addplot[name path=A,domain=-2:2] {0};
\tikzfillbetween[of=A and B]{red!20};
\addplot[domain=-2:2,red,line width=2pt] {-x};
\addplot[domain=-2:2,red,line width=2pt] {0};
\end{axis}
\end{tikzpicture}
\end{center}

% d
\task $f(x,\,y) = \displaystyle \frac{\ln x}{x^2+y^2-9}$ est définie lorsque $x>0$ et $x^2+y^2-9 \ne 0$.
%
\begin{center}
\begin{tikzpicture}
\begin{axis}[axis equal,xmin=-2,xmax=4,ymin=-4,ymax=4,axis background/.style={fill=green!20},xlabel=$x$,ylabel=$y$]
\filldraw [fill=red!20, draw=none] (-5,-4) rectangle (0,4);
\draw[red,line width=2pt] (0,-4) -- (0,4);
\draw[red,line width=2pt] (0,-3) arc [start angle=-90, end angle=90, radius=3];
\end{axis}
\end{tikzpicture}
\end{center}

\end{tasks}

\end{sol}

%%%%%%%%%%%%%%%%%%%%%%%%%%%%%%%%%%%%%%%%%%%%%%%%%%%%%%%%%%%%%%%%%%%%%%%%%%%%%%%%
\needspace{5cm}
\section{Calcul de dérivées partielles}

La dérivée partielle d'une fonction de plusieurs variables est sa dérivée par rapport à l'une de ses variables, les autres étant gardées constantes. Par exemple, prenons la fonction :
$$f(x,\,y) = x^2 \cos(y)$$
La dérivée par rapport à $x$ est :
$$
\displaystyle \frac{\partial}{\partial x} f(x,\,y) = 2x \cos(y)
$$
et la dérivée par rapport à $y$ est:
$$
\displaystyle \frac{\partial}{\partial y} f(x,\,y) = -x^2 \sin(y)
$$

On a donc considéré $\cos(y)$ comme une constante lors de la dérivation par rapport à $x$, et c'est $x^2$ qui a été considérée comme une constante lors de la dérivation par rapport à $y$.\\

\noindent \textbf{Remarque 1} : le symbol $\partial$ se dit ``D rond''. Lorsque la dérivée n'est pas partielle (pour une fonction à une variable), on utilisera le ``D droit'' : exemple $\frac{\text{d}}{\text{d}x} f(x)$\\ 
\noindent \textbf{Remarque 2} : la fraction $\frac{\partial}{\partial x}$ ne représente pas une division, il s'agit juste d'une convention d'écriture pour signifier que la dérivation doit être faite par rapport à la variable $x$, l'autre variable étant ``vu'' comme une constante. \\
\noindent \textbf{Remarque 3} : il est équivalent de noter $\frac{\partial f(x,\,y)}{\partial x}$ au lieu de $\frac{\partial}{\partial x}f(x,\,y)$.

\needspace{13cm}
\begin{exo}

\noindent \ding{45} Calculer les dérivées partielles $\dfrac{\partial f(x,\,y)}{\partial x}$ et $\dfrac{\partial f(x,\,y)}{\partial y}$ pour les fonctions suivantes.

\begin{tasks}(2)
\task $f(x,\,y)=x+y+3xy^2$
\task $f(x,\,y)=e^{-2x}\cos y$
\task $f(x,\,y)=(x^2+2y^3)\cos(xy)$
\task $f(x,\,y)=\sqrt{1+x^2 y^2}$
\end{tasks}
\end{exo}

\begin{sol}
\begin{tasks}(1)
\task $f(x,\,y)=x+y+3xy^2$
$$
\begin{cases}
\displaystyle \frac{\partial}{\partial x} f(x,\,y) &= 1+3y^2 \\[2mm]
\displaystyle \frac{\partial}{\partial y} f(x,\,y) &= 1+6xy 
\end{cases}
$$

\task $f(x,\,y)=e^{-2x}\cos y$
$$
\begin{cases}
\displaystyle \frac{\partial}{\partial x} f(x,\,y) &= -2\cos (y) e^{-2x}  \\[2mm]
\displaystyle \frac{\partial}{\partial y} f(x,\,y) &= -e^{-2x}\sin (y)
\end{cases}
$$

\task $f(x,\,y)=(x^2+2y^3)\cos(xy)$
$$
\begin{cases}
\displaystyle \frac{\partial}{\partial x} f(x,\,y) &= 2x \cos(xy) -y(x^2+2y^3)\sin(xy) \\[2mm]
\displaystyle \frac{\partial}{\partial y} f(x,\,y) &= 6y^2 \cos(xy)  -x(x^2+2y^3)\sin(xy)
\end{cases}
$$

\task $f(x,\,y)=\sqrt{1+x^2 y^2}$
$$
\begin{cases}
\displaystyle \frac{\partial}{\partial x} f(x,\,y) &= \displaystyle \frac{1}{2}(1+x^2 y^2)^{-\frac{1}{2}} (2y^2x) = \frac{xy^2}{\sqrt{1+x^2 y^2}} \\[2mm]
\displaystyle \frac{\partial}{\partial y} f(x,\,y) &= \displaystyle \frac{1}{2}(1+x^2 y^2)^{-\frac{1}{2}} (2x^2y) = \frac{x^2y}{\sqrt{1+x^2 y^2}}
\end{cases}
$$

\end{tasks}
\end{sol}



Pour les dérivées partielles secondes, la notation est la suivante : 
$$
\displaystyle \frac{\partial}{\partial x} \left( \frac{\partial}{\partial y} f(x,\,y) \right) \equiv \frac{\partial^2}{\partial x \partial y} f(x,\,y)
$$
Ce qui signifie qu'on dérive par rapport à $y$ puis à $x$.

Dans le cas où la dérivation se fait 2 fois suivant la même variable, $x$ par exemple, on notera : 
$$
\displaystyle \frac{\partial}{\partial x} \left(\frac{\partial}{\partial x} f(x,\,y) \right) \equiv  \frac{\partial^2}{\partial x^2} f(x,\,y)
$$

Il suffit donc de comprendre les notations des dérivées partielles du second ordre pour les calculer. On admettra, sans démonstrations que l'ordre de dérivation n'a pas d'importance :
$$
\displaystyle \frac{\partial^2}{\partial x \partial y} f(x,\,y) =  \frac{\partial^2}{\partial y \partial x} f(x,\,y)
$$ 



\begin{exo}

\noindent \ding{45} Calculer les dérivées partielles secondes $\dfrac{\partial^2 f(x,\,y)}{\partial x^2}$, $\dfrac{\partial^2 f(x,\,y)}{\partial y^2}$, $\dfrac{\partial^2 f(x,\,y)}{\partial y\partial x}$ et $\dfrac{\partial^2 f(x,\,y)}{\partial x\partial y}$ 
pour les fonctions suivantes. 

\hspace{5cm}Vérifier l'égalité $\dfrac{\partial^2 f(x,\,y)}{\partial y\partial x}=\dfrac{\partial^2 f(x,\,y)}{\partial x\partial y}$


\begin{tasks}(2)
\task $f(x,\,y)=e^{-2xy^2}$
\task $f(x,\,y)=x+y+3xy+x^4 + (xy)^2$
\task $f(x,\,y)=\displaystyle \frac{1}{x} + \frac{1}{y}$
\task $f(x,\,y)=\ln \vert xy \vert$
\end{tasks}
\end{exo}


\begin{sol}
\begin{tasks}(1)
\task $f(x,\,y)=e^{-2xy^2}$
$$
\begin{cases}
\displaystyle \frac{\partial}{\partial x} f(x,\,y) &=  -2y^2e^{-2xy^2} \\[2mm]
\displaystyle \frac{\partial}{\partial y} f(x,\,y) &=  -4xy e^{-2xy^2} \\[2mm]
\displaystyle \frac{\partial^2}{\partial x^2} f(x,\,y) &= 4y^4e^{-2xy^2} \\[2mm]
\displaystyle \frac{\partial^2}{\partial y^2} f(x,\,y) &= 16(xy)^2 e^{-2xy^2} \\[2mm]
\displaystyle \frac{\partial^2}{\partial x \partial y} f(x,\,y) &= 8xy^3 e^{-2xy^2} \\[2mm]
\displaystyle \frac{\partial^2}{\partial y \partial x} f(x,\,y) &= 8xy^3 e^{-2xy^2} \\
\end{cases}
$$

\task $f(x,\,y)=x+y+3xy+x^4 + (xy)^2$
$$
\begin{cases}
\displaystyle \frac{\partial}{\partial x} f(x,\,y) &= 1+3y+4x^3+2y^2x \\[2mm]
\displaystyle \frac{\partial}{\partial y} f(x,\,y) &= 1+3x +2x^2y \\[2mm]
\displaystyle \frac{\partial^2}{\partial x^2} f(x,\,y) &= 12x^2 + 2y^2  \\[2mm]
\displaystyle \frac{\partial^2}{\partial y^2} f(x,\,y) &= 2x^2 \\[2mm]
\displaystyle \frac{\partial^2}{\partial x \partial y} f(x,\,y) &= 3 + 4yx \\[2mm]
\displaystyle \frac{\partial^2}{\partial y \partial x} f(x,\,y) &= 3 + 4yx \\
\end{cases}
$$

\task $\displaystyle f(x,\,y)=\frac{1}{x} + \frac{1}{y}$
$$
\begin{cases}
\displaystyle \frac{\partial}{\partial x} f(x,\,y) &= \displaystyle  -\frac{1}{x^2}\\[2mm]
\displaystyle \frac{\partial}{\partial y} f(x,\,y) &= \displaystyle -\frac{1}{y^2} \\[2mm]
\displaystyle \frac{\partial^2}{\partial x^2} f(x,\,y) &= \displaystyle\frac{2}{x^3} \\[2mm]
\displaystyle \frac{\partial^2}{\partial y^2} f(x,\,y) &= \displaystyle \frac{2}{y^3}\\[2mm]
\displaystyle \frac{\partial^2}{\partial x \partial y} f(x,\,y) &= 0 \\[2mm]
\displaystyle \frac{\partial^2}{\partial y \partial x} f(x,\,y) &= 0 \\
\end{cases}
$$

\task $f(x,\,y)=\ln \vert xy \vert$
$$
\begin{cases}
\displaystyle \frac{\partial}{\partial x} f(x,\,y) &= \displaystyle \frac{1}{xy}y = \frac{1}{x} \\[2mm]
\displaystyle \frac{\partial}{\partial y} f(x,\,y) &= \displaystyle \frac{1}{xy}x = \frac{1}{y} \\[2mm]
\displaystyle \frac{\partial^2}{\partial x^2} f(x,\,y) &= \displaystyle -\frac{1}{x^2} \\[2mm]
\displaystyle \frac{\partial^2}{\partial y^2} f(x,\,y) &= \displaystyle -\frac{1}{y^2} \\[2mm]
\displaystyle \frac{\partial^2}{\partial x \partial y} f(x,\,y) &= 0 \\[2mm]
\displaystyle \frac{\partial^2}{\partial y \partial x} f(x,\,y) &= 0 \\
\end{cases}
$$

\end{tasks}
\end{sol}

%%%%%%%%%%%%%%%%%%%%%%%%%%%%%%%%%%%%%%%%%%%%%%%%%%%%%%%%%%%%%%%%%%%%%%%%%%%%%%%%

\section{Recherche de points critiques et d'extrema}


Avec une fonction à une variable, un extremum (minimum ou maximum) de la fonction $f(x)$ se trouve à la position $x_0$ où $f'(x)=\frac{\text{d}}{\text{d}x}f(x) = 0$. Bien entendu, $x_0$ doit être inclus dans le domaine de définition de $f(x)$.

Avec une fonction à deux variables défini sur $\mathcal{D}_f \subset \mathbb{R}^2$, un extremum de la fonction $f(x,\,y)$ pourra potentiellement se trouver à la position $\bm{M}=(x_0, y_0)$ où :

$$
\begin{cases}
\displaystyle \frac{\partial}{\partial x} f(x,\,y) & =0 \\[4mm]
\displaystyle \frac{\partial}{\partial y} f(x,\,y) & =0 \\
\end{cases}
$$

Ce système d'équations (pas forcement linéaires) peut avoir un certain nombre de solutions, c'est-à-dire qu'il existe aucun, un seul ou plusieurs \textbf{points critiques} $\bm{M}$, mais il est nécessaire que $\bm{M} \subset \mathcal{D}_f$.

\textbf{Remarque} : une façon plus concise de noter le système précédent est d'utiliser l'opérateur $\bm{\nabla}$ (prononcer ``nabla'') devant la fonction $f(x,\,y)$ pour définir son \textbf{gradient} $\bm{\nabla} f(x,\,y)$. On recherchera alors les points critiques aux positions où $\bm{\nabla} f(x,\,y) = \vec{0}$. Nous verrons ceci plus en détail en faisant les exercices.



Pour déterminer si un point critique est un extremum, il est nécessaire de calculer en ce point $(x_0,\,y_0)$ le \textbf{discriminant hessien} $\vert \bm{H}_f \vert$ défini comme suit :


$$ 
\begin{aligned}
\vert \bm{H}_f(x,\,y) \vert & =
\begin{vmatrix}
 \displaystyle \frac{\partial^2}{\partial x^2} f(x,\,y) & \displaystyle \frac{\partial^2}{\partial x\partial y} f(x,\,y) \\[5mm]
 \displaystyle \frac{\partial^2}{\partial y\partial x} f(x,\,y) & \displaystyle \frac{\partial^2}{\partial y^2} f(x,\,y)
\end{vmatrix} \\[4mm]
 & =
\begin{vmatrix}
r(x,y) & s(x,y) \\[3mm]
s(x,y) & t(x,y)
\end{vmatrix} \\[4mm]
 & = r(x,y) \times t(x,y) - \big(s(x,y)\big)^2  
\end{aligned}
$$


Différents cas de figure sont alors possibles :

\begin{itemize}
\item[\ding{42}] $\vert \bm{H}_f(x_0,y_0)\vert > 0 \Rightarrow \bm{M}$ est un extremum.
\item[\ding{42}] $\vert \bm{H}_f(x_0,y_0)\vert < 0 \Rightarrow \bm{M}$ n'est pas un extremum (il s'agit d'un point selle).
\item[\ding{42}] $\vert \bm{H}_f(x_0,y_0)\vert = 0 \Rightarrow \bm{M}$ est peut-être un extremum... ou pas
\end{itemize}

Dans le cas où $\bm{M}$ est un extremum, $\vert \bm{H}_f \vert = rt - s^2 > 0$, et donc les signes de $r$ et de $t$ sont les mêmes (avec $r\ne0$ et $t\ne0$). Il s'agira d'un \textbf{maximum local} si $r<0$ (ou $t<0$) et d'un \textbf{mininum local} sinon.


% test
%\tdplotsetmaincoords{70}{110}
%\begin{tikzpicture}[tdplot_main_coords]
%    \draw[thick,->] (0,0,0) -- (1,0,0) node[anchor=north east]{$x$};
%    \draw[thick,->] (0,0,0) -- (0,1,0) node[anchor=north west]{$y$};
%    \draw[thick,->] (0,0,0) -- (0,0,1) node[anchor=south]{$z$};
%\end{tikzpicture}

\begin{center}
\begin{tikzpicture}
\begin{axis}[title = {$\vert \bm{H}_f(x_0,y_0)\vert > 0 \Rightarrow$ extremum, $\quad\frac{\partial^2 f(x,\,y)}{\partial x^2} \text{ et } \frac{\partial^2 f(x,\,y)}{\partial y^2}$ positives $\Rightarrow$ minimum},
    xlabel = $x$, ylabel = $y$,
    ticks=none,
	colormap/blackwhite,
	]
    \addplot3[surf,shader=faceted interp,
    domain = -1:1,
    domain y = -1:1,
    samples = 10] {x^2 + y^2};
    
    \addplot3[red,domain=-1:1,samples y=0,line width=1.5pt] (x,0,x^2);
    \addplot3[red,domain=-1:1,samples y=0,line width=1.5pt] (0,x,x^2);
    \addplot3[green,domain=-0.5:0.5,samples y=0,line width=1.5pt] (x,0,0);
    \addplot3[green,domain=-0.5:0.5,samples y=0,line width=1.5pt] (0,x,0);
    \addplot3[only marks,point meta=explicit symbolic] coordinates { (0,0,0) };
    \node[pin=-90:$M$] at (0,0,0) {};
\end{axis}
\end{tikzpicture}



\begin{tikzpicture}
\begin{axis}[title = {$\vert \bm{H}_f(x_0,y_0)\vert > 0 \Rightarrow$ extremum, $\quad\frac{\partial^2 f(x,\,y)}{\partial x^2} \text{ et } \frac{\partial^2 f(x,\,y)}{\partial y^2}$ négatives $\Rightarrow$ maximum},
    xlabel = $x$, ylabel = $y$,
    ticks=none,
	colormap/blackwhite,
	]
    \addplot3[surf,shader=faceted interp,
    domain = -1:1,
    domain y = -1:1,
    samples = 10] {-x^2 - y^2};
    
    \addplot3[red,domain=-1:1,samples y=0,line width=1.5pt] (x,0,-x^2);
    \addplot3[red,domain=-1:1,samples y=0,line width=1.5pt] (0,x,-x^2);
    \addplot3[green,domain=-0.5:0.5,samples y=0,line width=1.5pt] (x,0,0);
    \addplot3[green,domain=-0.5:0.5,samples y=0,line width=1.5pt] (0,x,0);
    \addplot3[only marks,point meta=explicit symbolic] coordinates { (0,0,0) };
    \node[pin=90:$M$] at (0,0,0) {};
\end{axis}
\end{tikzpicture}

\begin{tikzpicture}
\begin{axis}[title = {$\vert \bm{H}_f(x_0,y_0)\vert < 0 \Rightarrow$ \textbf{pas} un extremum (point selle)},
    xlabel = $x$, ylabel = $y$,
    ticks=none,
	colormap/blackwhite,
	]
    \addplot3[surf,shader=faceted interp,
    domain = -1:1,
    domain y = -1:1,
    samples = 10] {x^2 - y^2};
    
    \addplot3[red,domain=-1:1,samples y=0,line width=1.5pt] (x,0,x^2);
    \addplot3[red,domain=-1:1,samples y=0,line width=1.5pt] (0,x,-x^2);
    \addplot3[green,domain=-0.5:0.5,samples y=0,line width=1.5pt] (x,0,0);
    \addplot3[green,domain=-0.5:0.5,samples y=0,line width=1.5pt] (0,x,0);
    \addplot3[only marks,point meta=explicit symbolic] coordinates { (0,0,0) };
    \node[pin=90:$M$] at (0,0,0) {};
\end{axis}
\end{tikzpicture}

\begin{tikzpicture}
\begin{axis}[title = {$\vert \bm{H}_f(x_0,y_0)\vert = 0 \Rightarrow$ on ne sait pas, mais il peut s'agir d'un maximum (ou minimum) :},
    xlabel = $x$, ylabel = $y$,
    ticks=none,
	colormap/blackwhite,
	]
    \addplot3[surf,shader=faceted interp,
    domain = -1:1,
    domain y = -1:1,
    samples = 10] {-x^2 - y^2 -2*x*y};
    
    \addplot3[red,domain=-1:1,samples y=0,line width=1.5pt] (x,0,-x^2);
    \addplot3[red,domain=-1:1,samples y=0,line width=1.5pt] (0,x,-x^2);
    \addplot3[green,domain=-0.5:0.5,samples y=0,line width=1.5pt] (x,0,0);
    \addplot3[green,domain=-0.5:0.5,samples y=0,line width=1.5pt] (0,x,0);
    \addplot3[only marks,point meta=explicit symbolic] coordinates { (0,0,0) };
    \node[pin={[inner sep=0pt, outer sep=0pt, pin distance=0.2cm]90:$M$}] at (0,0,0) {};
\end{axis}
\end{tikzpicture}

\begin{tikzpicture}
\begin{axis}[title = {... ou bien d'un point selle :},
    xlabel = $x$, ylabel = $y$,
    ticks=none,
	colormap/blackwhite,
	]
    \addplot3[surf,shader=faceted interp,
    domain = -1:1,
    domain y = -1:1,
    samples = 10] {y*x^2};
    
    \addplot3[red,domain=-1:1,samples y=0,line width=1.5pt] (x,0,0);
    \addplot3[red,domain=-1:1,samples y=0,line width=1.5pt] (0,x,0);
    \addplot3[green,domain=-0.5:0.5,samples y=0,line width=1.5pt] (x,0,0);
    \addplot3[green,domain=-0.5:0.5,samples y=0,line width=1.5pt] (0,x,0);
    \addplot3[only marks,point meta=explicit symbolic] coordinates { (0,0,0) };
    \node[pin=90:$M$] at (0,0,0) {};
\end{axis}
\end{tikzpicture}

\end{center}




\begin{exo}

\noindent Pour chacune des fonctions suivantes (toutes définies sur $\mathbb{R}^2$), calculer $\bm{\nabla} f(x,\,y)$ pour en déduire les points critiques. Pour ces points critiques, calculer le discriminant hessien pour dire si il est un extremum (si c'est le cas, préciser la valeur de $f$ et s'il s'agit d'un minimum ou d'un maximum).


\begin{tasks}(1)
\task $f(x,\,y) = x^2+y^2$
\task $f(x,\,y) = 1+x+y+x^2-xy+y^2$
\task $f(x,\,y) = x^3+y^3 + 3xy$
\end{tasks}
\end{exo}

\begin{sol}
\begin{tasks}(1)
\task $f(x,\,y) = x^2+y^2$,\\
$\bm{\nabla}f(x,\,y) = \vec{0} \Rightarrow \begin{cases} 2x &= 0 \\ 2y &= 0 \end{cases}$ donc le seul point critique est $M(0,0)$\\
$\vert \bm{H}_f (x,y)\vert =  \begin{vmatrix} r=2 & s=0 \\ s=0 & t=2\end{vmatrix} = 4 = \vert \bm{H}_f (0,0)\vert > 0$ \\
Donc $M(0,\, 0,\, 0)$ est une extremum, et comme $(r=2)>0$, il s'agit d'un minimum.\\
%
\begin{center}
\begin{tikzpicture}
\begin{axis}[title={$f(x,\,y) = x^2+y^2$}, xlabel = $x$, ylabel = $y$,
	colormap/blackwhite,
	]
    \addplot3[surf,shader=faceted interp,
    domain = -2:2,
    domain y = -2:2,
    samples = 10] {x^2 + y^2};
    
    \addplot3[red,domain=-2:2,samples y=0,line width=1pt] (x,0,x^2);
    \addplot3[red,domain=-2:2,samples y=0,line width=1pt] (0,x,x^2);
    \addplot3[green,domain=-0.5:0.5,samples y=0,line width=1.5pt] (x,0,0);
    \addplot3[green,domain=-0.5:0.5,samples y=0,line width=1.5pt] (0,x,0);
    \node[pin=-90:$M$] at (0,0,0) {};
    
    \addplot3[only marks,point meta=explicit symbolic] coordinates { (0,0,0) };
\end{axis}
\end{tikzpicture}
\end{center}

\task $f(x,\,y) = 1+x+y+x^2-xy+y^2$,\\
$\bm{\nabla}f(x,\,y) = \vec{0} \Rightarrow \begin{cases} 1+2x-y &= 0 \\ 1 -x + 2y &= 0 \end{cases}$ donc le seul point critique est $M(-1,-1)$\\
$\vert \bm{H}_f (x,y)\vert =  \begin{vmatrix} r=2 & s=-1 \\ s=-1 & t=2\end{vmatrix} = 3 = \vert \bm{H}_f (-1,-1)\vert > 0$ \\
Donc $M(-1,-1)$ est une extremum, et comme $(r=2)>0$, il s'agit d'un minimum.\\
%
\begin{center}
\begin{tikzpicture}
\begin{axis}[title={$f(x,\,y) = 1+x+y+x^2-xy+y^2$}, xlabel = $x$, ylabel = $y$,
	%grid,
	colormap/blackwhite,
	]
    \addplot3[surf,shader=faceted interp,
    domain = -3:2,
    domain y = -3:2,
    samples = 10] {1 + x + y + x^2 - x*y + y^2};
    
    \addplot3[red,domain=-3:2,samples y=0,line width=1pt] (x,-1,1+x-1+x^2+x+1);
    \addplot3[red,domain=-3:2,samples y=0,line width=1pt] (-1,x,x+1+x+x^2);
    \addplot3[green,domain=-1.5:-0.5,samples y=0,line width=1.5pt] (x,-1,0);
    \addplot3[green,domain=-1.5:-0.5,samples y=0,line width=1.5pt] (-1,x,0);
    \node[pin=-90:$M$] at (-1,-1,0) {};
    
    \addplot3[only marks,point meta=explicit symbolic] coordinates { (-1,-1,0) };
\end{axis}
\end{tikzpicture}
\end{center}

\task $f(x,\,y) = x^3+y^3 + 3xy$,\\
$\bm{\nabla}f(x,\,y) = \vec{0} \Rightarrow \begin{cases} 3x^2 + 3y &= 0 \\ 3y^2 + 3x  &= 0 \end{cases}$ ... $\Rightarrow x(3x^2+3) = 0 \Rightarrow x = 0 \text{ ou }-1$\\
Les valeurs de $y$ correspondantes sont respectivement 0 et -1, d'où les points critiques $A(0,0)$ et B(-1,-1).\\
$\vert \bm{H}_f (x,y)\vert =  \begin{vmatrix} r=6x & s=0 \\ s=0 & t=6y\end{vmatrix}$ \\
$\vert \bm{H}_f (0,0)\vert =  \begin{vmatrix} r=0 & s=0 \\ s=0 & t=0\end{vmatrix} = 0 \Rightarrow$ on ne sait pas.\\
$\vert \bm{H}_f (-1,-1)\vert =  \begin{vmatrix} r=-6 & s=0 \\ s=0 & t=-6\end{vmatrix} = 36 > 0 \Rightarrow B$ est un extremum.\\
Et comme $r$ et $t$ sont négatifs, $B$ est une maximum.\\
%
\begin{center}
%\begin{tikzpicture}
%\begin{axis}[title={$f(x,\,y) = x^3+y^3 + 3xy$}, xlabel = $x$, ylabel = $y$,
%	colormap/cool,
%	]
%    \addplot3[surf,shader=interp,
%    domain = -2:1,
%    domain y = -2:1,
%    samples = 30] {x^3 + y^3 + 3*x*y};
    
%    \addplot3[red,domain=-2:1,samples y=0,line width=1pt] (x,0,x^3);
%    \addplot3[red,domain=-2:1,samples y=0,line width=1pt] (0,x,x^3);
%    \addplot3[green,domain=-0.5:0.5,samples y=0,line width=1.5pt] (x,0,0);
%    \addplot3[green,domain=-0.5:0.5,samples y=0,line width=1.5pt] (0,x,0);
%    \node[pin=90:$A$] at (0,0,0) {};
    
%    \addplot3[red,domain=-2:1,samples y=0,line width=1pt] (x,-1,x^3-1-3*x);
%    \addplot3[red,domain=-2:1,samples y=0,line width=1pt] (-1,x,-1+x^3-3*x);
%    \addplot3[green,domain=-1.5:-0.5,samples y=0,line width=1.5pt] (x,-1,1);
%    \addplot3[green,domain=-1.5:-0.5,samples y=0,line width=1.5pt] (-1,x,1);
%    \node[pin=90:$B$] at (-1,-1,1) {};
    
%    \addplot3[only marks,point meta=explicit symbolic] coordinates { (0,0,0) (-1,-1,1) };
%\end{axis}
%\end{tikzpicture}


\begin{tikzpicture}
\begin{axis}[
    title={$f(x,\,y) = x^3+y^3 + 3xy$},
    xlabel=$x$,
    ylabel=$y$,
    colormap/blackwhite,
    samples=10, % Réduction du nombre d'échantillons
    shader=faceted interp, % Utilisation de faceted interp pour un rendu plus rapide
    %grid=none, % Désactive la grille
    %axis lines=center, % Affiche uniquement les axes
    %ticks=none, % Masque les graduations (optionnel)
]
    \addplot3[surf,
    domain=-2:1,
    domain y=-2:1,
    ] {x^3 + y^3 + 3*x*y};

    \addplot3[red,domain=-2:1,samples y=0,line width=1pt] (x,0,x^3);
    \addplot3[red,domain=-2:1,samples y=0,line width=1pt] (0,x,x^3);
    \addplot3[green,domain=-0.5:0.5,samples y=0,line width=1.5pt] (x,0,0);
    \addplot3[green,domain=-0.5:0.5,samples y=0,line width=1.5pt] (0,x,0);
    \node[pin=90:$A$] at (0,0,0) {};

    \addplot3[red,domain=-2:1,samples y=0,line width=1pt] (x,-1,x^3-1-3*x);
    \addplot3[red,domain=-2:1,samples y=0,line width=1pt] (-1,x,-1+x^3-3*x);
    \addplot3[green,domain=-1.5:-0.5,samples y=0,line width=1.5pt] (x,-1,1);
    \addplot3[green,domain=-1.5:-0.5,samples y=0,line width=1.5pt] (-1,x,1);
    \node[pin=90:$B$] at (-1,-1,1) {};

    \addplot3[only marks,samples=9] coordinates { (0,0,0) (-1,-1,1) };
\end{axis}
\end{tikzpicture}


\end{center}

\end{tasks}
\end{sol}


\begin{exo}

\noindent \ding{45} Même exercice pour les fonctions suivantes qui ont un domaine de définition plus compliqué.


\begin{tasks}(1)
\task $f(x,\,y) = \displaystyle \frac{1}{1-x} + \frac{1}{1-y} + \frac{1}{x+y}$, pour les points critiques $A(-1,\ -1)$; $B(-1,\ 3)$, $C(3,\ -1)$ et $D(1/3,\ 1/3)$
\task $f(x,\,y) = x(\ln^2 x + y^2)$
\task $f(x,\,y) = \displaystyle \frac{2(x+y) + 1}{xy}$
\end{tasks}
\end{exo}

\begin{sol}
\begin{tasks}(1)
\task $f(x,\,y) = \displaystyle \frac{1}{1-x} + \frac{1}{1-y} + \frac{1}{x+y}$, pour les points critiques $A(-1,\,-1)$; $B(-1,\,3)$, $C(3,\,-1)$ et $D(1/3,\,1/3)$\\
Notons que ces droites dans le plan $Oxy$ sont exclues de l'ensemble de définition : $x=1$, $y=1$ et $y=-x$.\\
On pourra vérifier, pour les points $A$ à $D$ que $\bm{\nabla} f(x,\,y) = \vec{0}$ avec :
$$\frac{\partial}{\partial x} f(x,\,y) = \frac{1}{(1-x)^2}  - \frac{1}{(x+y)^2}$$
$$\frac{\partial}{\partial y} f(x,\,y) = \frac{1}{(1-y)^2}  - \frac{1}{(x+y)^2}$$
La matrice hessienne est calculée à partir des relations suivantes :
$$r(x,y)=\frac{\partial^2}{\partial x^2} f(x,\,y) = \frac{2}{(1-x)^3}  + \frac{2}{(x+y)^3}$$
$$t(x,y)=\frac{\partial^2}{\partial y^2} f(x,\,y) = \frac{2}{(1-y)^3}  + \frac{2}{(x+y)^3}$$
$$s(x,y)=\frac{\partial^2}{\partial x \partial y} f(x,\,y) = \frac{2}{(x+y)^3}$$
Donc\\
$\vert \bm{H}_f (-1,-1)\vert =  \begin{vmatrix} r=0 & s=-2/8 \\ s=-2/8 & t=0\end{vmatrix} = -1/16 < 0 \Rightarrow A$ n'est pas un extremum.\\
$\vert \bm{H}_f (-1,3)\vert =  \begin{vmatrix} r=4/8 & s=2/8 \\ s=2/8 & t=0\end{vmatrix} = -1/16 < 0 \Rightarrow B$ n'est pas un extremum.\\
$\vert \bm{H}_f (3,-1)\vert =  \begin{vmatrix} r=0 & s=2/8 \\ s=2/8 & t=4/8\end{vmatrix} = -1/16 < 0 \Rightarrow C$ n'est pas un extremum.\\
$\vert \bm{H}_f (\frac{1}{3},\frac{1}{3})\vert =  \begin{vmatrix} r=27/2 & s=27/4 \\ s=27/4 & t=27/2\end{vmatrix} = \frac{2187}{16} > 0 \Rightarrow D$ est un minimum local à $f(1/3,1/3)=9/2$.\\
%
\begin{center}
\begin{tikzpicture}
\begin{axis}[title={$f(x,\,y) = \frac{1}{1-x} + \frac{1}{1-y} + \frac{1}{x+y}$}, xlabel = $x$, ylabel = $y$,
	%grid,
	colormap/blackwhite,
    view={10}{80},
	]
    \addplot3[surf,shader=interp,samples=60,
    domain = -2:4,
    domain y = -2:4] {1/(1-x) + 1/(1-y) + 1/(x+y)};
    
    \node[pin=90:$A$] at (-1,-1,0) {};
    \node[pin=0:$B$] at (-1,3,0) {};
    \node[pin=90:$C$] at (3,-1,0) {};
    \node[pin=90:$D$] at (0.333,0.333,0) {};
    
    \addplot3[only marks,point meta=explicit symbolic] coordinates { (-1,-1,0) (-1,3,1.5) (3,-1,1.5) (0.333,0.333,0) };
\end{axis}
\end{tikzpicture}
\end{center}


\task $f(x,\,y) = x(\ln^2 x + y^2)$, domaine de définition = $x>0$ et $y \in \mathbb{R}$
$$\frac{\partial}{\partial x} f(x,\,y) = \left( \ln^2 x  + y^2\right)  + \left( x \times 2\ln x \frac{1}{x}\right)= \ln^2 x + 2\ln x + y^2$$
$$\frac{\partial}{\partial y} f(x,\,y) = 2xy$$
Si $2xy = 0$ alors soit $x=0$, soit $y=0$, mais puisque $x>0$, la seule solution est $\boxed{y=0}$.\\
Dans la première équation, en prenant $y=0$, on obtient $\ln^2 x + 2\ln x = 0$.\\
Changement de variable $X=\ln x$ donne $X^2+2X = 0$ qui a pour solution $0$ ou $-2$.\\
$$X = 0 \Rightarrow \ln x = 0 \Rightarrow \boxed{x = 1}$$ 
$$X = -2 \Rightarrow \ln x = -2 \Rightarrow \boxed{x = e^{-2}}$$
Les points critiques sont donc $A(1,0)$ et $B(e^{-2},0)$.\\
La matrice hessienne est calculée à partir des relations suivantes :
$$r(x,y)=\frac{\partial^2}{\partial x^2} f(x,\,y) = \frac{2(\ln x +1)}{x}$$
$$t(x,y)=\frac{\partial^2}{\partial y^2} f(x,\,y) = 2x$$
$$s(x,y)=\frac{\partial^2}{\partial x \partial y} f(x,\,y) = 2y$$
Donc\\
$\vert \bm{H}_f (1,0)\vert =  \begin{vmatrix} r=2 & s=0 \\ s=0 & t=2\end{vmatrix} = 4 > 0 \Rightarrow A$ est un minimum.\\
$\vert \bm{H}_f (e^{-2},0)\vert =  \begin{vmatrix} r=-2e^{2} & s=0 \\ s=0 & t=2e^{-2}\end{vmatrix} = -4 < 0 \Rightarrow B$ pas un extremum.\\
%
\begin{center}
\begin{tikzpicture}
\begin{axis}[title={$f(x,\,y) = x(\ln^2 x + y^2)$}, xlabel = $x$, ylabel = $y$,
	%grid,
	colormap/blackwhite,
    view={-30}{40},
	]
    \addplot3[surf,shader=faceted interp,samples=10,
    domain = 0:1.5,
    domain y = -1.5:1.5] {x*(ln(x)*ln(x) + y^2) };
    
    \node[pin=90:$A$] at (1,0,0) {};
    \node[pin=90:$B$] at (0.13533,0,0.54132) {};
    
    \addplot3[only marks,point meta=explicit symbolic] coordinates { (1,0,0) (0.13533,0,0.54132) };
\end{axis}
\end{tikzpicture}
\end{center}

\task $f(x,\,y) = \displaystyle \frac{2(x+y) + 1}{xy}$, $x\ne0$ et $y\ne0$
$$\frac{\partial}{\partial x} f(x,\,y) = -\frac{2y+1}{x^2y}$$
$$\frac{\partial}{\partial y} f(x,\,y) = -\frac{2x+1}{xy^2}$$
Le seul point critique est $A(-\frac{1}{2},-\frac{1}{2})$ 
$$r(x,y)=\frac{\partial^2}{\partial x^2} f(x,\,y) = \frac{4y+2}{x^3y}$$
$$t(x,y)=\frac{\partial^2}{\partial y^2} f(x,\,y) =  \frac{4x+2}{xy^3}$$
$$s(x,y)=\frac{\partial^2}{\partial x \partial y} f(x,\,y) = \frac{1}{(xy)^2}$$
Donc\\
$\vert \bm{H}_f (-1/2,-1/2)\vert =  \begin{vmatrix} r=0 & s=16 \\ s=16 & t=0\end{vmatrix} = 16^2 < 0 \Rightarrow A$ pas extremum.\\
%
\begin{center}
\begin{tikzpicture}
\begin{axis}[
    title={$f(x,\,y) = \frac{2(x+y) + 1}{xy}$}, 
    xlabel = $x$, ylabel = $y$,	
    view={-30}{40},
    colormap/blackwhite
    ]
    \addplot3[surf,shader=faceted interp,samples=50,
    domain = -1.:1.,
    domain y = -1.:1.] {(2*(x+y) + 1)/(x*y)};
    
    \node[pin=90:$A$] at (-0.5,-0.5,12) {};
    
    \addplot3[only marks,point meta=explicit symbolic] coordinates { (-0.5,-0.5,12) };
\end{axis}
\end{tikzpicture}
\end{center}


\end{tasks}
\end{sol}

 % cours Manu + cours Vincent
%!TEX root = main.tex

\chapter{Calcul d'erreurs}
\label{chapter:calc-erreurs}



%%%%%%%%%%%%%%%%%%%%%%%%%%%%%%%%%%%%%%%%%%%%%%%%%%%%%%%%%%%%%%%%%%%%%%%%%%%%%%%%
\section{Calcul d'erreurs absolues et relatives}


Nous verrons ici comment estimer une erreur faite sur l’estimation d’une grandeur lorsqu'il est possible d'avoir une expression théorique de cette grandeur en fonction d'autres variables elles-mêmes sujettes à des incertitudes.
Cette quantification d’erreur est très importante et absolument fondamentale lorsqu'on mesure expérimentalement une grandeur. Elle permet un regard objectif sur le domaine de validité et les incertitudes sur les ordres de grandeur des mesures.


En guise d'illustration, prenons une situation que nous avons déjà rencontrée en TP de Mécanique des Structures. La flèche $f$ en milieu de portée d'une poutre de longueur $L$, de poids linéique $q$, et de section carrée $b^2$, posée sur deux appuis est :
%
$$
\displaystyle f = \frac{qL^4}{6.4\, E b^4}
$$
%
Si nous souhaitons faire une mesure expérimentale du module d'Young $E$ de son matériau, on peut faire une petite expérience, et calculer $E$ comme suit :
$$
E = E(q,L,f,b) = \frac{qL^4}{6.4\, f b^4}
$$
%
La question est alors : quelle précision peut-on espérer obtenir sur la valeur de $E$ que l'on aura déterminée à partir de $q$, $L$, $b$ et une mesure de $f$ ?


Pour y répondre, nous avons besoin d'une notion de calcul différentiel (que nous ne détaillerons pas) :
%
$$
\displaystyle
\phi(x_1,\ x_2,\ \cdots) \Rightarrow \Delta \phi(x_1,\ x_2,\ \cdots) = \left\vert\frac{\partial \phi(\cdots)}{\partial x_1}\right\vert \Delta x_1 + \left\vert\frac{\partial \phi(\cdots)}{\partial x_2}\right\vert \Delta x_2 + \cdots
$$
De façon plus concise :
$$
\displaystyle
\Delta \phi(x_1,\ x_2,\ \cdots) = \sum_{i=1}^{n} \left\vert\frac{\partial \phi(x_1,\ \cdots \ ,x_n)}{\partial x_i}\right\vert \Delta x_i 
$$


Cette relation permet de connaître l'erreur absolue $\Delta \phi$ que l'on fait, en fonction des estimations des incertitudes $\Delta x_i$ sur les variables dont elle dépend. Il est souvent plus intéressant d'exprimer cette erreur relativement à la valeur mesurée. Il s'agit de l'erreur relative $\Delta \phi / \phi$.

On notera une grandeur quelconque $G \pm \Delta G$ pour définir sa gamme de valeurs possibles.


Reprenons l'exemple de l'estimation de $E$ à partir de la flexion d'une poutre. Le poids linéique est $q = 25 \pm 1$~N/m, la portée est $L = 4.00 \pm 0.005$~m, le côté de section est $b = 0.050 \pm 0.001$~m, et on mesure une flèche $f = 0.010 \pm 0.002$~m. L'expression de $E$ permet de définir son erreur absolue :
%
$$
\displaystyle
\Delta E = \left\vert\frac{\partial E}{\partial q}\right\vert\Delta q + \left\vert\frac{\partial E}{\partial L}\right\vert\Delta L + \left\vert\frac{\partial E}{\partial b}\right\vert\Delta b + \left\vert\frac{\partial E}{\partial f}\right\vert\Delta f 
$$ 
%
Ce qui se ré-écrit en déterminant les dérivées partielles :
%
$$
\displaystyle
\Delta E = \frac{L^4}{6.4\, f b^4}\Delta q + \frac{4qL^3}{6.4\, f b^4} \Delta L + \frac{4qL^4}{6.4\, f b^5}\Delta b + \frac{qL^4}{6.4\, f^2 b^4}\Delta f 
$$ 
%
Après calcul, on a : 
%
$$
\displaystyle
\Delta E = (6.4\times10^{8}) \Delta q + (1.6\times10^{10}) \Delta L + (1.28\times10^{12}) \Delta b + (1.6\times10^{12}) \Delta f 
$$ 
$$
\displaystyle
\Delta E = (6.4\times10^{8})\times 1 + (1.6\times10^{10})\times 0.005 + (1.28\times10^{12})\times 0.001 + (1.6\times10^{12})\times 0.002 
$$ 
$$
\displaystyle
\Delta E = 5.2\times 10^{9}\text{ Pa} = 5.2\text{ GPa}
$$ 
%
On peut par ailleurs calculer une valeur de $E(q,\,L,\,b,\,f) = 16$~GPa, mais compte tenu des incertitudes, on doit plutôt dire :
%
$$
E = 16 \pm 5.2 \text{ GPa}
$$
%
On comprend ainsi que cette façon de mesurer un module d'Young n'est pas du tout précise ! On comprend également que c'est le manque de précision sur $L$ et sur $b$ qui est le plus pénalisant, car ils apparaissent avec une puissance de 3 et 4, respectivement.



\begin{exo}
\noindent \ding{45} Quelle erreur (absolue et relative) est faite sur la surface $S = a\times b$ d'un terrain rectangulaire si ses dimensions ont la même incertitude $\Delta \ell$ ?\\
Si $a=31$~m et $b=29$~m sont connues au mètre près, quantifier ces erreurs.
\end{exo}


\begin{sol}
\noindent Erreur absolue : $\Delta S = b \Delta \ell + a \Delta \ell = (a+b)\Delta \ell $\\
Erreur relative : $\frac{\Delta S}{S} = \frac{a+b}{ab} \Delta \ell $\\
Application numérique : l'erreur pourra aller jusqu'à $\Delta S = (31+29) \times 1 = 60$~m$^2$, en plus ou en moins, ce qui rapporté à la surface du terrain est une erreur de $60/(31\times 29)\simeq 0.0667$, soit environ $\pm6.67$~\%.
\end{sol}

\needspace{8cm}
\begin{exo}
\noindent \ding{45} La fréquence de résonance $f$ d'un circuit RLC a pour expression : $ f=\frac{1}{2\pi \sqrt{LC}}$\\
On donne : $L = 0.40\pm0.01$~henrys, et $C=(800\pm1)\times 10^{-6}$~farads.\\
Calculer la fréquence de résonance (en hertz) et ses incertitudes absolue et relative. 
\end{exo}


\begin{sol}
$$
\Delta f = \left\vert \frac{\partial f}{\partial L} \right \vert \Delta L + \left\vert \frac{\partial f}{\partial C} \right \vert \Delta C
$$
$$
\Delta f = \frac{C}{4\pi\sqrt{(CL)^3}} \Delta L + \frac{L}{4\pi\sqrt{(CL)^3}} \Delta C
$$
$$
\Delta f = \frac{800\times 10^{-6}}{4\pi\sqrt{(800\times 10^{-6} \times 0.40)^3}}\times 0.01 + \frac{0.40}{4\pi\sqrt{(800\times 10^{-6} \times 0.40)^3}}\times 10^{-6} \simeq 0.12
$$
$$
\boxed{f=8.90\pm0.12 \text{ Hz}}
$$
\end{sol}


\begin{exo}
\noindent \ding{45} La vitesse d'un corps lâché d'une hauteur $h$ est $v=\sqrt{2gh}$ (où $g=9.81\pm0.05$~m/s$^2$ est l'accélération de pesanteur sur terre).\\
\begin{itemize}
\item[\ding{45}] Quelle est la hauteur de lâcher si on mesure une vitesse d'impact au sol de $24$~m/s avec une précision de $\pm1$~m/s
\item[\ding{45}] Si on réalise un lâcher à une hauteur de $25.0\pm0.1$~m, et qu'on mesure une vitesse d'impact de $22\pm1$~m/s, donner une estimation de $g$. 
\end{itemize}
\end{exo}


\begin{sol}
$$
h(v,g)=\frac{v^2}{2g} \Rightarrow \tfrac{\partial}{\partial v}h(v,g) = \frac{v}{g} \text{ et } \tfrac{\partial}{\partial g}h(v,g) = \frac{-v^2}{2g^2}
$$
$$
h=\frac{24^2}{2 \times 9.81} \simeq 29.36\text{ m}
$$
$$
\Delta h = \frac{v}{g} \Delta v + \frac{v^2}{2g^2} \Delta g = \frac{24}{9.81} \times 1 + \frac{24^2}{2\times 9.81^2} \times 0.05 \simeq 2.60\text{ m}
$$
$$
\boxed{h=29.36\pm2.60 \text{ m}}
$$
$$
g(v,h)=\frac{v^2}{2h} \Rightarrow \tfrac{\partial}{\partial v}g(v,g) = \frac{v}{h} \text{ et } \tfrac{\partial}{\partial h}g(v,g) = \frac{-v^2}{2h^2}
$$
$$
g=\frac{22^2}{2 \times 25} = 9.68\text{ m/s}^2
$$
$$
\Delta g = \frac{v}{h} \Delta v + \frac{v^2}{2h^2} \Delta h = \frac{22}{25} \times 1 + \frac{22^2}{2\times 25^2} \times 0.1 \simeq 0.92\text{ m}
$$
$$
\boxed{g=9.68\pm0.92 \text{ m/s}^2}
$$
\end{sol}

\needspace{15cm}
\begin{exo}
\noindent \ding{45} Nous avons vu dans l'exercice précédant, que l'estimation de l'accélération $g$ de pesanteur sur terre était plutôt mauvaise à partir de la mesure de la vitesse d'un corps en chute libre (c'est pourtant comme ça qu'on la mesure de nos jours, mais avec un dispositif sophistiqué).

Galilée est l'un des premiers à s'intéresser au mouvement des pendules simples. Il découvre en 1632 que la période $T$ du pendule ne dépend pas  de la masse qui est suspendue, mais uniquement de la longueur $L$ du fil. En 1659, Huygens trouve l'expression exacte de la période d'un pendule :
$$
T = 2\pi \sqrt{\frac{L}{g}}
$$
%
\begin{itemize}
\item[\ding{45}] Déterminer l’expression de $g$ en fonction de $L$ et de $T$.
\item[\ding{45}] Déterminer l'expression donnant l'incertitude $\Delta g$ en fonction des incertitudes sur la longueur $L$ et sur le temps $T$ que met la masse pour revenir à son point de départ (notés respectivement $\Delta L$ et $\Delta T$).
\item[\ding{45}] Un expérience permet de mesurer une période de $0.78\pm0.02$~s lorsqu'une masse de $100\pm5$~gramme est attachée au bout d'un fil de $15.0\pm0.2$~cm. Calculer l’accélération de la pesanteur en m/s$^2$ et ses incertitudes absolue et relative.
\end{itemize}
\end{exo}


\begin{sol}
$$
g(L,T)=4L\left(\frac{\pi}{T}\right)^2
$$
$$
\Delta g(L,T)= \frac{4\pi^2}{T^2} \Delta L + \frac{8L\pi^2}{T^3} \Delta T
$$
$$
\Delta g = \frac{4\pi^2}{0.78^2} \times 0.002 + \frac{8\times0.15\times\pi^2}{0.78^3}\times 0.02 \simeq 0.52\text{ m/s}^2
$$
$$
\boxed{g = 9.73\pm0.52\text{ m/s}^2 \qquad \frac{\Delta g}{g} = 5.3\%}
$$
\end{sol}

\needspace{8cm}
\begin{exo}
\noindent Historiquement, une estimation de la circonférence de la terre a été proposée par \'Eratosthène sur la base de la figure suivante.

\begin{center}
\includegraphics[width=6cm]{figs/eratosthene.png}
\end{center}

Selon son raisonnement (pas détaillé ici), la circonférence $C$ de la terre pouvait être calculée à partir de la distance $D$ entre Alexandrie et Syène, et l'angle $\theta$ (en degré) :  
$$
C = \frac{360^\circ}{\theta}D
$$

La distance entre Syène et Alexandrie était de $D=789\pm40$~km (sauf qu'à l'époque elle se mesurait en nombre de stades), 
et l'angle $\theta$ pouvait être déterminer en mesurant l'ombre portée d'un bâton (gnomon) au moment où les rayons de soleil arrivaient au fond d'un puis à Syène (comme sur la figure).
\begin{itemize}
\item[\ding{45}] On mesure une ombre portée de $L = 12.5\pm0.5$~cm d'un bâton d'un mètre. Que vaut l'angle $\theta$ et son incertitude $\Delta \theta$ en degré ?\\
Rappel : $\displaystyle \frac{\text{d}}{\text{d}x} \tan^{-1} \, (x) = \frac{1}{1+x^2}$
\item[\ding{45}] Quelle est donc la circonférence $C$ de la terre, et quelle erreur absolue commet-on sur cette estimation ?
\end{itemize}

\end{exo}


\begin{sol}
$$ 
\tan \theta = \frac{L}{100} \Rightarrow \theta(L) = \tan^{-1} \left( \frac{L}{100} \right) = \tan^{-1} \left( \frac{12.5}{100} \right) \simeq \boxed{7.125\text{ degré}}
$$
$$
\frac{\partial}{\partial L} \theta(L) =\frac{1}{1+\frac{L^2}{100^2}} \times \frac{1}{100} = \frac{1}{100+\frac{L^2}{100}}
= \frac{1}{100+\frac{(12.5)^2}{100}} \simeq 0.01
$$
$$
\Delta \theta = \frac{\partial \theta(L)}{\partial L}  \Delta L = 0.01 \times 0.5 = \boxed{0.005\text{ degré}}
$$
$$
\boxed{\theta = 7.125\pm0.005\text{ degré}}
$$
$$
C = \frac{360}{7.125}\times 789 = \boxed{39865\text{ km}}
$$
$$
\frac{\partial C(\theta, D)}{\partial \theta} = -\frac{360}{\theta^2}D
\quad\text{et}\quad
\frac{\partial C(\theta, D)}{\partial D} = \frac{360}{\theta}
$$
$$
\Delta C = \frac{360}{\theta^2}D \times \Delta \theta  +   \frac{360}{\theta}  \times \Delta D
= \frac{360}{7.125^2}789 \times 0.005  +   \frac{360}{7.125}  \times 40 \simeq \boxed{28 + 2021 = 2049\text{ km}}
$$
Donc la circonférence de la terre est estimée à $\boxed{39865\pm2049\text{ km}}$, ce qui correspond très bien aux estimations actuelles de l'ordre de 40000~km à l'équateur ou par les pôles. On peut remarquer que l'erreur la plus pénalisante vient de l'estimation de la distance $D$ et que l'erreur sur l'angle $\theta$ est très faible. 

En terme d'erreur relative : $\Delta C/C = \Delta C_\theta/C + \Delta C_D/C = 0.1\%+ 5.0\% = 5.1\%$ 
\end{sol}

 % cours Manu + cours Vincent

% MAT5 ?

% Python ?

%\input{appendix}

\end{document}
