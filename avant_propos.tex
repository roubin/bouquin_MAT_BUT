%!TEX root = main.tex

\chapter*{Avant propos}
\label{chapter:avant-propos}

Ce manuel de mathématiques a été conçu à partir des enseignements dispensés à l'IUT de Grenoble dans le cadre du BUT Génie Civil -- Construction Durable (GCCD). Il s'adresse avant tout aux étudiants de cette formation, mais pourra également servir de référence à toute personne souhaitant acquérir ou consolider les bases mathématiques nécessaires à la compréhension des phénomènes rencontrés en génie civil.

Les chapitres réunis dans cet ouvrage couvrent un ensemble de notions essentielles au programme du BUT. Certains d'entre eux entretiennent un lien direct avec les applications du génie civil : la topographie, la mécanique des structures ou encore la modélisation de phénomènes physiques y trouvent naturellement leur place. D'autres, en revanche, traitent de concepts plus abstraits ou dont la relation avec le domaine professionnel est moins immédiate et/ou évidente. Leur présence répond à un autre objectif tout aussi important : préparer certains  étudiants à la poursuite d'études et à la maîtrise des outils mathématiques indispensables à la compréhension de modèles plus complexes.

L'apprentissage des mathématiques ne se limite pas à l'application de formules. C'est une discipline de raisonnement et de méthode, qui demande rigueur, curiosité et persévérance. C'est pourquoi chaque chapitre comporte des exercices de nature variée. Certains sont purement mathématiques : ils permettent de se familiariser avec les notions fondamentales, d'acquérir les automatismes et de développer la logique nécessaire à toute démarche scientifique. D'autres proposent un ancrage dans des situations concrètes, souvent inspirées du génie civil, pour montrer que les mathématiques ne sont pas une fin en soi mais un outil puissant de compréhension et de conception.

L'objectif de ce manuel est donc double : offrir une formation solide et structurée en mathématiques, et en même temps, aider les étudiants à percevoir la portée et l'utilité de cette discipline dans leur futur métier. Nous espérons qu'il accompagnera chacun dans la découverte, parfois exigeante mais toujours formatrice, du langage mathématique qui sous-tend les sciences de l'ingénierie et de la construction.


% Il faudra entrer dans le détail du contenu chapitre par chapitre

