%!TEX root = main.tex

\chapter{Géométrie analytique tridimensionnelle}
\label{chapter:geospace}


\section{Introduction}



%%%%%%%%%%%%%%%%%%%%%%%%%%%%%%%%%%%%%%%%%%%
\section{Produits scalaire et vectoriel (rappels et extension tridimensionnelle)}


Il existe différentes solutions pour définir un vecteur. La solution la plus couramment rencontrée est la connaissance de ses composantes dans un système cartésien bidimensionnel (2D) ou tridimensionnel (3D). 
En physique, ces composantes auront la même unité que le vecteur ; par exemple, des mètres, des newtons, des pascals, etc.
Une autre solution, plutôt utilisée en configuration 2D, est de se donner \uuline{la norme}, \uline{le sens} et l'\uwave{inclinaison} du vecteur. Par exemple, on peut définir un vecteur force ainsi : ``une force \uwave{verticale} de \uuline{2~kN} \uline{vers le bas}''.


Le \textbf{produit scalaire} est une opération de multiplication entre deux vecteurs dont le résultat est un scalaire (c'est-à-dire un simple nombre). Le symbole de cette multiplication est un point.
%
\begin{enumerate}
\item Si les composantes sont connues
%
\begin{equation*}
\vv{a}\cdot \vv{b} = \begin{pmatrix} a_x\\a_y\\a_z \end{pmatrix} \cdot \begin{pmatrix} b_x\\b_y\\b_z \end{pmatrix}
= a_x \times b_x + a_y \times b_y + a_z \times b_z
\end{equation*}

\item Si les normes et les orientations des deux vecteurs sont connues
%
\begin{equation*}
\vv{a} \cdot \vv{b} = \Vert\vv{a}\Vert \times \Vert\vv{b}\Vert \times \cos(\vv{a},\vv{b})
\end{equation*}
\end{enumerate}
%
Le produit scalaire est commutatif : $\vv{a} \cdot \vv{b} = \vv{b} \cdot \vv{a}$. Il est utile en géométrie pour réaliser des projections. La valeur projetée du vecteur $\vv{a}$ sur une direction donnée par le vecteur unitaire $\vv{u}$ est $\vv{a} \cdot \vv{u}$.

La \textbf{norme} d'un vecteur est définie, de façon similaire aux simples nombres réels, comme ``la racine carré de son carré'' : 
\begin{equation*}
\Vert \vv{a} \Vert = \sqrt{\vv{a}^{\, 2}} = \sqrt{\vv{a}\cdot\vv{a}} = \sqrt{a_x^2+a_y^2+a_z^2}
\end{equation*}


\begin{exo}

\noindent \ding{45} Calculer la norme $\Vert \vv{v} \Vert = \sqrt{v_x^2 + v_y^2 + v_z^2}$ des vecteurs suivants pour en déduire le vecteur $\vv{u}_v$ de norme unitaire associé (on dira qu'on ``normalise'' le vecteur $\vv{v} \rightarrow \vv{u}_v = \vv{v}/\Vert\vv{v}\Vert$.

\begin{tasks}(3)
\task $\vv{v}=\begin{pmatrix} 3 \\ 2 \end{pmatrix}$
\task $\vv{v}=\begin{pmatrix} 2 \\ 4 \end{pmatrix}$
\task $\vv{v}=\begin{pmatrix} 2 \\ 2 \\ 2 \end{pmatrix}$
\task $\vv{v}=\begin{pmatrix} 3 \\ 2 \\ 1 \end{pmatrix}$
\task $\vv{v}=\begin{pmatrix} \pi \\ 2\pi \end{pmatrix}$
\end{tasks}

\end{exo}


\begin{sol}
\begin{tasks}(1)
\task $\left \Vert \begin{pmatrix} 3 \\ 2 \end{pmatrix} \right \Vert = \sqrt{9+4} = \sqrt{13}$ se normalise en $\begin{pmatrix} 3/\sqrt{13} \\ 2/\sqrt{13} \end{pmatrix}$
\task $\left \Vert\begin{pmatrix} 2 \\ 4 \end{pmatrix}\right \Vert = \sqrt{4+16} = \sqrt{20} = 2\sqrt{5}$ se normalise en $\begin{pmatrix} 1/\sqrt{5} \\ 2/\sqrt{5} \end{pmatrix}$
\task $\left \Vert\begin{pmatrix} 2 \\ 2 \\ 2 \end{pmatrix}\right \Vert = \sqrt{16} = 4$ se normalise en $\begin{pmatrix} 1/2 \\ 1/2 \\ 1/2 \end{pmatrix}$
\task $\left \Vert\begin{pmatrix} 3 \\ 2 \\ 1 \end{pmatrix}\right \Vert = \sqrt{9+4+1} = \sqrt{14}$ se normalise en $\begin{pmatrix} 3/\sqrt{14} \\ 2/\sqrt{14} \\ 1/\sqrt{14} \end{pmatrix}$
\task $\left \Vert\begin{pmatrix} \pi \\ 2\pi \end{pmatrix}\right \Vert = \sqrt{\pi^2+4\pi^2} = \pi\sqrt{5}$ se normalise en $\begin{pmatrix} 1/\sqrt{5} \\ 2/\sqrt{5} \end{pmatrix}$
\end{tasks}
\end{sol}



Pour calculer la valeur projetée $a_u$ d'un vecteur $\vv{a}$ sur une droite de vecteur directeur unitaire $\vv{u}$, il suffit de faire une multiplication scalaire : $a_u = \vv{a}\cdot\vv{u}$.

Si le vecteur directeur $\vv{d}$ de la droite sur laquelle on veut projeter n'est pas unitaire (c'est-à-dire que sa norme n'est pas égale à 1), alors il faut préalablement le normaliser : 
%
\begin{equation*}
a_d = \vv{a}\cdot \frac{\vv{d}}{\Vert \vv{d} \Vert} = \frac{\vv{a}\cdot\vv{d}}{\Vert \vv{d} \Vert}
\end{equation*}

Pour calculer le vecteur projeté $\vv{a}_u$ d'un vecteur $\vv{a}$ sur une droite de vecteur directeur unitaire $\vv{u}$, il suffit de multiplier le vecteur unitaire $\vv{u}$ par la valeur projetée $a_u$ : $\vv{a}_u = a_u \vv{u}$.

Si le vecteur directeur $\vv{d}$ de la droite sur laquelle on veut projeter n'est pas unitaire, alors il faut préalablement le normaliser (le rendre unitaire) :
%
\begin{equation*}
\displaystyle \vv{a}_d = a_u \times \frac{\vv{d}}{\Vert \vv{d} \Vert} = \frac{\vv{a}\cdot\vv{d}}{\Vert \vv{d} \Vert} \times \frac{\vv{d}}{\Vert \vv{d} \Vert} = \frac{(\vv{a}\cdot\vv{d})}{\Vert \vv{d} \Vert^2} \times \vv{d}
\end{equation*}


\begin{exo}

\noindent \ding{45} Déterminer la valeur projetée $v_{AB}$ du vecteur $\vv{v}$ sur la droite $(AB)$ à l'aide des données suivantes.

\begin{tasks}[item-indent=4pt, column-sep=2em](2)
\task $\vv{v} = \begin{pmatrix} 3 \\ 2 \end{pmatrix}$ et $\vv{AB} = \begin{pmatrix} -1 \\ 1 \end{pmatrix}$
\task $\vv{v} = \begin{pmatrix} -2 \\ 2 \end{pmatrix}$ et $\vv{AB} = \begin{pmatrix} 10 \\ 0 \end{pmatrix}$
\task $\vv{v} = \begin{pmatrix} 2 \\ 3 \\ 1 \end{pmatrix}$  et $\vv{AB} = \begin{pmatrix} 1 \\ 0 \\ -1 \end{pmatrix}$
\task $\vv{v} = \begin{pmatrix} 2 \\ 1 \\ 3 \end{pmatrix}$, $\vv{OA} = \begin{pmatrix} 1 \\ 1 \\ 1 \end{pmatrix}$ et $\vv{OB} = \begin{pmatrix} 1 \\ 0 \\ -1 \end{pmatrix}$
\end{tasks}

\end{exo}


\begin{sol}
\begin{tasks}(1)
\task $\displaystyle v_{AB}=\begin{pmatrix} 3 \\ 2 \end{pmatrix} \cdot \begin{pmatrix} -1/\sqrt{2} \\ 1/\sqrt{2} \end{pmatrix} = \frac{-3}{\sqrt{2}} +  \frac{2}{\sqrt{2}} =  \frac{-1}{\sqrt{2}}$
\task $\displaystyle v_{AB}=\begin{pmatrix} -2 \\ 2 \end{pmatrix} \cdot \begin{pmatrix} 1 \\ 0 \end{pmatrix} = -2$
\task $\displaystyle v_{AB}=\begin{pmatrix} 2 \\ 3 \\ 1 \end{pmatrix} \cdot \begin{pmatrix} 1/\sqrt{2} \\ 0  \\ -1/\sqrt{2} \end{pmatrix} = \frac{2}{\sqrt{2}} - \frac{1}{\sqrt{2}} = \frac{1}{\sqrt{2}} $
\task $\displaystyle v_{AB}=\begin{pmatrix} 2 \\ 1 \\ 3 \end{pmatrix} \cdot \begin{pmatrix} 0 \\ -1/\sqrt{5}  \\ -2/\sqrt{5} \end{pmatrix} = \frac{-1}{\sqrt{5}} - \frac{6}{\sqrt{5}} = \frac{-7}{\sqrt{5}} $
\end{tasks}
\end{sol}

\needspace{4cm}

\begin{exo}

\noindent \ding{45} Déterminer la force projetée $\vv{F}_{AB}$ de la force $\vv{F}$ sur la droite $(AB)$ à l'aide des données suivantes.

\begin{tasks}(2)
\task $\vv{F} = \begin{pmatrix} 2 \\ 3 \end{pmatrix}$ et $\vv{AB} = \begin{pmatrix} 2 \\ -1 \end{pmatrix}$
\task $\vv{F} = \begin{pmatrix} qa \\ T \end{pmatrix}$ et $\vv{AB} = \begin{pmatrix} 1/\sqrt{2} \\ 1/\sqrt{2} \end{pmatrix}$
\task $\vv{F} = \begin{pmatrix} 1 \\ 2 \\ 2 \end{pmatrix}$  et $\vv{AB} = \begin{pmatrix} -1 \\ 1 \\ 0 \end{pmatrix}$
\task $\vv{F} = \begin{pmatrix} -1 \\ 2 \\ 1 \end{pmatrix}$  et $\vv{AB} = \begin{pmatrix} 1 \\ -1 \\ 2 \end{pmatrix}$
\task*(2) $\vv{F} = \begin{pmatrix} qL \\ T \\ T\cos(\alpha) \end{pmatrix}$, $\vv{OA} = \begin{pmatrix} 1 \\ 3 \\ 3 \end{pmatrix}$ et $\vv{OB} = \begin{pmatrix} 2 \\ 3 \\ 2 \end{pmatrix}$
\end{tasks}

\end{exo}


\begin{sol}
\begin{tasks}(1)
\task $\displaystyle \vv{F}_{AB} = \left \{ \begin{pmatrix} 2 \\ 3 \end{pmatrix}\cdot \begin{pmatrix} 2/\sqrt{5} \\ -1/\sqrt{5} \end{pmatrix} \right \} \begin{pmatrix} 2/\sqrt{5} \\ -1/\sqrt{5} \end{pmatrix} = \frac{4-3}{\sqrt{5}} \begin{pmatrix} 2/\sqrt{5} \\ -1/\sqrt{5} \end{pmatrix} = \begin{pmatrix} 2/5 \\ -1/5 \end{pmatrix} $
\task $\displaystyle \vv{F}_{AB} = \left \{ \begin{pmatrix} qa \\ T \end{pmatrix}\cdot \begin{pmatrix} 1/\sqrt{2} \\ 1/\sqrt{2} \end{pmatrix} \right \} \begin{pmatrix} 1/\sqrt{2} \\ 1/\sqrt{2} \end{pmatrix} = \frac{qa+T}{\sqrt{2}} \begin{pmatrix} 1/\sqrt{2} \\ 1/\sqrt{2} \end{pmatrix} = \begin{pmatrix} (qa+T)/2 \\ (qa+T)/2 \end{pmatrix} = \frac{qa+T}{2} \begin{pmatrix} 1 \\ 1 \end{pmatrix}$
\task $\displaystyle \vv{F}_{AB} = \left \{ \begin{pmatrix} 1 \\ 2 \\ 2 \end{pmatrix}\cdot \begin{pmatrix} -1/\sqrt{2} \\ 1/\sqrt{2} \\ 0 \end{pmatrix} \right \} \begin{pmatrix} -1/\sqrt{2} \\ 1/\sqrt{2} \\ 0 \end{pmatrix} = \frac{-1+2}{\sqrt{5}} \begin{pmatrix} -1/\sqrt{2} \\ 1/\sqrt{2} \\ 0 \end{pmatrix} = \begin{pmatrix} -1/2 \\ 1/2 \\ 0 \end{pmatrix} $
\task $\displaystyle \vv{F}_{AB} = \left \{ \begin{pmatrix} -1 \\ 2 \\ 1 \end{pmatrix}\cdot \begin{pmatrix} 1/\sqrt{6} \\ -1/\sqrt{6} \\ 2/\sqrt{6} \end{pmatrix} \right \} \begin{pmatrix} 1/\sqrt{6} \\ -1/\sqrt{6} \\ 2/\sqrt{6} \end{pmatrix} = \frac{-1}{\sqrt{6}}\begin{pmatrix} 1/\sqrt{6} \\ -1/\sqrt{6} \\ 2/\sqrt{6} \end{pmatrix} = \begin{pmatrix} -1/6 \\ 1/6 \\ -2/6 \end{pmatrix} $
\task $\displaystyle \vv{F}_{AB} = \left \{ \begin{pmatrix} qL \\ T \\ T\cos(\alpha) \end{pmatrix}\cdot \begin{pmatrix} 1/\sqrt{2} \\ 0 \\ -1/\sqrt{2} \end{pmatrix} \right \} \begin{pmatrix} 1/\sqrt{2} \\ 0 \\ -1/\sqrt{2} \end{pmatrix} = \frac{qL-T\cos(\alpha)}{2}\begin{pmatrix} 1 \\ 0 \\ -1 \end{pmatrix} $
\end{tasks}
\end{sol}


Le \textbf{produit vectoriel} est une autre opération de multiplication entre deux vecteurs. Elle est très utile en géométrie (mais pas uniquement). Le résultat d'un produit vectoriel est un vecteur, de direction perpendiculaire à chacun des vecteurs multipliés, et dont le sens est donné par la ``règle de la main droite'' : 
%
\begin{center}
\includegraphics[width=4cm]{figs/right_hand.jpg}
\end{center}
%
Dans l'ordre dans lequel on dit ``$\,\vv{u}$ vectoriel $\vv{v}$ égal $\vv{u}\wedge\vv{v}\ $'', on pense ``$1\wedge2 = 3$'', ce qui correspond à l'ordre des doigts de la main \textbf{droite} : pouce, index, majeur. La direction du majeur est alors la direction de $\vv{u}\wedge\vv{v}$. La norme de $\vv{u}\wedge\vv{v}$ est :
%
\begin{equation*}
\Vert \vv{u}\wedge\vv{v} \Vert = \Vert \vv{u} \Vert \times \Vert \vv{v} \Vert \times \vert \sin(\vv{u}, \vv{v}) \vert
\end{equation*}

Il est également possible de faire l'opération de produit vectoriel à partir des composantes des vecteurs :
%
\begin{equation*}
\vv{a} \times \vv{b} = \vv{a} \wedge \vv{b} =
\begin{pmatrix} a_x \\ a_y \\ a_z \end{pmatrix} \wedge
\begin{pmatrix} b_x \\ b_y \\ b_z \end{pmatrix}
=
\begin{pmatrix} a_y b_z - a_z b_y  \\ a_z b_x - a_x b_z \\ a_x b_y - a_y b_x \end{pmatrix}
\end{equation*}

\needspace{5cm}
Nous verrons en TD comment ``retenir'' cette formule. Remarquons que le symbol utilisé en france est $\wedge$, alors que la croix $\times$ est plus communément utilisée dans le reste du monde. 


\begin{exo}

\noindent \ding{45} Calculer les produits vectoriels suivants

\begin{tasks}(2)
\task $\begin{pmatrix} 1 \\ 2 \\ 3 \end{pmatrix} \wedge
\begin{pmatrix} -1 \\ 2 \\ 1 \end{pmatrix} $
\task $\begin{pmatrix} -1 \\ 4 \\ 2 \end{pmatrix} \wedge
\begin{pmatrix} 1 \\ 0 \\ 5 \end{pmatrix} $
\task $\begin{pmatrix} 1 \\ 1 \\ 5 \end{pmatrix} \wedge
\begin{pmatrix} -1 \\ -2 \\ 2 \end{pmatrix} $
\task $\begin{pmatrix} -5 \\ 2 \\ 1 \end{pmatrix} \wedge
\begin{pmatrix} -1 \\ -2 \\ 3 \end{pmatrix} $
\end{tasks}
\end{exo}


\begin{sol}
\begin{tasks}(1)
\task $\begin{pmatrix} 1 \\ 2 \\ 3 \end{pmatrix} \wedge
\begin{pmatrix} -1 \\ 2 \\ 1 \end{pmatrix} =  \begin{pmatrix} 2-6 \\ -3-1 \\ 2-(-2) \end{pmatrix} =  \begin{pmatrix} -4 \\ -4 \\ 4 \end{pmatrix} $
\task $\begin{pmatrix} -1 \\ 4 \\ 2 \end{pmatrix} \wedge
\begin{pmatrix} 1 \\ 0 \\ 5 \end{pmatrix} =  \begin{pmatrix} 20-0 \\ 2-(-5) \\ 0-4 \end{pmatrix}=  \begin{pmatrix} 20 \\ 7 \\ -4 \end{pmatrix}$
\task*(2) $\begin{pmatrix} 1 \\ 1 \\ 5 \end{pmatrix} \wedge
\begin{pmatrix} -1 \\ -2 \\ 2 \end{pmatrix} =  \begin{pmatrix} 2-(-10) \\ -5-2 \\ -2-(-1) \end{pmatrix}=  \begin{pmatrix} 12 \\ -7 \\ -1 \end{pmatrix}$
\task*(2) $\begin{pmatrix} -5 \\ 2 \\ 1 \end{pmatrix} \wedge
\begin{pmatrix} -1 \\ -2 \\ 3 \end{pmatrix} =  \begin{pmatrix} 6 -(-2)\\ -1-(-15) \\ 10-(-2) \end{pmatrix}=  \begin{pmatrix} 8 \\ 14 \\ 12 \end{pmatrix}$
\end{tasks}
\end{sol}


La surface (l'aire) d'un triangle peut être déterminée comme la moitié de la norme du produit vectoriel des vecteurs formés par deux de ses cotés. 
La surface $S$ du triangle formé par les vecteurs $\vv{C}_1$ et $\vv{C}_2$ est :
%
\begin{equation*}
S = \frac{1}{2} \Vert \vv{C}_1 \wedge \vv{C}_2 \Vert
\end{equation*}

En particulier, l'aire d'un triangle $ABC$ peut se calculer avec l'une des formules suivantes (et bien d'autres).

\begin{equation*}
S = \frac{1}{2} \Vert \vv{AB} \wedge \vv{AC} \Vert = \frac{1}{2} \Vert \vv{BA} \wedge \vv{BC} \Vert
= \frac{1}{2} \Vert \vv{CA} \wedge \vv{CB} \Vert
\end{equation*}


\begin{comment}
Schéma d'interprétation (à faire avec l'enseignant)
% 
\ifthenelse{\equal{\showDrawings}{1}}{
\begin{center}
\includegraphics[height=4cm]{figs/SurfTriangle.png}
\end{center}
}{\vspace*{5cm}}
\end{tcolorbox}

\needspace{8cm}
\end{comment}


\begin{exo}

\noindent \ding{45} Déterminer l'expression \textbf{littérale} de la surface $S$ du triangle $ABC$ sachant que...

\begin{tasks}(1)
\task $\vv{AB} = \vv{U} = \begin{pmatrix} U_x \\ U_y \end{pmatrix}$ et $\vv{BC} = \vv{V} = \begin{pmatrix} V_x \\ V_y \end{pmatrix}$
\task $\vv{OA} = \begin{pmatrix} a_x \\ a_y \end{pmatrix}$ ; $\vv{OB} = \begin{pmatrix} b_x \\ b_y \end{pmatrix}$ ; $\vv{OC} = \begin{pmatrix} c_x \\ c_y \end{pmatrix}$
\task $\vv{AB} = \begin{pmatrix} L \\ 0 \\ 1 \end{pmatrix}$ et $\vv{AC} = \begin{pmatrix} D \cos(\theta) \\ D \sin(\theta) \\ 0 \end{pmatrix}$
\end{tasks}
\end{exo}

\begin{sol}
\begin{tasks}(1)
\task $S = \frac{1}{2} \left\Vert\begin{pmatrix} U_x \\ U_y \end{pmatrix} \wedge \begin{pmatrix} V_x \\ V_y \end{pmatrix} \right\Vert = \frac{1}{2} \left\vert U_x V_y - U_y V_x \right\vert $
\task $S  =  \frac{1}{2} \left\Vert\begin{pmatrix} b_x-a_x \\ b_y-a_y \end{pmatrix} \wedge \begin{pmatrix} c_x-a_x \\ c_y-a_y \end{pmatrix} \right\Vert = \frac{1}{2} \left\vert (b_x-a_x) (c_y-a_y) - (b_y-a_y) (c_x-a_x)\right\vert$
\task $S = \frac{1}{2} \left\Vert \begin{pmatrix} L \\ 0 \\ 1 \end{pmatrix} \wedge \begin{pmatrix} D \cos(\theta) \\ D \sin(\theta) \\ 0 \end{pmatrix} \right\Vert = \left\Vert \begin{pmatrix} -D\sin(\theta) \\ D\cos(\theta) \\ LD\sin(\theta) \end{pmatrix}\right\Vert = \sqrt{ D^2 \big(\cos^2(\theta) + (1 + L^2) \sin^2 (\theta) \big)}$
\end{tasks}
\end{sol}

Rappelons, si cela est utile, que 
%
\begin{equation*}
\vv{a} \cdot \vv{b} = 0 \Rightarrow \text{les vecteur sont orthogonaux}
\end{equation*}
%
\begin{equation*}
\vv{a} \wedge \vv{b} = \vv{0} \Rightarrow \text{les vecteur sont colinéaires}
\end{equation*}

Si les vecteurs $\vv{a}$ et $\vv{b}$ portent des droites, l'application pratique sera de déterminer si ces droites sont perpendiculaires ou parallèles. 



%%%%%%%%%%%%%%%%%%%%%%%%%%%%%%%%%%%%%%%%%%%%%%%%%%%%%%%%%%%%%%%%%%%%%%%%%%%%%%%%
\section{Equation cartésienne d'un plan}


Soit un plan ($P$) de normale $\vv{n} = (n_x,\ n_y,\ n_z)^T$ et passant par le point $A$, de coordonnée $\vv{OA} = (A_x,\ A_y,\ A_z)^T$. Pour qu'un point $M$, de coordonnées cartésiennes $\vv{OM} = (x,\ y,\ z)^T$ appartienne au plan ($P$), il est nécessaire que :
%
\begin{equation*}
\vv{AM} \cdot \vv{n} = 0
\Rightarrow
(\vv{OM}-\vv{OA}) \cdot \vv{n} = 0
\Rightarrow
\begin{pmatrix} x-A_x \\ y-A_y \\ z-A_z \end{pmatrix}
\cdot
\begin{pmatrix} n_x \\ n_y \\ n_z \end{pmatrix} = 0
\end{equation*}
%
Ceci se développe en
%
\begin{equation*}
(x-A_x)n_x + (y-A_y)n_y + (z-A_z)n_z = 0
\end{equation*}
%
\begin{equation*}
\Rightarrow n_x x  + n_y y  + n_z z - (A_x n_x + A_y n_y + A_z n_z) = 0
\end{equation*}
%
On voit donc que l'équation cartésienne d'un plan passant par $A$ et de normale $\vv{n}$ (pas nécessairement de norme unitaire) est :
\begin{equation*}
a x  + b y  + c z + d = 0
\end{equation*}
%
avec $a=n_x$, $b=n_y$, $c=n_z$ et $d=-\vv{OA}\cdot \vv{n}$.



\begin{comment}
Schéma d'interprétation (à faire avec l'enseignant)
%
\ifthenelse{\equal{\showDrawings}{1}}{
\begin{center}
\includegraphics[height=6cm]{figs/eqPlan.png}
\end{center}
}{\vspace*{6cm}}
\end{comment}

\needspace{8cm}
\begin{exo}

\noindent \ding{45} Déterminer l'équation cartésienne du plan...

\begin{tasks}(1)
\task passant par le point $A= \begin{pmatrix} -2 \\ 0 \\ 5\end{pmatrix}$ et de normale $\vv{n} = \begin{pmatrix} 2 \\ -6 \\ 4 \end{pmatrix}$ 
\task passant par les points $A= \begin{pmatrix} 1 \\ -2 \\ 7\end{pmatrix}$, $B= \begin{pmatrix} 2 \\ 2 \\ 1\end{pmatrix}$ et $C= \begin{pmatrix} 1 \\ 1 \\ 5\end{pmatrix}$
\task tangent à une sphère de rayon $3\sqrt{3}$ centrée en $A= \begin{pmatrix} 1 \\ 2 \\ 3\end{pmatrix}$, et de normale $\vv{n} = \begin{pmatrix} 1 \\ 1 \\ -1 \end{pmatrix}$
\task tangent à une sphère de rayon $R$ centrée en $A= \begin{pmatrix} A_x \\ A_y \\ A_z\end{pmatrix}$, et de normale unitaire $\vv{n} = \begin{pmatrix} n_x \\ n_y \\ n_z \end{pmatrix}$
\end{tasks}
\end{exo}


\begin{sol}
\begin{tasks}(1)
\task $d = -\vv{OA}\cdot\vv{n} = -(-4 + 20) = -16$, équation du plan : $2x-6y+4z-16 = 0$, qu'on peut simplifier en $\boxed{x-3y+2z-8 = 0}$
\task $\vv{n} = \vv{AB} \wedge \vv{AC} = \begin{pmatrix} 1 \\ 4 \\ -6\end{pmatrix} \wedge \begin{pmatrix} 0 \\ 3 \\ -2\end{pmatrix} = \begin{pmatrix} 10 \\ 2 \\ 3\end{pmatrix}$, $d = -\vv{OA}\cdot\vv{n} = -(10  -4 + 21)  = -27 $, équation du plan : $\boxed{10x+2y+3z-27 = 0}$
\task Point du plan : $\vv{OA} + 3\sqrt{3}\frac{\vv{n}}{\sqrt{3}} = \begin{pmatrix} 1+3 \\ 2+3 \\ 3-3\end{pmatrix} = \begin{pmatrix} 4 \\ 5 \\ 0\end{pmatrix}$, $d=- \begin{pmatrix} 4 \\ 5 \\ 0\end{pmatrix}\cdot\begin{pmatrix} 1 \\ 1 \\ -1\end{pmatrix}= -(4+5)= -9$, équation du plan : $\boxed{x+y-z-9=0}$
\task Point du plan : $\vv{OA} + R\vv{n}$, $d=-(\vv{OA} + R\vv{n})\cdot\vv{n} = -(\vv{OA}\cdot\vv{n} + R\vv{n}\cdot\vv{n})$, équation du plan : $\boxed{n_x x+n_y y-n_z z -(\vv{OA}\cdot\vv{n} + R\vv{n}^2) =0}$
\end{tasks}
\end{sol}



%%%%%%%%%%%%%%%%%%%%%%%%%%%%%%%%%%%%%%%%%%%%%%%%%%%%%%%%%%%%%%%%%%%%%%%%%%%%%%%%
\needspace{8cm}
\section{Equations paramétrique et cartésienne d'une droite}



Soit une droite ($D$) de vecteur directeur $\vv{d} = (d_x,\ d_y,\ d_z)^T$ et passant par le point $A$, de coordonnée $\vv{OA} = (A_x,\ A_y,\ A_z)^T$. Pour qu'un point $M$, de coordonnées cartésiennes $\vv{OM} = (x,\ y,\ z)^T$ appartienne à la droite ($D$), il est nécessaire que :
%
\begin{equation*}
\vv{AM} = k \vv{d} \quad (\text{avec } k \in \mathbb{R})
\Rightarrow
\begin{pmatrix} x-A_x \\ y-A_y \\ z-A_z \end{pmatrix}
= k
\begin{pmatrix}  d_x \\  d_y \\  d_z \end{pmatrix} 
\Rightarrow
\begin{pmatrix} x\\ y \\ z \end{pmatrix}
=
\begin{pmatrix} A_x + k d_x \\ A_y + k d_y \\ A_z + k d_z \end{pmatrix}
\end{equation*}
%
Notons que le vecteur directeur $\vv{d}$ n'est pas obligatoirement unitaire. 
Les \textbf{équations paramétriques} de cette droite consistent simplement à réécrire ceci sous forme d'un système de trois équations utilisant le même paramètre $k$ :
%
\begin{equation*}
\begin{cases}
x & = A_x + k \times d_x \\
y & = A_y + k \times d_y \\
z & = A_z + k \times d_z
\end{cases}
\quad (k \in \mathbb{R})
\end{equation*}


\begin{comment}
Schéma d'interprétation (à faire avec l'enseignant)
\ifthenelse{\equal{\showDrawings}{1}}{
\begin{center}
\includegraphics[height=4cm]{figs/eqParamDroite.png}
\end{center}
}%
{\vspace*{5cm}}% if not set to 1
\end{comment}


\begin{exo}


\noindent \ding{45} Déterminer les équations paramétriques de la droite...

\begin{tasks}(1)
\task passant par le point $A= \begin{pmatrix} 2 \\ 1 \\ -3\end{pmatrix}$ et de vecteur directeur normal $\vv{u} = \begin{pmatrix} 1 \\ 1 \\ 1 \end{pmatrix}$
\task passant par les points $A= \begin{pmatrix} 2 \\ 0 \\ 1\end{pmatrix}$ et $B= \begin{pmatrix} 0 \\ 3 \\ -1\end{pmatrix}$
\end{tasks}
\end{exo}


\begin{sol}
\begin{tasks}(2)
\task $\begin{cases} x &= 2+k \\ y &= 1+k  \\ z &= -3+k \end{cases} \quad (k \in \mathbb{R})$ 
\task $\begin{cases} x &= 2-2k \\ y &= 0+3k  \\ z &= 1-2k \end{cases} \quad (k \in \mathbb{R})$ 
\end{tasks}
\end{sol}



\needspace{5cm}
En remarquant qu'il existe dans les équations paramétriques d'une droite, une valeur unique du paramètre $k$ pour un point donné sur cette droite, on peut calculer $k$ de trois façons :
%
\begin{equation*}
k = \frac{x - A_x}{d_x} = \frac{y - A_y}{d_y} = \frac{z - A_z}{d_z} 
\end{equation*}
%
Ceci constitue une façon alternative de définir une droite par ses \textbf{équations cartésiennes} :
%
\begin{equation*}
\begin{cases}
\displaystyle \frac{x-A_x}{d_x} & = \displaystyle \frac{y-A_y}{d_y}  \\
\displaystyle \frac{y-A_y}{d_y} & = \displaystyle \frac{z-A_z}{d_z}
\end{cases}
\end{equation*}
%
ou encore :
%
\begin{equation*}
\begin{cases}
d_y \times x - d_x \times y + (d_x \times A_y - d_y \times A_x)  & = 0  \\
d_z \times y - d_y \times z + (d_y \times A_z - d_z \times A_y) & = 0
\end{cases}
\end{equation*}
%
Enfin, il est intéressant d'interpréter l'équation cartésienne d'une droite comme l'intersection de deux plans :
\begin{equation*}
\begin{cases}
n_x \times x + n_y \times y + \text{\colorbox{lightgray}{$0 \times z$}} + d & = 0 \\
\text{\colorbox{lightgray}{$0 \times x$}} + n_y' \times y + n_z' \times z + d' & = 0
\end{cases}
\end{equation*}

%\needspace{10cm}
\begin{comment}
Schéma d'interprétation (à faire avec l'enseignant)
%
\ifthenelse{\equal{\showDrawings}{1}}{
\begin{center}
\includegraphics[height=6cm]{figs/eqCartesienneDroite3D.png}
\end{center}
}{\vspace*{6cm}}
\end{comment}


\begin{exo}

\noindent \ding{45} Déterminer les équations cartésiennes pour chacun des cas de l'exercice précédent rappelés ci-après :

\begin{tasks}(1)
\task passant par le point $A= \begin{pmatrix} 2 \\ 1 \\ -3\end{pmatrix}$ et de vecteur directeur normal $\vv{u} = \begin{pmatrix} 1 \\ 1 \\ 1 \end{pmatrix}$
\task passant par les points $A= \begin{pmatrix} 2 \\ 0 \\ 1\end{pmatrix}$ et $B= \begin{pmatrix} 0 \\ 3 \\ -1\end{pmatrix}$
\end{tasks}

\end{exo}

\begin{sol}
\begin{tasks}(1)
\task $\begin{cases} x &= 2+k \\ y &= 1+k  \\ z &= -3+k \end{cases} \Rightarrow x-2 = y-1 = z+3$.\\ 
La première égalité s'écrit $x-y-1=0$ (en transférant tous les termes vers la gauche) et la seconde égalité se ré-écrit $y-z-4=0$. \\
D'où les équations cartésiennes de la droite : $\boxed{\begin{cases}x-y-1&=0\\y-z-4&=0 \end{cases}}$   
\task $\displaystyle \begin{cases} x &= 2-2k \\ y &= 0+3k  \\ z &= 1-2k \end{cases} \Rightarrow \frac{x-2}{-2} = \frac{y}{3} = \frac{z-1}{-2}$.\\ 
D'où les équations cartésiennes de la droite : 
$\begin{cases}\displaystyle -\frac{x}{2}-\frac{y}{3}+1&=0\\[2mm] \displaystyle \frac{y}{3}+\frac{z}{2}-\frac{1}{2}&=0 \end{cases}$ \\
Il est interessant de simplifier l'écriture de ces équations en multipliant par une valeur bien choisie :\\[2mm]
$\begin{cases}\displaystyle -6\times \left( -\frac{x}{2}-\frac{y}{3}+1\right) &= -6\times0  \\[2mm] \displaystyle 6\times \left(\frac{y}{3}+\frac{z}{2}-\frac{1}{2}\right)&=6\times0 \end{cases} \Rightarrow \boxed{\begin{cases} 3x +2y -6 &=0\\ 2y + 3z -3 &=0 \end{cases}}$ 
\end{tasks}
\end{sol}




%%%%%%%%%%%%%%%%%%%%%%%%%%%%%%%%%%%%%%%%%%%%%%%%%%%%%%%%%%%%%%%%%%%%%%%%%%%%%%%%
\section{Intersections :\\ plan~$\cap$~plan (droite) et droite~$\cap$~plan (point)}


L'intersection de deux plans ($P$) et ($P'$) en 3D est une droite ($D$) si les plans se coupent et ne sont donc pas parallèles (c'est-à-dire que leurs normales ne sont pas colinéaires). Les coordonnées $x$, $y$ et $z$ des points de la droite d'intersection ($D$) appartiennent à la fois au plan ($P$) et au plan ($P'$). Elles satisfont donc aux deux équations cartésiennes de ces plans :
%
$$
\begin{cases}
a x + b y + c z + d  & = 0  \\
a' x + b' y + c' z + d'  & = 0 
\end{cases}
$$
%
Il s'agit ici d'un système de deux équations avec trois inconnues.
La méthode pour trouver les équations paramétriques de la droite ($D$) consiste à poser arbitrairement $x=k$ (tout autre choix comme $z=k$ serait aussi valable). Ceci ajoute une équation au système précédent, et en remplaçant $x$ par $k$ le premier système à deux équations n'aura plus que deux inconnues (ici $y$ et $z$) :   
%
$$
\begin{cases}
x & = k \\
\text{et} &
\begin{cases}
a x + b y + c z + d  & = 0  \\
a' x + b' y + c' z + d'  & = 0 
\end{cases}
\end{cases}
\quad\Rightarrow\quad
\begin{cases}
x & = k \\
\text{et} &
\begin{cases}
b y + c z   & = -d - a k  \\
b' y + c' z   & = -d' - a' k 
\end{cases}
\end{cases}
$$
%
En résolvant ce système d'équations avec la méthode de son choix, il devient possible de trouver $y$ et $z$, et de se ramener à un système d'équations de la forme suivante : 
%
$$
\begin{cases}
x & = 0 + 1 \times k \\
y & = A_y + d_y \times k  \\
z & = A_z + d_z \times k 
\end{cases}
\quad (k \in \mathbb{R})
$$
%
Il s'agit là des équations paramétriques de la droite d'intersection ($D$).


\begin{comment}
Schéma d'interprétation (à faire avec l'enseignant)
%
\ifthenelse{\equal{\showDrawings}{1}}{
\begin{center}
\includegraphics[height=6cm]{figs/intersectionPlans.png}
\end{center}
}{\vspace*{6cm}} 
\end{comment}


\begin{exo}

\noindent \ding{45} Déterminer les équations paramétriques de la droite, intersection des plans $P_1$ et $P_2$.

\begin{tasks}(1)
\task $P_1 : x+y+z=0\quad$ et $\quad P_2 : 2x+3y+z-4 = 0$
\task $P_1 : x-3y+2z=5\quad$ et $\quad P_2 : 2x+y+7z = 1$
\task $P_1 : x+y+z+1=0\quad$ et $\quad P_2 : 2x-y-z+3 = 1$
\end{tasks}

\end{exo}

\begin{sol}
\begin{tasks}(1)
\task $P_1 : x+y+z=0\quad$ et $\quad P_2 : 2x+3y+z-4 = 0$
$$
\begin{cases}
z & = k \\
\text{et} &
\begin{cases}
x+y+z  & = 0  \\
2x+3y+z-4  & = 0 
\end{cases}
\end{cases}
\Rightarrow
\begin{cases}
z & = k \\
\text{et} &
\begin{cases}
x+y   &= -k  \\
2x+3y   &= 4-k 
\end{cases}
\end{cases}
$$
$$
\Rightarrow
\begin{cases}
z & = k \\
\text{et} &
\begin{cases}
x   & = -k - y  \\
-2(k+y)+3y  & = 4-k 
\end{cases}
\end{cases}
\Rightarrow
\boxed{
\begin{cases}
x   &= -4 -2k  \\
y   &= 4 + k \\
z & = k 
\end{cases}
}
$$

\task $P_1 : x-3y+2z=5\quad$ et $\quad P_2 : 2x+y+7z = 1$
$$
\begin{cases}
x & = k \\
\text{et} &
\begin{cases}
k-3y+2z & = 5  \\
2k+y+7z  & = 1 
\end{cases}
\end{cases}
\Rightarrow
\begin{cases}
x & = k \\
\text{et} &
\begin{cases}
k-3y+2z   &= 5 - k  \\
y+7z   &= 1 - 2k 
\end{cases}
\end{cases}
$$
$$
\Rightarrow
\begin{cases}
x & = k \\
\text{et} &
\begin{cases}
y   & = -\frac{5}{3} + \frac{2}{3}(k+z) \\
z   & =  -\frac{7}{23}k  + \frac{8}{23}
\end{cases}
\end{cases}
\Rightarrow
\boxed{
\begin{cases}
x   &= k  \\
y   &= -\frac{33}{23} + \frac{3}{23}k  \\
z   &=  \frac{8}{23}  -\frac{7}{23}k  
\end{cases}
}
$$

\task $P_1 : x+y+z+1=0\quad$ et $\quad P_2 : 2x-y-z+3 = 1$
$$
\begin{cases}
y & = k \\
\text{et} &
\begin{cases}
x+k+z+1  & = 0  \\
2x-k-z+2  & = 0 
\end{cases}
\end{cases}
\Rightarrow
\begin{cases}
y & = k \\
\text{et} &
\begin{cases}
x   &= -k-1-z  \\
z   &= -k 
\end{cases}
\end{cases}
$$
$$
\Rightarrow
\begin{cases}
y & = k \\
\text{et} &
\begin{cases}
x  & = -1  \\
z  & = -k 
\end{cases}
\end{cases}
\Rightarrow
\boxed{
\begin{cases}
x   &= -1  \\
y   &= k \\
z & = -k 
\end{cases}
}
$$

\end{tasks}
\end{sol}



L'intersection d'un plan ($P$) et d'une droite ($D$) en 3D est un point si le plan et la droite se coupent (c'est-à-dire que la normale au plan et le vecteur directeur de la droite ne sont pas orthogonaux). Les coordonnées $x$, $y$ et $z$ du point d'intersection appartiennent à la fois au plan ($P$) et à la droite ($D$). Elles satisfont donc aux équations suivantes, respectivement pour la droite et le plan :
%
$$
\begin{cases}
x & = A_x + d_x \times k \\
y & = A_y + d_y \times k  \\
z & = A_z + d_z \times k 
\end{cases}
\quad (k \in \mathbb{R})
\qquad \text{et} \qquad
a x + b y + c z + d  = 0
$$
%
En insérant les équations paramétriques de la droite ($D$) dans l'équation cartésienne du plan ($P$), on obtient :
%
$$
a (A_x + d_x \times k) + b (A_y + d_y \times k) + c (A_z + d_z \times k) + d  = 0
$$
%
où la seule inconnue est le paramètre $k$ qui peut alors être déterminé :
%
$$
(a d_x + b d_y + c d_z )k + (a A_x + b A_y + c A_z + d) = 0 
$$
%
$$
\Rightarrow k = \frac{-(a A_x + b A_y + c A_z + d)}{(a d_x + b d_y + c d_z )} 
$$

% un mot pour dire qu'il ne faut pas obligatoirement prendre cette relation comme une formule, mais plutôt comprendre comment y arriver. Dans la pratique, ce sera plus simple

Une fois que la valeur de $k$ est trouvée, il suffit de l'utiliser dans les équations paramétriques de la droite pour finaliser le calcul des composante $x$, $y$ et $z$ du point d'intersection.


\begin{comment}
Schéma d'interprétation (à faire avec l'enseignant)
%
\ifthenelse{\equal{\showDrawings}{1}}{
\begin{center}
\includegraphics[height=6cm]{figs/intersectionDroitePlan.png}
\end{center}
}{\vspace*{6cm}} 
\end{comment}


\begin{exo}

\noindent \ding{45} Déterminer le point d'intersection de la droite $\Delta$ et du plan $P$

\begin{tasks}(1)
\task Droite $\Delta$ passant par $(1\ 2\ 3)^T$ de vecteur directeur $(0\ 1\ -1)^T$ et plan $P$ d'équation $z -2 = 0$
\task Droite $\Delta$ passant par $(0\ 0\ 0)^T$ de vecteur directeur $(1\ 1\ 2)^T$ et plan $P$ d'équation $3x+y-z -6= 0$
\end{tasks}
\end{exo}

\begin{sol}
\begin{tasks}(1)
\task Droite $\Delta$ passant par $(1\ 2\ 3)^T$ de vecteur directeur $(0\ 1\ -1)^T$ et plan $P$ d'équation $z -2 = 0$
$$
\begin{cases}
x & = 1 \\
y & = 2 +  k  \\
z & = 3 - k 
\end{cases}
\quad (k \in \mathbb{R})
\qquad \text{et} \qquad
z - 2  = 0
$$
$$
\Rightarrow (3-k) -2 = 0 \Rightarrow \boxed{k = 1} \Rightarrow \Delta \cap P = \begin{pmatrix} 1 \\ 3 \\ 2 \end{pmatrix}
$$

\task Droite $\Delta$ passant par $(0\ 0\ 0)^T$ de vecteur directeur $(1\ 1\ 2)^T$ et plan $P$ d'équation $3x+y-z -6 = 0$
$$
\begin{cases}
x & = k \\
y & = k  \\
z & = 2k 
\end{cases}
\quad (k \in \mathbb{R})
\qquad \text{et} \qquad
3x+y-z - 6=0
$$
$$
\Rightarrow 3k+k-2k = 6 \Rightarrow 2k = 6 \Rightarrow \boxed{k = 3} \Rightarrow \Delta \cap P = \begin{pmatrix} 3 \\ 3 \\ 6 \end{pmatrix}
$$

\end{tasks}
\end{sol}



